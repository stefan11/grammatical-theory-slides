%% -*- coding:utf-8 -*-

\subtitle{Government \& Binding}

\section{Government \& Binding (GB)}


\huberlintitlepage[22pt]


\outline{

% \begin{itemize}
% \item Begriffe
% \item Phrasenstrukturgrammatiken
% \item Government \& Binding (GB)
% \item Generalisierte Phrasenstrukturgrammatik (GPSG)
% \item Lexikalisch-Funktionale Grammatik (LFG)
% %\item Lexical Mapping Theory (LMT)
% %\item PATR
% \item Kategorialgrammatik (CG)
% \item Kopfgesteuerte Phrasenstrukturgrammatik (HPSG)
% %\item Konstruktionsgrammatik (CxG)
% \item Baumadjunktionsgrammatik (TAG)
% \end{itemize}

\tableofcontents
}

\frame{
\frametitle{Reading material}

\citew[Section~3.1--3.3]{MuellerGT-Eng}


}


\subsection{History and motivation}


\frame{
\frametitle{Phrase structure grammars and natural language}

Chomsky: generlizations cannot be captured with PSGs (\eg active/passive alternations) $\to$
      transformations:

\begin{table}[H]
\begin{tabular}{@{}l@{~}l@{~}l@{~}l}
NP& V &NP &$\to$ 3 [\sub{AUX} be] 2en [\sub{PP} [\sub{P} by] 1]\\
1 & 2 &3\\
\end{tabular}
\end{table}

\eal
\ex John loves Mary.
\ex Mary is loved by John.
\zl

A tree with the sequence of symbols on the left-hand site is mapped to a tree with the sequence of
symbols on the right-hand side.

}


\frame{
\frametitle{Transformation of an active tree into a passive tree}

\vfill

%\oneline{%
\hfill
\begin{forest}
%sm edges
[S, for tree={parent anchor=south, child anchor=north}
  [NP [Kim] ]
  [VP
    [V [loves] ]
    [NP [Sandy] ] 
  ]]
\end{forest}
\hspace{1em}
\raisebox{6\baselineskip}{$\leadsto$}
\hspace{1em}
  \begin{forest}
  %sm edges
  [S, for tree={parent anchor=south, child anchor=north}
  	[NP[Sandy]]
	[VP
	[Aux[is, tier=word]]
	[V[loved, tier=word]]
	[PP
	[P[by, tier=word]]
	[NP[Kim, tier=word]]]]]
\end{forest}
\hfill\mbox{}
%}

\vfill

\begin{tabular}{@{}l@{~}l@{~}l@{~}l}
NP& V &NP &$\to$ 3 [\sub{AUX} be] 2en [\sub{PP} [\sub{P} by] 1]\\
1 & 2 &3\\
\end{tabular}

\vfill

}


\frame{
\frametitle{Complexity, transformations and natural languages}

\small
\begin{itemize}
\item There are different complexity levels for phrase structure grammars.\\(Chomsky Hierarchy, Type 3--0)
\pause
\item What we saw so far are so called context free grammars. They are of type 2.
\pause
\item Maximal level (type 0) is too powerful for human langauges.\\
$\to$ Researchers wanted to be more restrictive.
\pause
\item Grammars with general transformations correspond to PSGs with type 0 complexity \citep{PR73a-u}.
\pause
\item Transformations are not sufficiently restricted,\\Interactions are not tractable,\\
      There have been problems with transformations deleting material (see \citew{Klenk2003a})
\pause
\item $\to$ new theoretical approaches, Government \& Binding \citep{Chomsky81a}: restrictions for the form of grammar
  rules, elements can be connected to the position in a tree they were coming from, general
  principles to restrict the power of transformations

\end{itemize}

}

\settowidth\jamwidth{(Japanese)}
\frame{
\frametitlefit{Hypothesis regarding language acquisition: Principles \& Paramaters}

\begin{itemize}
\item<+-> Some of our linguistics knowledge is innate.\\
          (Not all linguists agree with this assumption! Discussion: \citew{MuellerGT-Eng})
\item<+-> Principles all linguistic structures have to obey
\item<+-> These principles are parametrized $\to$ there is choice\\
      A parameter may be set differently for different languages.

\medskip
\pause
      Example: \\

Principle: A head is placed before or after its complements depending on the value of the parameter \textsc{position}.


\medskip
%% \begin{tabular}[t]{@{}l@{ }l@{}}
%%                 Englisch  &$\to$ Verb steht vor Komplementen\\
%%                 Japanisch &$\to$ Verb steht nach Komplementen\\
%%                 \end{tabular}


\eal
\ex  be showing pictures of himself \jambox{(English)}
\ex
\gll zibun         -no syasin-o mise-te iru\\
     \textsc{self} of  picture  showing be\\\jambox{(Japanese)}
\zl

\end{itemize}

}

\lecture{T-Modell}{t-modell-lec}

\subsection{The T-model}



\frame{
\frametitle{Deep and Surface Structure}

\begin{itemize}[<+->]
\item Chomsky claimed that simple PSGs cannot capture certain regularities.\\
      \eg the relation between active and passive sentences.

\item Therefore he assumes an underlying structure,\\
      the so-called \blaubf{Deep Structure}.

\item A structure can be mapped onto another structure.

Parts may be deleted or moved to other positions in trees in such mappings.

As a result of such transformations a new structure is derived, the so-called \blaubf{Surface Structure}.

\medskip
\begin{tabular}{@{}l@{ = }l@{}}
\emph{Surface Structure} & S Structure\\
\emph{Deep Structure} & D Structure\\
\end{tabular}
\end{itemize}

}

%\beamertemplatebackfindforwardnavigationsymbolshorizontal


\frame[label=t-modell]{
\frametitle{The T-model}


%% \centerline{%
%% \resizebox{0.8\linewidth}{!}{
%% \begin{tabular}{@{}ccc@{}}
%% \xbar-Theorie der \\
%% \node{psr}{Phrasenstrukturregeln} & & \hyperlink{lexikon}{\node{lex}{Lexikon}}\\[6ex]
%% &\hyperlink{ds}{\node{ds}{D-Struktur}}\\[2ex]
%% \mc{3}{@{}c@{}}{\hyperlink{move-alpha}{move-$\alpha$}}\\[4ex]
%% &\hyperlink{ss}{\node{ss}{S-Struktur}}\\[6ex]
%% \node{tilg}{Tilgungsregeln},             && \node{ana}{Regeln des anaphorischen Bezugs,}\\
%% \node{filter}{Filter, phonol.\ Regeln}  && \node{quant}{der Quantifizierung und der Kontrolle}\\[6ex]
%% \hyperlink{pf}{\node{pf}{Phonetische}}             && \hyperlink{lf}{\node{lf}{Logische}}\\
%% \hyperlink{pf}{Form (PF)}               && \hyperlink{lf}{Form (LF)}\\
%% \end{tabular}
%% \anodeconnect{psr}{ds}\anodeconnect{lex}{ds}
%% \anodeconnect{ds}{ss}
%% \anodeconnect{ss}{tilg}\anodeconnect{ss}{ana}
%% \anodeconnect{filter}{pf}\anodeconnect{quant}{lf}
%% }}

\vfill
\centerline{%
\begin{forest}
for tree = {edge={->},l=4\baselineskip}
[D-structure
     [S-structure,edge label={node[midway,right]{move $\alpha$}} 
            [Deletion rules{,}\\Filter{,} phonol.\ rules
                    [Phonetic\\Form (PF)]]
            [Anaphoric rules{,}\\rules of quantification and control
                    [Logical\\Form (LF)]]]]
\end{forest}}

\vfill

}



\gotobuttonleft{t-modell}{T-model}

\frame[label=lexikon]{
\frametitle{The T-model: The lexicon}
\showsingleitemframe
\savespace
\begin{itemize}
\item<+> Contains a lexical entry for every word with information about:
\begin{itemize}\itemsep0pt
\item morphophonological structure
\item syntactic features
\item valence frame
\item \ldots
\end{itemize}
Contains list for word forms and morphemes and morphology component

\item<+> The lexicon is the interface between syntax and semantic interpretation of word forms.

\item<+> Vocabulary is not determined by UG (not innate),\\
just structural conditions are determined by UG.\\
(assumption not shared by all linguists)

\item<+> Morphosyntactic features (\eg gender) are not pre-determined:\\
Universal grammar provides a toolbox (claim not falsifiable).
% und setzt einige Minimalanforderungen.
\end{itemize}

}

\frame{
\frametitle{The T modell: D Structure, Move-$\alpha$ and S Structurr (I)}

\begin{itemize}
\item<+-> Phrase structure $\to$\\
We can describe relations between constituents.

\item<+-> A certain format for rules is given (\xbar-Schema).

Lexicon + structures of \xbar syntax = base for D Structure

\hypertarget{ds}{D Structure} = syntactic representation of valence frames of particular words as determined in the lexicon.
\end{itemize}


}

\frame{
\frametitle{The T-modell: D Structure, Move-$\alpha$ and S Structure (II)}

\begin{itemize}
\item<+-> constituents may be appearing at different places at the surface than the one determined
  by the valence frame:
\eal
\ex 
\gll {}[dass] der Mann dem Kind das Buch gibt\\
     {}\spacebr{}that the.\NOM{} man the.\DAT{} woman the.\ACC{} book gives\\
\glt `that the man gives the woman the book'
\ex 
\gll Gibt der Mann dem Kind das Buch?\\
	 gives the.\NOM{} man the.\DAT{} woman the.\ACC{} book\\
\glt `Does the man give the woman the book?'
\ex 
\gll Der Mann gibt dem Kind das Buch.\\
	 the.\NOM{} man gives the.\DAT{} woman the.\ACC{} book\\
\glt `The man gives the woman the book.'
\zl
\item<+-> therefore transformational rules for reordering:\\
\hypertarget{move-alpha}{Move $\alpha$} = "`Move anything anywhere!"'

What exactly can be moved where and for which reason is determined
by principles.

\end{itemize}


}

\frame[label=ss,shrink=5]{
\frametitle{The T-modell: D Structure, Move-$\alpha$ and S Structure (III)}

\begin{itemize}
\item Relations between predicates and their arguments as determined by lexical entries must be
  recoverable on all representational levels for semantic interpretation.

\item
$\to$ Starting place of moved elements is marked with traces.

\eal
\ex 
\gll {}[dass] der Mann dem Kind das Buch gibt\\
	 {}\spacebr{}that the man the woman the book gives\\
\glt `that the man gives the woman the book'
\ex 
\gll Gibt$_i$ der Mann dem Kind das Buch \_$_i$?\\
	 gives the man the woman the book\\
\glt `Does the man give the woman the book?'
\ex 
\gll {}[Der Mann]$_j$ gibt$_i$ \_$_j$ dem Kind das Buch \_$_i$.\\
	 {}\spacebr{}the man gives {} the woman the book\\
\glt `The man gives the woman the book.'
\zl

Different traces are marked by indices.\\
Sometimes also \emph{e} for empty element and \emph{t} for trace.

\item
\hypertarget{ss}{S Structure} is a surface"=like structure but should not be equated with the structure of actual utterances.
\end{itemize}
}


\frame[label=pf]{
\frametitle{The T-model: Phonetic Form}

PF is the phonetic form of a sentence, the string of phonemes that are actually pronounced. 

The mapping from S Structure to PF incorporates the phonological laws.


\pause
Example: \emph{wanna} contraction

\eal
\ex The students want to visit Paris.
\ex The students wanna visit Paris.
\zl
The contratcion in (\mex{0}) is licenced by the optional rule in (\mex{1}):
\ea
want + to $\to$ wanna
\z


}

\frame[label=lf]{
\frametitle{The T-model: Logical Form (I)}

\begin{itemize}
\item Logical Form is a syntactic level mediating between S Structure and semantic interpretation of
  a sentence.

anaphoric reference (binding): what can pronouns refer to?
\eal
\ex 
\gll Peter kauft einen Tisch. Er gefällt ihm.\\
	 Peter buys a table(\mas) he likes him\\
\glt `Peter is buying a table. He likes it/him.'
\ex 
\gll Peter kauft eine Tasche. Er gefällt ihm.\\
	 Peter buys a bag(\fem) he likes him\\
\glt `Peter is buying a bag. He likes it/him.'
\ex 
\gll Peter kauft eine Tasche. Er gefällt sich.\\
	 Peter buys a bag(\fem) he likes himself\\
\glt `Peter is buying a bag. He likes himself.'
\zl
\end{itemize}

}


\frame{
\frametitle{The T-model: Logical Form (II)}

\begin{itemize}
\item Quantification:
\ea
Every man loves a woman.
\z
$\forall x \exists y (man(x) \to (woman(y) \wedge love(x,y))$\\
$\exists y \forall x (man(x) \to (woman(y) \wedge love(x,y))$

\item Some accounts try to derive the readings via movement of quantifiers in trees \citep{May85a-u}.
\end{itemize}



}


\frame{
\frametitle{The T-model: Logical Form (III)}
\smallframe

Control theory:\\
How is the semantic role of the subject of the infinitive filled?


\eal
\ex 
\gll Der Professor schlägt dem Studenten vor, die Klausur noch mal zu~~~~~~ schreiben.\\
     the professor suggests the student \textsc{part} the test once again to write\\
\glt `The professor advises the student to take the test again.'
\ex 
\gll Der Professor schlägt dem Studenten vor, die Klausur nicht zu bewerten.\\
	 the professor suggests the student \textsc{part} the test not to grade\\
\glt `The professor suggests to the student not to grade the test.'
\ex 
\gll Der Professor schlägt dem Studenten vor, gemeinsam ins Kino zu gehen.\hspace{-3pt}\\
	 the professor suggests the student \textsc{part} together into cinema to go\\
\glt `The professor suggests to the student to go to the cinema together.'
\zl

}

%\beamertemplatenavigationsymbolsempty 
\mode<beamer>{\beamertemplatebackfindforwardnavigationsymbolshorizontal}

\lecture{Lexicon}{lexicon}

\subsection{The lexicon}

\frame{
\frametitle{Lexicon: Basic terminology (I)}

\small
\begin{itemize}
\item meaning of words $\to$ combinatoric potential with certain semantic roles\\ (``acting person'' or ``affected thing'')

Example: meaning representation of (\mex{1}a) is (\mex{1}b):
\eal
\ex Judit beats the grandmaster.
\ex \relation{beat}(x,y)
\zl
\pause
\item This is subsumed under the terms \alert{valency} and \alert{selection}.

Note:\\
Semantic valence may differ from syntactic valence! (see \citealp[Section~1.6]{MuellerGT-Eng})
\pause
\item Another term is \alert{subcategorization}:

\emph{beat} is subcategorized for a subject and an object.

The word \emph{subcategorize} somehow developed its own life:\\
\emph{X subcategorizes for Y} is used for \emph{X selects Y}.
\end{itemize}


}


\frame{
\frametitle{Lexicon: Basic terminology (II)}
\savespace
\begin{itemize}
\item \emph{beat} is also called the \blaubf{predicate}\\
(since \relation{beat} is the logical predicate).

\pause
\item Subject and object are \alert{arguments} of the predicate.
\pause
\item Several terms for selectional requirement (some semantic, some syntactic, some
  mixed):  
\alert{argument structure}, \alert{valence frame}, \alert{subcategorization frame}, \alert{thematic grid}
and \alert{theta"=grid} or \alert{$\theta$-grid}
\pause
\item \alert{Adjuncts} modify semantic predicates.\\
If semantic aspects are discussed, the term is \alert{modifier}.\\
Adjuncts are not listed as part of valence frames.
\end{itemize}



}



\frame{
\frametitle{The Theta"=Criterion}

\label{theta-kriterium}%
Arguments are placed into certain positions in the clause (argument positions).

\alert{Theta"=Criterion} \citep[\page 36]{Chomsky81a}:
\begin{itemize}
\item Each theta"=role is assigned to exactly one argument position.
\item Every phrase in an argument position receives exactly one theta"=role.
\end{itemize}


}


\frame{
\frametitle{External argument and internal arguments}

\savespace
\begin{itemize}[<+->]
\item Arguments are ordered: there are higher- and lower"=ranked arguments

\item The highest"=ranked argument of verbs and adjectives has a special status. 

It is often (and always in some languages) realized in a position
outside of the verb or adjective phrase, it is called the \alert{\hypertarget{ext-arg}{external argument}}. 


\item The remaining arguments occur in positions inside of the VP or AP.

Term: \alert{internal argument} or \alert{complement}

\item For simple sentences: external argument = subject.
\end{itemize}

}

\frame{
\frametitle{Theta roles}
%\savespace

\begin{itemize}
\item<+->There are three classes of theta"=roles.
\item<+->Class 1 is usually the highest role, class 3 the lowest.
\begin{itemize}
\item<+-> Class 1: \alert{agent} (acting individual), the cause of an action or feeling (stimulus), holder of a certain property
\item<+-> Class 2: \alert{experiencer} (perceiving individual), the person profiting from something (\alert{beneficiary})
(or the opposite: the person affected by some kind of damage), \alert{possessor} (owner or soon"=to"=be owner of something, or the opposite:
someone who has lost or is lacking something) 
\item<+-> Class 3: \alert{patient} (affected person or thing), \alert{theme}
\end{itemize}

\item<+-> Caution!\\
Rather inconsistent assignment of roles by different authors. Proto-roles a la \citet{Dowty91a} may
be the only feasible way to deal with the problem.
\nocite{Gruber65a-u,Fillmore68,Fillmore71a-u,Jackendoff72a-u,Dowty91a}
\end{itemize}
}


\frame{
\frametitle{A lexical entry (I)}

Which information do we need to use a word appropriately?

Answer: The mental lexicon contains \alert{lexical entries} with the specific properties of
syntactic words needed to use that word grammatically.

Some of these properties are the following:
\begin{itemize}
\item form
\item meaning (semantics)
\item grammatical features: syntactic word class $+$ morphosyntactic features   
\item theta"=grid
\end{itemize}



}

\frame{
\frametitle{A lexical entry (II)}


{\footnotesize
\begin{tabular}[t]{|l|ll|}
\hline
form     & \emph{helf-} `help'&\\\hline
semantics & \relation{helfen}     &\\\hline
grammatical features  & \multicolumn{2}{l|}{verb}\\\hline\hline
%                       & \multicolumn{2}{l|}{3rd person singular indicative present active}\\\hline\hline

%\setlength{\arrayrulewidth}{9pt}
theta"=grid                &&\\\hline
theta"=roles                & \underline{agent} & beneficiary\\[2mm]\hline
grammatical particularities &                   & dative\\\hline
\end{tabular}
}

\bigskip
Arguments are ordered according to their ranking:\\
the highest argument is furthest left. 

In this case, the highest argument is the external argument.

The external argument is underlined.


}



\lecture{\xbar Theory}{x-bar}
\subsection{\xbar Theory}




\frame{

\frametitle{Comment on distribution of \xbar rules}

\xbar Theory is assumed in many other frameworks as well:\\
\begin{itemize}
%\item Government \& Binding (GB):\\\citew*{Chomsky93a}
\item Lexical Functional Grammar (LFG):\\\citew{Bresnan82a-ed,Bresnan2001a,BF96a-ed,Berman2003a}
\item Generalized Phrase Structure Grammar (GPSG):\\ \citew*{GKPS85a}
\end{itemize}

Sometimes different categories are assuemd.\\
In particular so-called functional categories (\eg INFL).

No assumptions about universality and innateness are made in most other theories.

}

\subsubsection{Heads}


\frame{
\frametitle{\xbar Theory: Heads}
\pause
Head determines the most important properties of a phrase.
\eal
\ex 
\gll Kim \blaubf{schläft}.\\
     Kim sleeps\\
\ex 
\gll Kim \blaubf{mag} Sandy.\\
     Kim likes Sandy\\
\ex 
\gll \blaubf{in} diesem Haus\\
     in this  house\\
\ex 
\gll ein \blaubf{Haus}\\
     a blue house\\
\zl


}

\subsubsection{Lexical categories}

\frame{
\frametitle{\xbar Theory: Lexical categories}
\label{slide-lex-kat-gb}%

categories are divided into lexical and functional categories\\ 
($\approx$ correlates roughly with the difference between open and closed word classes)

Lexical categories: 
\begin{itemize}
\item V = verb
\item N = noun
\item A = adjective
\item P = preposition
\item Adv = adverb
\end{itemize}
% Abney87a:64--65 Meinunger2000a:38--39

}


\frame{
\frametitle{\xbar Theory: Lexical categories (cross classification)}
%\label{slide-lex-kat-gb}%

Attempt to use binary features to cross-classify lexical categories:

\bigskip

\centerline{\renewcommand{\arraystretch}{1.5}
\begin{tabular}{|c|c|c|}\hline
 & $-$ V & + V \\\hline
$-$ N & P = [ $-$ N, $-$ V] &  V = [ $-$ N, + V] \\\hline
  + N & N = [+ N, $-$ V]    &  A = [+ N, + V]\\\hline
\end{tabular}
}
\pause

\bigskip
Cross classification $\to$ simple way to refer to adjectives and verbs:\\
all lexical categories that are [ + V] are either verbs or adjectives.

Generalizations are possible \eg: [ + N] categories may bear case
% [ $-$ N]-Kategorien können Kasus regieren (allerdings auch A)
\medskip

Note: Adverbs can be treated as prepositions not selecting an argument.\nocite{Chomsky70a}

}



\frame[shrink]{
\frametitle{Head position dependent on the decomposed category?}

Nouns and prepositions are head-initial:

\eal
\ex
\gll \alert{für} Marie\\
	 for Marie\\
\ex 
\gll \alert{Bild} von Maria\\
	 picture of Maria\\
\zl

Adjectives and verbs are head-final:
\eal
\ex 
\gll dem König \alert{treu}\\
     the king loyal\\
\glt `Loyal to the king'
\ex 
\gll der [dem Kind \alert{helfende}] Mann\\
     the the child helping man\\
\glt `the man helping the child'
\ex 
\gll dem Mann \alert{helfen}\\
     the man help\\
\glt `help the man'
\zl


}

\frame{
\frametitle{Head position dependent on the decomposed category? (II)}

$\to$ \begin{tabular}[t]{@{}l@{}}
      {}[+ V] $\equiv$ head-final\\
      {}[$-$ V] $\equiv$ head-initial\\
      \end{tabular}

\pause
Problem: postpositions (P = [$-$ V])
\eal
\ex
\gll des Geldes \alert{wegen}\\
     the money because\\
\glt `because of the money'
\ex 
\gll die Nacht \alert{über}\\
     the night during\\
\glt `during the night'
\zl
\pause

Assume a new feature with binary value?\\
But then we would get four new categories in total.\\
But we need only one.

So, maybe this binary encoding is not such a good idea after all.



}



\subsubsection{Functional categories}

\frame{
\frametitle{\xbar Theory: Functional categories}

No cross-classification:\bigskip


\begin{tabular}{lp{\linewidth}@{}}
C   & Complementizer (subordinating conjunctions such as \emph{dass} `that')\\
I   & Finiteness (as well as Tense and Mood);\\
    & also Infl in earlier work (inflection),\\
    & T in more recent work (Tense) \\
D   & Determiner (article, demonstrative)\\
\end{tabular}


}



\subsubsection{Assumptions and rules}


\frame{
\frametitle{\xbar Theory: Assumptions}

\begin{itemize}
\item \alert{Endocentricity}:\\
Every phrase has a head and every head is part of a phrase.

more technically: every head projects to a phrase. 


\pause
\item \alert{Binary branching} (predominant assumption today):\nocite{Kayne84a-u}\\
Non-terminal nodes are binary branching,\\
that is, there are no teneray branching nodes or nodes with more daughters.
\pause
\item \alert{Non-Tangling Condition}:\\
The branches of tree structures cannot cross.\\

\end{itemize}

}


\subsection{German clause structure}

%\if 0


\subsubsection{Excursus: The English CP and IP}

\frame[shrink]{
\frametitlefit{The structure of the German clause: Excursus: The English CP and IP}


\begin{itemize}
\item In early work the following rules were assumed for English:
\eal
\ex S $\to$ NP VP
\ex S $\to$ NP Infl VP
\zl

\begin{forest}
sm edges
[S
  [NP
  	[Ann,roof]]
  [INFL
  	[will]]
  [VP
  	[V$'$
		[V$^0$[read]]
		[NP[the newspaper, roof]]]]]
\end{forest}

\pause
\item These rules do not adhere to the \xbar schema.


\end{itemize}

}



\subsubsubsection{C and I}



\subsubsubsection{The English IP and VP}

\frame{
\frametitle{Excursus: The English IP and VP: Auxiliaries}

\savespace\small\parskip0pt

\medskip
\vfill

\centerline{
\scalebox{.9}{%
\begin{forest}
sm edges
[IP
  [NP
  	[Ann,roof]]
  [I$'$
  	[I$^0$
  		[will]]
	[VP
  	[V$'$
		[V$^0$[read]]
		[NP[the newspaper, roof]]]]]]
\end{forest}}}


\begin{itemize}
\item Instead of earlier approaches: INFL as head, INFL selecting a VP as complement.
\item Auxiliaries are placed in \inull (=~Aux).
\item Sentential adverbs may be placed between auxiliary and main verb.
\end{itemize}

}


\frame[shrink]{
\frametitle{Excursus: The English IP and VP: Clauses w/o auxiliary}

\savespace\small\parskip0pt

\centerline{%
\scalebox{.9}{%
\begin{forest}
sm edges
[IP
	[NP[Ann,roof]]
	[I$'$
		[I$^0$[-s]]
		[VP
			[V$'$
				[V$^0$[read-]]
				[NP[the newspaper, roof]]]]]]
\end{forest}}
}

\begin{itemize}
\item Auxiliaries are placed in \inull (=~Aux).
\item Position may contain the inflectional affix. The finite verb moves there.

(Various variants of the theory \ldots. Some assume lowering of the affix, some assume an empty I
position and connection to the finite verb. For German, the best version seems to be to not assume I
at all \citep{Haider93a,Haider97a}.)
\end{itemize}

}


\frame{
\frametitle{English clauses with complementizer}



\hfill\scalebox{0.73}{%
\begin{forest}
sm edges
[CP
[C$'$
	[C$^0$[that]]
	[IP
		[NP[Ann,roof]]
		[I$'$
			[I$^0$[will]]
			[VP
				[V$'$
					[V$^0$[read]]
					[NP[the newspaper, roof]]]]]]]]
\end{forest}}\hfill\hfill\mbox{}

\begin{itemize}
\item The complementizer (\emph{that}, \emph{because}, \ldots) requires an IP.
\end{itemize}

}

\subsubsubsection{The English CP, IP and VP}

\frame[shrink]{
%\frame{
\frametitle{The English CP, IP and VP: Questions}

\savespace\small\smallexamples\parskip0pt\itemsep0pt


\centerline{%
\scalebox{.8}{
\begin{forest}
sm edges
[IP
		[NP[Ann,roof]]
		[I$'$
			[I$^0$[\trace$_k$]]
			[VP
				[V$'$
					[V$^0$[read]]
					[NP[the newspaper,roof]]]]]]
\end{forest}}}

\begin{itemize}
\item Ye/no questions are formed by fronting the auxiliary:
\ea
Will Ann read the newspaper?
\z
\pause
\item \emph{wh} questions are formed by additionally preposing a constituent:
\ea
What will Ann read?
\z
\item The auxiliary moves to the position of the complementizer.
\end{itemize}

}


\frame[shrink]{
\frametitle{English CP, IP and VP: Questions}

\vfill

\hfill
\scalebox{.9}{
\begin{forest}
sm edges
[CP
[XP [\trace]]
[C$'$
	[C$^0$[will$_k$]]
	[IP
		[NP[Ann,roof]]
		[I$'$
			[I$^0$[\trace$_k$]]
			[VP
				[V$'$
					[V$^0$[read]]
					[NP[the newspaper,roof]]]]]]]]
\end{forest}}
\hfill
\only<2->{\scalebox{.9}{
\begin{forest}
sm edges
[CP
[NP[what$_i$]]
[C$'$
	[C$^0$[will$_k$]]
	[IP
		[NP[Ann,roof]]
		[I$'$
			[I$^0$[\trace$_k$]]
			[VP
				[V$'$
					[V$^0$[read]]
					[NP[\trace$_i$]]]]]]]]
\end{forest}}}
\hfill\mbox{}

\vfill

\pause

}




\subsubsection{Topology of the German clause}

\frame{
\frametitle{Topology of the German clause (I)}

Before turning to the CP/IP system in grammars of German we have to sort out some terminology:

\begin{itemize}
\item Approaches to German constituent order often refer to topological fields.

\pause
\item Important works on topological fields are:\\
\citew{Drach37},  \citew{Reis80a} and \citew{HoehleTopo,Hoehle86}.

\pause
\item We will use \alert{Vorfeld}, \alert{linke/rechte Satzklammer},
\alert{Mittelfeld} and \alert{Nachfeld}.

\citew{Bech55a} introduced further fields for verbal complexes but we will ignore them here.
\end{itemize}


}

\frame{
\frametitle{Verb positions and terminology}

\savespace
\begin{itemize}
\item Verb-final position
      \ea
\gll Peter hat erzählt, \rot<5->{dass} er das Eis \braun<5->{\gruen<4>{gegessen}} \braun<5->{\blauit<-4>{hat}}.\\
     Peter has told that he the ice.cream eaten has\\
%\glt `Peter said that he has eaten the ice cream.'
      \z
\pause
\item Verb-final position
        \ea
\gll      \rot<4->{\blauit<-3>{Hat}} Peter das Eis \braun<5->{\gruen<4>{gegessen}}?\\
	 has Peter the ice.cream eaten\\
%\glt `Has Peter eaten the ice cream?'
      \z
\pause
\item Verb-second poisiton
      \ea
\gll  Peter \rot<4->{\blauit<-3>{hat}} das Eis \braun<5->{\gruen<4>{gegessen}}.\\
	 Peter has the ice.cream eaten\\
%\glt `Peter has eaten the ice cream.'
      \z
\end{itemize}


\pause
\begin{itemize}[<+->]
\item verbal elements continuous in (\mex{-2}) only
\item \rot<5->{left} and \braun<5->{right} sentence bracket
\item complementizer (\emph{weil}, \emph{dass}, \emph{ob}) in left sentence bracket
\item complementizer and finite verb have complementary distribution \citep{Hoehle97a}
\item region before, between and after the brackets: \alert{Vorfeld}, \alert{Mittelfeld}, \alert{Nachfeld}
\end{itemize}


}

\frame{
\frametitle{Topology of German clauses}


\resizebox{\textwidth}{!}{
\begin{tabular}{@{}lllll@{}}
\rowcolor{structure!15}Vorfeld & left bracket & Mittelfeld                             & right bracket & Nachfeld                   \\ 
\\
\rowcolor{structure!10}Karl    & schläft.      &                                        &                &                            \\
                       Karl    & hat           &                                        & geschlafen.    &                            \\
\rowcolor{structure!10}Karl    & erkennt       & Maria.                                 &                &                            \\
                       Karl    & färbt         & den Mantel                             & um             & den Maria kennt.           \\
\rowcolor{structure!10}Karl    & hat           & Maria                                  & erkannt.       &                             \\
                       Karl    & hat           & Maria als sie aus dem Zug stieg sofort & erkannt.       &                             \\
\rowcolor{structure!10}Karl    & hat           & Maria sofort                           & erkannt        & als sie aus dem Zug stieg. \\
                       Karl    & hat           & Maria zu erkennen                      & behauptet.     &            \\
\rowcolor{structure!10}Karl    & hat           &                                        & behauptet      & Maria zu erkennen.         \\ 
\\
\rowcolor{structure!10}        & Schläft       & Karl?                                  &                &                            \\
                               & Schlaf!       &                                        &                &                             \\
\rowcolor{structure!10}        & Iß            & jetzt dein Eis                         & auf!           &                             \\
        & Hat           & er doch das ganze Eis alleine          & gegessen.            &      \\  \\
\rowcolor{structure!10}        & weil          & er das ganze Eis alleine               & gegessen hat   & ohne sich zu schämen.\\
        & weil          & er das ganze Eis alleine               & essen können will    & ohne gestört zu werden.    \\
\rowcolor{structure!10}wer     &               & das ganze Eis alleine                  & gegessen hat.  &                             \\
\end{tabular}
}

}

% \frame{
% \frametitle{Der Prädikatskomplex}
% %


% \begin{itemize}
% \item<+-> mehrere Verben in der rechten Satzklammer: Verbalkomplex
% \item<+-> manchmal wird auch von diskontinuierlichen Verbalkomplexen
%       gesprochen (Initialstellung das Finitums)
% \item<+-> auch prädikative Adjektive (\mex{1}a) und Resultativprädikate (\mex{1}b) werden zum Prädikatskomplex gezählt:
%       \eal
%       \ex dass Karl seiner Frau treu ist
%       \ex dass Karl das Glas leer trinkt
%       \zl
% \end{itemize}

% }

\frame{
\frametitle{The Rangprobe}
%


\begin{itemize}
\item<+-> Fields may be empty.
      \ea
      \field{Der Delphin}{VF} \field{gibt}{LS} \field{dem Kind den Ball,}{MF} \field{das er kennt}{NF}.
      \z
\item<+-> Test: Rangprobe \citep[\page 72]{Bech55a}
\eal
\ex[]{
\gll Der Delphin hat [dem Kind] den Ball gegeben, [das er kennt].\\
     the dolphin has \spacebr{}the child the ball given \spacebr{}who he knows\\
\glt `The dolphin has given the ball to the child who it knows.'
}
\ex[*]{
\gll Der Delphin hat [dem Kind] den Ball, [das er kennt,] gegeben.\\
     the dolphin has \spacebr{}the child the ball \spacebr{}who he knows given\\
}
\zl
Replacing the finite verb by an auxiliary forces the main verb into the right sentence bracket.
\pause
\ea{
\gll Der Delphin hat [dem Kind, das er kennt,] den Ball gegeben.\\
     the dolphin has \spacebr{}the child who he knows the ball given\\
}
\z

\end{itemize}

}


\frame[shrink]{
\frametitle{Recursion}

\begin{itemize}
\item \citet*[\page 82]{Reis80a}: Recursion: Vorfeld can contain other topological fields:
\eal
\label{Beispiel-topologisch-komplexes-Vorfeld}
\ex
\gll Die Möglichkeit, etwas zu verändern, ist damit verschüttet für lange lange Zeit.\\
	 the possibility something to change is there.with buried for long long time\\
\glt `The possibility to change something will now be gone for a long, long time.'	  
\ex 
\gll {}[Verschüttet für lange lange Zeit] ist damit die Möglichkeit,      etwas zu ver"-ändern.\\
      \spacebr{}buried for long long time ist there.with the possibility  something to change\\
\ex 
\gll Wir haben schon seit langem gewußt, daß du kommst.\\
     we have \particle{} since long known that you come\\
\glt `We have known for a while that you are coming.'
\ex 
\gll {}[Gewußt, daß du kommst,] haben wir schon seit langem.\\
	 \spacebr{}known that you come have we \particle{} since long\\
\zl
% \pause
% \item Permutations occurring in the Mittelfeld can also take place within complex Vorfleds.

% \eal
% \ex {}[\gruen{Seiner Tochter} \blau{ein Märchen} erzählen] wird er wohl müssen.
% \ex {}[\blau{Ein Märchen} \gruen{seiner Tochter} erzählen] wird er wohl müssen.
% \zl
\end{itemize}

}


\frame{
\frametitle{Exercise}

Assign topological fields in the sentences in (\mex{1}):
\eal
\ex Der Mann hat gewonnen, den alle kennen.
\ex Sie gibt ihm das Buch, das Conny empfohlen hat.
\ex Maria hat behauptet, dass das nicht stimmt.
\ex Conny hat das Buch gelesen,\\das Maria der Schülerin empfohlen hat,\\die neu in die Klasse gekommen ist.
\ex Komm!
\zl

}

\subsubsection{The German CP and IP}

\frame{
\frametitle{The topological model paired with CP, IP, VP (I)}

\vfill
\centerline{
\scalebox{.7}{
        \begin{forest}
            sn edges original,empty nodes
            [CP
              [{}
                [XP,terminus
                  [SpecCP\\prefield, name=p1
                  ]
                ]
              ]
              [\hspaceThis{$'$}C$'$
                    [{}
                      [C, terminus
                        [C \\left SB, name=c0
                        ]
                      ]
                    ]
                    [IP
                      [{}
                        [XP, terminus
                          [{IP (without I, V )\\middle field}
                            [SpecIP\\subject position, set me left, name=specip
                            ]
                            [phrases inside\\the VP, name=p3
                            ]
                          ]
                        ]
                      ]
                      [\hspaceThis{$'$}I$'$
                              [VP, name=vp
                                [V, name=v0, terminus, no path, anchor=east
                                  [{V , I \\right SB}, name=p2, set me left
                                  ]
                                ]
                              ]
                              [{}
                                    [I , terminus, name=io
                                    ]
                              ]
                      ]
                    ]
              ]
            ]
            \draw [thick]
              (p1.north west) rectangle (io.east |- p3.south);
            \draw
              ($(c0.north east)!1/2!(specip.west |- c0.north east)$) coordinate (p6) -- (p6 |- p3.south)
              ($(p1.north east)!1/2!(c0.north west)$) coordinate (p4) -- (p3.south -| p4)
              ($(specip.north east)!1/2!(p3.north west)$) coordinate (p5) -- (p3.south -| p5)
              ($(p2.north west)!1/2!(p2.north west -| p3.east)$) coordinate (p7) -- (p3.south -| p7)
              (p6 |- p2.south) -- (p2.south -| p7)
              (vp.south) -- (v0.center -| p3.west) -- (v0.west)
              (v0.east) -- +(4.5pt,0) -- (vp.south)
              ;
        \end{forest}}}

}

% \frame{
% \frametitle{Das topologische Modell mit CP, IP, VP (II)}

% \footnotesize
% \resizebox{0.99\textwidth}{!}{
% \begin{tabular}{|l|l|l|l|l|}
% \hline
% %
% SpecCP    & \cnull      & \mc{2}{l|}{IP (ohne \inull, \vnull)} & \vnull, \inull\\
% Vorfeld   & Linke       & \mc{2}{l|}{Mittelfeld}                      & Rechte\\\cline{3-4}
%           & Satzklammer & SpecIP           & Phrasen innerhalb der VP & Satzklammer\\
%           &             & Subjektsposition &                          &\\\hline\hline
%           & dass         & Anna & [das Buch] [auf den Tisch] & legt$_k$ [ \_ ]$_k$\\
% \pause
%           & ob  & Anna & [das Buch] [auf den Tisch] & legt$_k$ [ \_ ]$_k$ \\\hline\hline
% \pause
% \ifthenelse{\boolean{gb-intro}}{
% wer$_i$      & [ \_ ] & [ t ]$_i$ & [das Buch] [auf den Tisch] & legt$_k$ [ \_ ]$_k$\\
% \pause
% was$_i$      & [ \_ ] & Anna & [ t ]$_i$ [auf den Tisch] & legt$_k$ [ \_ ]$_k$\\\hline\hline
% \pause
% }{}
%           & Legt$_k$ & Anna & [das Buch] [auf den Tisch]? & [ t ]$_k$ [ t ]$_k$\\
% \pause
%           & Legt$_k$ & Anna & [das Buch] [auf den Tisch], & [ t ]$_k$ [ t ]$_k$ \\\hline\hline
% \pause
% Anna$_i$     & legt$_k$ & [ t ]$_i$ & [das Buch] [auf den Tisch] & [ t ]$_k$ [ t ]$_k$\\
% \pause
% Wer$_i$      & legt$_k$ & [ t ]$_i$ & [das Buch] [auf den Tisch]? & [ t ]$_k$ [ t ]$_k$\\
% \pause
% {}[Das Buch]$_i$ & legt$_k$ & Anna & [ t ]$_i$ [auf den Tisch] & [ t ]$_k$ [ t ]$_k$\\
% \pause
% Was$_i$      & legt$_k$ & Anna & [ t ]$_i$ [auf den Tisch]? & [ t ]$_k$ [ t ]$_k$\\
% \pause
% {}[Auf den Tisch]$_i$ & legt$_k$ &Anna & [das Buch] [ t ]$_i$ & [ t ]$_k$ [ t ]$_k$\\\hline
% \end{tabular}
% }

% \pause
% \vfill
% Achtung: Die Bezeichner SpecCP u.\ SpecIP sind keine Kategoriensymbole. Sie kommen
% in Grammatiken mit Ersetzungsregeln nicht vor! Sie bezeichnen nur Positionen im Baum.

% }

\subsubsubsection{German as SOV language}

\frame{
\frametitle{German as SOV language}

\begin{itemize}

\item Heads of VP and IP (\vnull and \inull) are serialized to the right of their
  arguments. 

Together they form the right sentence bracket.


\pause
\item All other arguments and adjuncts are serialized to the left of them and form the Mittelfeld.

\pause
\item Typologically, German is a SOV language (basic order subject--object--verb), which is reflected at the D Structure level.

\begin{itemize}
\item SOV German, \ldots
\item SVO English, French, \ldots
\item VSO Welsh, Arabic, \ldots 
\end{itemize}
App.\ 40\,\% of all languages are SOV languages, app.\,35\,\% are SVO.

\item See \citew{MuellerGermanic} for discussion of Germanic and the classification of German.

\pause
\item Nice result of SOV structure: The closer a constituent is related to the verb, the closer it
  is to the right sentence bracket, even in sentences with inital finite verb and empty right
  sentence bracket.
\end{itemize}

}

\frame{
\frametitlefit{Motivation of SOV order as basic order: Particles}

\citew%[S.\,34--36]
{Bierwisch63a}: Verb particles form a close unit with the verb:
\eal
\ex 
\gll weil sie morgen \alert{an-fängt}\\
     because she tomorrow \textsc{part}-starts\\
\glt `because he is starting tomorrow'
\ex 
\gll Sie \alert{fängt} morgen \alert{an}.\\
     she starts tomorrow \textsc{part}\\
\glt `She is starting tomorrow.'
\zl
This unit can only be seen in verb"=final structures, which speaks for the fact that this structure
reflects the base order.
}

% \frame{
% \frametitle{Stellung der infiniten Verben}


% \eal
% \ex Dieses Buch sollte gelesen werden müssen.
% \ex This book should have been read.
% \zl

% }

\frame{
\frametitle{Sometimes SOV is the only option}

Sometimes SOV is the only option \citep[\page 370--371]{Hoehle2018Projektionsstufen}:
\eal
\ex[]{
\gll weil sie das Stück heute ur-auf-führen\\
	 because they the play today \textsc{pref}-\textsc{part}-lead\\
\glt `because they are performing the play for the first time today'
}
\ex[*]{
\gll Sie ur-auf-führen heute das Stück.\\
     they \textsc{pref}-\textsc{part}-lead today the play\\
}
\ex[*]{
\gll Sie führen heute das Stück ur-auf.\\
     they lead today the play \textsc{pref}-\textsc{part}\\
}
\zl

This is backformation. \emph{Ur-auf-führung} is wrongly assumed to be derived from the verb \emph{uraufführen}.

}

\frame{
\frametitle{Order in subordinated sentences}

Verbs in non-finite subordinated clauses and in finite subordinated clauses introduced by a
conjunction are positioned at the end (ignoring extraposition):
\eal
\ex 
\gll Der Clown versucht, Kurt-Martin die Ware zu geben.\\
     the clown tries Kurt-Martin the goods to give\\
\glt `The clown is trying to give Kurt-Martin the goods.'
\ex 
\gll dass der Clown Kurt-Martin die Ware gibt\\
	 that the clown Kurt-Martin the goods gives\\
\glt `that the clown gives Kurt-Martin the goods'
\zl
}

\frame{
\frametitle{Order of verbs in SVO and SOV languages}

\citet{Oersnes2009b}: 
\eal
\ex 
\gll dass er ihn gesehen$_3$ haben$_2$ muss$_1$\\
	 that he him seen have must\\\hfill(German)
\ex 
\gll at han må$_1$ have$_2$ set$_3$ ham\\
     that he must have seen him\\\hfill(Danish)
\glt `that he must have seen him'
\zl

OV: embedding verbs go to the end\\
VO: embedding verbs go to the beginning

(ignore the Dutch for the moment \ldots)


}

\frame{
\frametitle{Scope}

%\citew[Section~2.3]{Netter92}:
\citew{Netter92}:
Adverbs outscope material to their right (preference only?):

\eal
\ex 
\gll dass er [absichtlich [nicht lacht]]\\
     that he \spacebr{}intentionally \spacebr{}not laughs\\
\glt `that he is intentionally not laughing'
\ex 
\gll dass er [nicht [absichtlich lacht]]\\
     that he \spacebr{}not \spacebr{}intentionally laughs\\
\glt `that he is not laughing intentionally'
\zl
\pause
The scoping does not change if the verb is in initial position:
\eal
\ex 
\gll Er lacht$_i$ [absichtlich [nicht \_$_i$]].\\
     he laughs \spacebr{}intentionally \spacebr{}not\\
\glt `He is intentionally not laughing.'
\ex 
\gll Er lacht$_i$  [nicht [absichtlich \_$_i$]].\\
     he laughs \spacebr{}not \spacebr{}intentionally\\
\glt `He is not laughing intentionally.'
\zl
}


\subsubsubsection{C -- The left sentence bracket}


\frame{
\frametitle{\cnull{} -- The left sentence bracket in embedded clauses}

\cnull corresponds to the left sentence bracket and is filled as follows:
\begin{itemize}
\item In embedded sentences with subordinating conjunction\\ the conjunction (the complementizer) is
  placed in \cnull, as in English. 

The verb stays in the right sentence bracket.
\ea
\gll dass jeder diese Frau kennt\\
     that everybody this woman knows\\
\glt `that everybody knows this woman'
\z

\pause
\item The verb moves from V to I.

\end{itemize}

}


\frame{
\frametitle{V to I movement in embedded clauses}

\centerline{%
\scalebox{.77}{%
\begin{forest}
sm edges
[CP
[\hspaceThis{$'$}C$'$
	[C [dass;that]]
	[IP
		[NP [jeder;everybody,roof]]
		[\hspaceThis{$'$}I$'$
			[VP
				[\hspaceThis{$'$}V$'$
					[NP[diese Frau;this woman, roof]]
					[V [\trace$_j$]]]]
			[I [kenn-$_j$ -t;know- -s]]]]]]
\end{forest}
}
}

}


\frame{
\frametitle{\cnull{} -- The left sentence bracket in V1 and V2 clauses}

\begin{itemize}
\item The finite verb is moved via \inull to \cnull in verb-first and verb-second clauses: \\
\vnull $\to$  \inull $\to$ \cnull. 
\eal
\settowidth\jamwidth{(Verb in \vnull)}
\ex 
\gll dass jeder diese Frau kenn- -t\\
     that everybody this woman know- -s\\ \jambox{(Verb in \vnull)}
\ex 
\gll dass jeder     diese Frau \_$_i$ [kenn-$_i$ -t]\\
     that everybody this  woman {}   \spacebr{}know- -s\\ \jambox{(Verb in \inull)}
\ex 
\gll {}[Kenn-$_i$ -t]$_j$ jeder diese Frau \_$_i$ \_$_j$?\\
     \spacebr{}know- -s   everybody this woman\\ \jambox{(Verb in \cnull)}
\zl

\end{itemize}

}

\frame{
\frametitle{V to I to C movement in V1/V2 clauses}

\centerline{%
\scalebox{.77}{%
\begin{forest}
sm edges
[CP
[\hspaceThis{$'$}C$'$
	[C [(kenn-$_j$ -t)$_k$;knows]]
	[IP
		[NP [jeder;everybody,roof]]
		[\hspaceThis{$'$}I$'$
			[VP
				[\hspaceThis{$'$}V$'$
					[NP [diese Frau; this woman, roof]]
					[V [\trace$_j$]]]]
			[I  [\trace$_k$]]]]]]
\end{forest}
}}


}

\subsubsubsection{SpecCP -- The Vorfeld}

\exewidth{(235)}

\frame{
\frametitle{SpecCP -- The Vorfeld in declarative clauses (I)}

The position SpecCP corresponds to the Vorfeld and is filled as follows:
\begin{itemize}
\item Declarative clauses: XP is moved to the Vorfeld.
\ea
\gll Gibt der Mann dem Kind jetzt den Mantel?\\
     gives the.\NOM{} man the.\DAT{} child now the.\ACC{} coat\\
\glt `Is the man going to give the child the coat now?'
\z

\eal
\ex 
\gll Der Mann gibt dem Kind jetzt den Mantel.\\
     the.\NOM{} man gives the.\DAT{} child now the.\ACC{} coat\\
\glt `The man is giving the child the coat now.'
\pause
\ex 
\gll Dem Kind gibt der Mann jetzt den Mantel.\\
     the.\DAT{} child gives the.\NOM{} man now the.\ACC{} coat\\
\pause
\ex 
\gll Den Mantel gibt der Mann dem Kind jetzt.\\
	 the.\ACC{} coat gives the.\NOM{} man the.\DAT{} child now\\
\pause
\ex 
\gll Jetzt gibt der Mann dem Kind den Mantel.\\
	 now gives the.\NOM{} man the.\DAT{} child the.\ACC{} coat\\
\zl
\end{itemize}

}


\frame{

\frametitle{Verb movement and movement to SpecCP}

\vfill
\centerline{\scalebox{0.77}{
\begin{forest}
sm edges
[CP
[NP$_i$ [diese Frau;this woman, roof]]
[\hspaceThis{$'$}C$'$
	[C [(kenn-$_j$ -t)$_k$; know- -s]]
	[IP
		[NP [jeder;everybody,roof]]
		[\hspaceThis{$'$}I$'$
			[VP
				[\hspaceThis{$'$}V$'$
					[NP[\trace$_i$]]
					[V [\trace$_j$]]]]
			[I  [\trace$_k$]]]]]]
\end{forest}}}
\vfill


}

\frame[shrink]{
\frametitle{SpecCP -- The Vorfeld in declarative clauses (II)}

\begin{itemize}
\item The crucial factor for deciding which phrase to move is the \emph{information structure} of the sentence. Material connected to previously mentioned or otherwise"=known information is 
placed further left (preferably in the prefield) and new information tends to occur to the right. Fronting to the
prefield in declarative clauses is often referred to as
\alert{topicalization}. 

\item But this is rather a misnomer, since the focus (informally: the constituent being asked for)
  can also occur in the prefield. Expletives as well.

\pause
\item Caution:\\
      Movement to the Vorfeld does not have the same status as fronting in English!
\end{itemize}

}

\frame{
\frametitle{Nonlocal dependencies}

\begin{itemize}
\item Analysis also works for nonlocal dependencies:
\ea
  \gll {}[Um zwei Millionen Mark]$_i$ soll er versucht haben,~~~~ [eine Versicherung \_$_i$ zu betrügen].\footnotemark\\
     {}\spacebr{}around two million Deutsche.Marks should he tried have \spacebr{}an insurance.company {} to deceive\\
\footnotetext{%
         taz, 04.05.2001, p.\,20.
}
\glt `He apparently tried to cheat an insurance company out of two million Deutsche Marks.'
      \z
Step-wise movement: the fronted constituent first moves to the specifier position of the phrase it
originates from than to the next specifier of the next maximal projection and so on until it reaches
the uppermost SpecCP position.
\end{itemize}

}




\frame{
\frametitle{Exercise}

\smallframe
Draw the syntax trees for the fowllowing sentences:
\eal
\ex 
\gll dass der Delphin dem Kind hilft\\
     that the.\NOM{} dolphin the.\DAT{} child helps\\
\glt `that the dolphin helps the child'
\ex 
\gll dass der Delphin den Hai attackiert\\
     that the.\NOM{} dolphin the.\ACC{} shark attacks\\
\glt `that the dolphin attacks the shark'
\ex 
\gll dass der Hai attackiert wird\\
     that the.\NOM{} shark attacked is\\
\glt `that the shark is attacked'
\ex 
\gll Der Hai wird attackiert.\\
     the.\NOM{} shark is attacked\\
\glt `The shark is attacked.'
\ex 
\gll Der Delphin hilft dem Kind.\\
     the dolphin.\NOM{} helps the.\DAT{} child\\
\glt `The dolphin is helping the child.'
\zl

}


%      <!-- Local IspellDict: en_US-w_accents -->
