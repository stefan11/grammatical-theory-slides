%% -*- coding:utf-8 -*-

\subtitle{Government \& Binding}

\section{Government \& Binding (GB)}


\huberlintitlepage[22pt]


\outline{

\begin{itemize}
\item Introduction and basic terms
\item Phrase structure grammar
\item Government \& Binding (GB)
\item Generalized Phrase Structure Grammar (GPSG)
\item Lexical Functional Grammar (LFG)
%\item PATR
\item Categorial Grammar (CG)
\item Head-Driven Phrase Structure Grammar (HPSG)
%\item Konstruktionsgrammatik (CxG)
\item Tree Adjoning Grammar (TAG)
\end{itemize}

%\tableofcontents
}

\frame{
\frametitle{Reading material}

\citew[Section~3.1]{MuellerGT-Eng}


}

\subsection{General remarks on the representational format}

\subsubsection{History and motivation}


\frame{
\frametitle{Phrase structure grammars and natural language}

Chomsky: generlizations cannot be captured with PSGs (\eg active/passive alternations) $\to$
      transformations:

\begin{table}[H]
\begin{tabular}{@{}l@{~}l@{~}l@{~}l}
NP& V &NP &$\to$ 3 [\sub{AUX} be] 2en [\sub{PP} [\sub{P} by] 1]\\
1 & 2 &3\\
\end{tabular}
\end{table}

\eal
\ex Kim loves Sandy.
\ex Sandy is loved by Kim.
\zl

A tree with the sequence of symbols on the left-hand site is mapped to a tree with the sequence of
symbols on the right-hand side.

}


\frame{
\frametitle{Transformation of an active tree into a passive tree}

\vfill

%\oneline{%
\hfill
\begin{forest}
%sm edges
[S, for tree={parent anchor=south, child anchor=north}
  [NP [Kim] ]
  [VP
    [V [loves] ]
    [NP [Sandy] ] 
  ]]
\end{forest}
\hspace{1em}
\raisebox{6\baselineskip}{$\leadsto$}
\hspace{1em}
  \begin{forest}
  %sm edges
  [S, for tree={parent anchor=south, child anchor=north}
  	[NP[Sandy]]
	[VP
	[Aux[is, tier=word]]
	[V[loved, tier=word]]
	[PP
	[P[by, tier=word]]
	[NP[Kim, tier=word]]]]]
\end{forest}
\hfill\mbox{}
%}

\vfill

\begin{tabular}{@{}l@{~}l@{~}l@{~}l}
NP& V &NP &$\to$ 3 [\sub{AUX} be] 2en [\sub{PP} [\sub{P} by] 1]\\
1 & 2 &3\\
\end{tabular}

\vfill

}


\frame{
\frametitle{Complexity, transformations and natural languages}

\small
\begin{itemize}
\item There are different complexity levels for phrase structure grammars.\\(Chomsky Hierarchy, Type 3--0)
\pause
\item What we saw so far are so called context free grammars. They are of type 2.
\pause
\item Maximal level (type 0) is too powerful for human langauges.\\
$\to$ Researchers wanted to be more restrictive.
\pause
\item Grammars with general transformations correspond to PSGs with type 0 complexity \citep{PR73a-u}.
\pause
\item Transformations are not sufficiently restricted,\\interactions are not tractable,\\
      there have been problems with transformations deleting material (see \citew{Klenk2003a}).
\pause
\item $\to$ new theoretical approaches, Government \& Binding \citep{Chomsky81a}: restrictions for the form of grammar
  rules, elements can be connected to the position in a tree they were coming from, general
  principles to restrict the power of transformations

\end{itemize}

}

\settowidth\jamwidth{(Japanese)}
\frame{
\frametitlefit{Hypothesis regarding language acquisition: Principles \& Paramaters}

\begin{itemize}
\item Some of our linguistics knowledge is innate.\\
          (Not all linguists agree with this assumption! Discussion: \citew{MuellerGT-Eng})
\pause
\item Principles all linguistic structures have to obey
\pause
\item These principles are parametrized $\to$ there is choice\\
      A parameter may be set differently for different languages.

\medskip
\pause
      Example: \\

Principle: A head is placed before or after its complements depending on the value of the parameter \textsc{position}.


\medskip
%% \begin{tabular}[t]{@{}l@{ }l@{}}
%%                 Englisch  &$\to$ Verb steht vor Komplementen\\
%%                 Japanisch &$\to$ Verb steht nach Komplementen\\
%%                 \end{tabular}


\eal
\ex  be showing pictures of himself \jambox{(English)}
\ex
\gll zibun         -no syasin-o mise-te iru\\
     \textsc{self} of  picture  showing be\\\jambox{(Japanese)}
\zl

\end{itemize}

}

\lecture{T-Modell}{t-modell-lec}

\subsubsection{The T-model}



\frame{
\frametitle{Deep and Surface Structure}

\begin{itemize}[<+->]
\item Chomsky claimed that simple PSGs cannot capture certain regularities.\\
      \eg the relation between active and passive sentences.

\item Therefore he assumes an underlying structure,\\
      the so-called \blaubf{Deep Structure}.

\item A structure can be mapped onto another structure.

Parts may be deleted or moved to other positions in trees in such mappings.

As a result of such transformations a new structure is derived, the so-called \blaubf{Surface Structure}.

\medskip
\begin{tabular}{@{}l@{ = }l@{}}
\emph{Surface Structure} & S Structure\\
\emph{Deep Structure} & D Structure\\
\end{tabular}
\end{itemize}

}

%\beamertemplatebackfindforwardnavigationsymbolshorizontal


\frame[label=t-modell]{
\frametitle{The T-model}


%% \centerline{%
%% \resizebox{0.8\linewidth}{!}{
%% \begin{tabular}{@{}ccc@{}}
%% \xbar-Theorie der \\
%% \node{psr}{Phrasenstrukturregeln} & & \hyperlink{lexikon}{\node{lex}{Lexikon}}\\[6ex]
%% &\hyperlink{ds}{\node{ds}{D-Struktur}}\\[2ex]
%% \mc{3}{@{}c@{}}{\hyperlink{move-alpha}{move-$\alpha$}}\\[4ex]
%% &\hyperlink{ss}{\node{ss}{S-Struktur}}\\[6ex]
%% \node{tilg}{Tilgungsregeln},             && \node{ana}{Regeln des anaphorischen Bezugs,}\\
%% \node{filter}{Filter, phonol.\ Regeln}  && \node{quant}{der Quantifizierung und der Kontrolle}\\[6ex]
%% \hyperlink{pf}{\node{pf}{Phonetische}}             && \hyperlink{lf}{\node{lf}{Logische}}\\
%% \hyperlink{pf}{Form (PF)}               && \hyperlink{lf}{Form (LF)}\\
%% \end{tabular}
%% \anodeconnect{psr}{ds}\anodeconnect{lex}{ds}
%% \anodeconnect{ds}{ss}
%% \anodeconnect{ss}{tilg}\anodeconnect{ss}{ana}
%% \anodeconnect{filter}{pf}\anodeconnect{quant}{lf}
%% }}

\vfill
\centerline{%
\begin{forest}
for tree = {edge={->},l=4\baselineskip}
[D-structure
     [S-structure,edge label={node[midway,right]{move $\alpha$}} 
            [Deletion rules{,}\\Filter{,} phonol.\ rules
                    [Phonetic\\Form (PF)]]
            [Anaphoric rules{,}\\rules of quantification and control
                    [Logical\\Form (LF)]]]]
\end{forest}}

\vfill

}



\gotobuttonleft{t-modell}{T-model}

\frame[label=lexikon]{
\frametitle{The T-model: The lexicon}
\showsingleitemframe
\savespace
\begin{itemize}
\item<+> Contains a lexical entry for every word with information about:
\begin{itemize}\itemsep0pt
\item morphophonological structure
\item syntactic features
\item valence frame
\item \ldots
\end{itemize}
Contains list for word forms and morphemes and morphology component

\item<+> The lexicon is the interface between syntax and semantic interpretation of word forms.

\item<+> Vocabulary is not determined by UG (not innate),\\
just structural conditions are determined by UG.\\
(assumption not shared by all linguists)

\item<+> Morphosyntactic features (\eg gender) are not pre-determined:\\
Universal grammar provides a toolbox (claim not falsifiable).
% und setzt einige Minimalanforderungen.
\end{itemize}

}

\frame{
\frametitle{The T modell: D Structure, Move-$\alpha$ and S Structurr (I)}

\begin{itemize}
\item<+-> Phrase structure $\to$\\
We can describe relations between constituents.

\item<+-> A certain format for rules is given (\xbar-Schema).

Lexicon + structures of \xbar syntax = base for D Structure

\hypertarget{ds}{D Structure} = syntactic representation of valence frames of particular words as determined in the lexicon.
\end{itemize}


}

\frame{
\frametitle{The T-modell: D Structure, Move-$\alpha$ and S Structure (II)}

\begin{itemize}
\item<+-> constituents may be appearing at different places at the surface than the one determined
  by the valence frame:
\eal
\ex 
\gll {}[dass] der Mann dem Kind das Buch \alert{gibt}\\
     {}\spacebr{}that the.\NOM{} man the.\DAT{} woman the.\ACC{} book gives\\
\glt `that the man gives the woman the book'
\ex 
\gll \alert{Gibt} der Mann dem Kind das Buch?\\
	 gives the.\NOM{} man the.\DAT{} woman the.\ACC{} book\\
\glt `Does the man give the woman the book?'
\ex 
\gll Der Mann \alert{gibt} dem Kind das Buch.\\
	 the.\NOM{} man gives the.\DAT{} woman the.\ACC{} book\\
\glt `The man gives the woman the book.'
\zl
\item<+-> therefore transformational rules for reordering:\\
\hypertarget{move-alpha}{Move $\alpha$} = "`Move anything anywhere!"'

What exactly can be moved where and for which reason is determined
by principles.

\end{itemize}


}

\frame[label=ss,shrink=5]{
\frametitle{The T-modell: D Structure, Move-$\alpha$ and S Structure (III)}

\begin{itemize}
\item Relations between predicates and their arguments as determined by lexical entries must be
  recoverable on all representational levels for semantic interpretation.

\pause
\item
$\to$ Starting place of moved elements is marked with traces.

\eal
\ex 
\gll {}[dass] der Mann dem Kind das Buch gibt\\
	 {}\spacebr{}that the man the woman the book gives\\
\glt `that the man gives the woman the book'
\ex 
\gll Gibt$_i$ der Mann dem Kind das Buch \_$_i$?\\
	 gives the man the woman the book\\
\glt `Does the man give the woman the book?'
\ex 
\gll {}[Der Mann]$_j$ gibt$_i$ \_$_j$ dem Kind das Buch \_$_i$.\\
	 {}\spacebr{}the man gives {} the woman the book\\
\glt `The man gives the woman the book.'
\zl

Different traces are marked by indices.\\
Sometimes also \emph{e} for empty element and \emph{t} for trace.
\pause

\item
\hypertarget{ss}{S Structure} is a surface"=like structure but should not be equated with the structure of actual utterances.
\end{itemize}
}


\frame[label=pf]{
\frametitle{The T-model: Phonetic Form}

PF is the phonetic form of a sentence, the string of phonemes that are actually pronounced. 

The mapping from S Structure to PF incorporates the phonological laws.


\pause
Example: \emph{wanna} contraction

\eal
\ex The students want to visit Paris.
\ex The students wanna visit Paris.
\zl
The contratcion in (\mex{0}) is licenced by the optional rule in (\mex{1}):
\ea
want + to $\to$ wanna
\z


}

\frame[label=lf]{
\frametitle{The T-model: Logical Form (I)}

\begin{itemize}
\item Logical Form is a syntactic level mediating between S Structure and semantic interpretation of
  a sentence.

anaphoric reference (binding): what can pronouns refer to?
\eal
\ex 
\gll Peter kauft einen Tisch. Er gefällt ihm.\\
	 Peter buys a table(\mas) he likes him\\
\glt `Peter is buying a table. He likes it/him.'
\pause
\ex 
\gll Peter kauft eine Tasche. Er gefällt ihm.\\
	 Peter buys a bag(\fem) he likes him\\
\glt `Peter is buying a bag. He likes it/him.'
\pause
\ex 
\gll Peter kauft eine Tasche. Er gefällt sich.\\
	 Peter buys a bag(\fem) he likes himself\\
\glt `Peter is buying a bag. He likes himself.'
\zl
\end{itemize}

}


\frame{
\frametitle{The T-model: Logical Form (II)}

\begin{itemize}
\item Quantification:
\ea
Every dolphin attacks a shark.
\z
$\forall x \exists y (dolphin(x) \to (shark(y) \wedge attack(x,y))$\\
$\exists y \forall x (dolphin(x) \to (shark(y) \wedge attack(x,y))$

\item Some accounts try to derive the readings via movement of quantifiers in trees \citep{May85a-u}.
\end{itemize}



}


\frame{
\frametitle{The T-model: Logical Form (III)}
\smallframe

Control theory:\\
How is the semantic role of the subject of the infinitive filled?


\eal
\ex 
\gll Die Professorin schlägt der Studentin vor, die Klausur noch mal zu~~~~~~ schreiben.\\
     the professor suggests the student \textsc{part} the test once again to write\\
\glt `The professor advises the student to take the test again.'
\pause
\ex 
\gll Die Professorin schlägt der Studentin vor, die Klausur nicht zu bewerten.\\
	 the professor suggests the student \textsc{part} the test not to grade\\
\glt `The professor suggests to the student not to grade the test.'
\pause
\ex 
\gll Die Professorin schlägt der Studentin vor, gemeinsam ins Kino zu gehen.\hspace{-3pt}\\
	 the professor suggests the student \textsc{part} together into cinema to go\\
\glt `The professor suggests to the student to go to the cinema together.'
\zl

}

%\beamertemplatenavigationsymbolsempty 
\mode<beamer>{\beamertemplatebackfindforwardnavigationsymbolshorizontal}

\lecture{Lexicon}{lexicon}

\subsubsection{The lexicon}

\frame{
\frametitle{Lexicon: Basic terminology (I)}

\small
\begin{itemize}
\item meaning of words $\to$ combinatoric potential with certain semantic roles\\ (``acting person'' or ``affected thing'')

Example: meaning representation of (\mex{1}a) is (\mex{1}b):
\eal
\ex Judit beats the grandmaster.
\ex \relation{beat}(x,y)
\zl
\pause
\item This is subsumed under the terms \alert{valency} and \alert{selection}.

Note:\\
Semantic valence may differ from syntactic valence! (see \citealp[Section~1.6]{MuellerGT-Eng})
\pause
\item Another term is \alert{subcategorization}:

\emph{beat} is subcategorized for a subject and an object.

The word \emph{subcategorize} somehow developed its own life:\\
\emph{X subcategorizes for Y} is used for \emph{X selects Y}.
\end{itemize}


}


\frame{
\frametitle{Lexicon: Basic terminology (II)}
\savespace
\begin{itemize}
\item \emph{beat} is also called the \blaubf{predicate}\\
(since \relation{beat} is the logical predicate).

\pause
\item Subject and object are \alert{arguments} of the predicate.
\pause
\item Several terms for selectional requirement (some semantic, some syntactic, some
  mixed):  
\alert{argument structure}, \alert{valence frame}, \alert{subcategorization frame}, \alert{thematic grid}
and \alert{theta"=grid} or \alert{$\theta$-grid}
\pause
\item \alert{Adjuncts} modify semantic predicates.\\
If semantic aspects are discussed, the term is \alert{modifier}.\\
Adjuncts are not listed as part of valence frames.
\end{itemize}



}



\frame{
\frametitle{The Theta"=Criterion}

\label{theta-kriterium}%
Arguments are placed into certain positions in the clause (argument positions).

\alert{Theta"=Criterion} \citep[\page 36]{Chomsky81a}:
\begin{itemize}
\item Each theta"=role is assigned to exactly one argument position.
\item Every phrase in an argument position receives exactly one theta"=role.
\end{itemize}


}


\frame{
\frametitle{External argument and internal arguments}

\savespace
\begin{itemize}[<+->]
\item Arguments are ordered: there are higher- and lower"=ranked arguments

\item The highest"=ranked argument of verbs and adjectives has a special status. 

It is often (and always in some languages) realized in a position
outside of the verb or adjective phrase, it is called the \alert{\hypertarget{ext-arg}{external argument}}. 


\item The remaining arguments occur in positions inside of the VP or AP.

Term: \alert{internal argument} or \alert{complement}

\item For simple sentences: external argument = subject.
\end{itemize}

}

\frame{
\frametitle{Theta roles}
%\savespace

\begin{itemize}
\item<+->There are three classes of theta"=roles.
\item<+->Class 1 is usually the highest role, class 3 the lowest.
\begin{itemize}
\item<+-> Class 1: \alert{agent} (acting individual), the cause of an action or feeling (stimulus), holder of a certain property
\item<+-> Class 2: \alert{experiencer} (perceiving individual), the person profiting from something (\alert{beneficiary})
(or the opposite: the person affected by some kind of damage), \alert{possessor} (owner or soon"=to"=be owner of something, or the opposite:
someone who has lost or is lacking something) 
\item<+-> Class 3: \alert{patient} (affected person or thing), \alert{theme}
\end{itemize}

\item<+-> Caution!\\
Rather inconsistent assignment of roles by different authors. Proto-roles a la \citet{Dowty91a} may
be the only feasible way to deal with the problem.
\nocite{Gruber65a-u,Fillmore68,Fillmore71a-u,Jackendoff72a-u,Dowty91a}
\end{itemize}
}


\frame{
\frametitle{A lexical entry (I)}

Which information do we need to use a word appropriately?

Answer: The mental lexicon contains \alert{lexical entries} with the specific properties of
syntactic words needed to use that word grammatically.

Some of these properties are the following:
\begin{itemize}
\item form
\item meaning (semantics)
\item grammatical features: syntactic word class $+$ morphosyntactic features   
\item theta"=grid
\end{itemize}



}

\frame{
\frametitle{A lexical entry (II)}


{\footnotesize
\begin{tabular}[t]{|l|ll|}
\hline
form     & \emph{helf-} `help'&\\\hline
semantics & \relation{helfen}     &\\\hline
grammatical features  & \multicolumn{2}{l|}{verb}\\\hline\hline
%                       & \multicolumn{2}{l|}{3rd person singular indicative present active}\\\hline\hline

%\setlength{\arrayrulewidth}{9pt}
theta"=grid                &&\\\hline
theta"=roles                & \underline{agent} & beneficiary\\[2mm]\hline
grammatical particularities &                   & dative\\\hline
\end{tabular}
}

\bigskip
Arguments are ordered according to their ranking:\\
the highest argument is furthest left. 

In this case, the highest argument is the external argument.

The external argument is underlined.


}



\lecture{\xbar Theory}{x-bar}
\subsubsection{\xbar Theory}




\frame{

\frametitle{Comment on distribution of \xbar rules}

\xbar Theory is assumed in many other frameworks as well:\\
\begin{itemize}
%\item Government \& Binding (GB):\\\citew*{Chomsky93a}
\item Lexical Functional Grammar (LFG):\\\citew{Bresnan82a-ed,Bresnan2001a,BF96a-ed,Berman2003a}
\item Generalized Phrase Structure Grammar (GPSG):\\ \citew*{GKPS85a}
\end{itemize}

Sometimes different categories are assuemd.\\
In particular so-called functional categories (\eg INFL).

No assumptions about universality and innateness are made in most other theories.

}

\subsubsubsection{Heads}


\frame{
\frametitle{\xbar Theory: Heads}
\pause
Head determines the most important properties of a phrase.
\eal
\ex 
\gll Kim \alert{schläft}.\\
     Kim sleeps\\
\ex 
\gll Kim \alert{mag} Sandy.\\
     Kim likes Sandy\\
\ex 
\gll \alert{in} diesem Haus\\
     in this  house\\
\ex 
\gll ein \alert{Haus}\\
     a   house\\
\zl


}

\subsubsubsection{Lexical categories}

\frame{
\frametitle{\xbar Theory: Lexical categories}
\label{slide-lex-kat-gb}%

categories are divided into lexical and functional categories\\ 
($\approx$ correlates roughly with the difference between open and closed word classes)

Lexical categories: 
\begin{itemize}
\item V = verb
\item N = noun
\item A = adjective
\item P = preposition
\item Adv = adverb
\end{itemize}
% Abney87a:64--65 Meinunger2000a:38--39

}


\frame{
\frametitle{\xbar Theory: Lexical categories (cross classification)}
%\label{slide-lex-kat-gb}%

Attempt to use binary features to cross-classify lexical categories:

\bigskip

\centerline{\renewcommand{\arraystretch}{1.5}
\begin{tabular}{|c|c|c|}\hline
 & $-$ V & + V \\\hline
$-$ N & P = [ $-$ N, $-$ V] &  V = [ $-$ N, + V] \\\hline
  + N & N = [+ N, $-$ V]    &  A = [+ N, + V]\\\hline
\end{tabular}
}
\pause

\bigskip
Cross classification $\to$ simple way to refer to adjectives and verbs:\\
all lexical categories that are [ + V] are either verbs or adjectives.

Generalizations are possible \eg: [ + N] categories may bear case
% [ $-$ N]-Kategorien können Kasus regieren (allerdings auch A)
\medskip

Note: Adverbs can be treated as prepositions not selecting an argument.\nocite{Chomsky70a}

}



\frame[shrink]{
\frametitle{Head position dependent on the decomposed category?}

Nouns and prepositions are head-initial:

\eal
\ex
\gll \alert{für} Maria\\
	 for Maria\\
\ex 
\gll \alert{Bild} von Maria\\
	 picture of Maria\\
\zl

Adjectives and verbs are head-final:
\eal
\ex 
\gll dem König \alert{treu}\\
     the king loyal\\
\glt `Loyal to the king'
\ex 
\gll der [dem Kind \alert{helfende}] Mann\\
     the the child helping man\\
\glt `the man helping the child'
\ex 
\gll dem Mann \alert{helfen}\\
     the man help\\
\glt `help the man'
\zl


}

\frame{
\frametitle{Head position dependent on the decomposed category? (II)}

$\to$ \begin{tabular}[t]{@{}l@{}}
      {}[+ V] $\equiv$ head-final\\
      {}[$-$ V] $\equiv$ head-initial\\
      \end{tabular}

\pause
Problem: postpositions (P = [$-$ V])
\eal
\ex
\gll des Geldes \alert{wegen}\\
     the money because\\
\glt `because of the money'
\ex 
\gll die Nacht \alert{über}\\
     the night during\\
\glt `during the night'
\zl
\pause

Assume a new feature with binary value?\\
But then we would get four new categories in total.\\
But we need only one.

So, maybe this binary encoding is not such a good idea after all.



}



\subsubsubsection{Functional categories}

\frame{
\frametitle{\xbar Theory: Functional categories}

No cross-classification:\bigskip


\begin{tabular}{lp{\linewidth}@{}}
C   & Complementizer (subordinating conjunctions such as \emph{dass} `that')\\
I   & Finiteness (as well as Tense and Mood);\\
    & also Infl in earlier work (inflection),\\
    & T in more recent work (Tense) \\
D   & Determiner (article, demonstrative)\\
\end{tabular}


}



\subsubsubsection{Assumptions}


\frame{
\frametitle{\xbar Theory: Assumptions}

\begin{itemize}
\item \alert{Endocentricity}:\\
Every phrase has a head and every head is part of a phrase.

more technically: every head projects to a phrase. 


\pause
\item \alert{Binary branching} (predominant assumption today):\nocite{Kayne84a-u}\\
Non-terminal nodes are binary branching,\\
that is, there are no teneray branching nodes or nodes with more daughters.
\pause
\item \alert{Non-Tangling Condition}:\\
The branches of tree structures cannot cross.\\

\end{itemize}

}

\subsubsubsection{English clause structure}

\frame[shrink]{
\frametitle{English clause structure and \xbar Theory}


\begin{itemize}
\item In early work the following rules were assumed for English:
\eal
\ex S $\to$ NP VP
\ex S $\to$ NP Infl VP
\zl

\begin{forest}
sm edges
[S
  [NP
  	[Ann,roof]]
  [INFL
  	[will]]
  [VP
  	[V$'$
		[V$^0$[read]]
		[NP[the newspaper, roof]]]]]
\end{forest}

\pause
\item These rules do not adhere to the \xbar schema.


\end{itemize}

}



%\subsubsubsection{C and I}



%\subsubsubsection{The English IP and VP}

\frame{
\frametitle{The English IP and VP: Auxiliaries}

\savespace\small\parskip0pt

\medskip
\vfill

\centerline{
\scalebox{.9}{%
\begin{forest}
sm edges
[IP
  [NP
  	[Ann,roof]]
  [I$'$
  	[I$^0$
  		[will]]
	[VP
  	[V$'$
		[V$^0$[read]]
		[NP[the newspaper, roof]]]]]]
\end{forest}}}


\begin{itemize}
\item Instead of earlier approaches: INFL as head, INFL selecting a VP as complement.
\item Auxiliaries are placed in \inull (=~Aux).
\item Sentential adverbs may be placed between auxiliary and main verb.
\end{itemize}

}


\frame[shrink]{
\frametitle{The English IP and VP: Clauses without auxiliary}

\savespace\small\parskip0pt

\centerline{%
\scalebox{.9}{%
\begin{forest}
sm edges
[IP
	[NP[Ann,roof]]
	[I$'$
		[I$^0$[-s]]
		[VP
			[V$'$
				[V$^0$[read-]]
				[NP[the newspaper, roof]]]]]]
\end{forest}}
}

\begin{itemize}
\item Auxiliaries are placed in \inull (=~Aux).
\item Position may contain the inflectional affix. The finite verb moves there.

(Various variants of the theory \ldots. Some assume lowering of the affix, some assume an empty I
position and connection to the finite verb. For German, the best version seems to be to not assume I
at all \citep{Haider93a,Haider97a}.)
\end{itemize}

}



\subsubsection{c-command, m-command, and government}

\frame{
\frametitle{c-command, m-command, and government}

\begin{itemize}[<+->]
\item Case and (internal) theta roles are assigned under government.

\item Government is a syntactic relation in phrase structure.

\item Government relies on m-command.\\
      c-command is similar to m-command and needed for \hyperlink{bt}{Binding Theory}.

\end{itemize}
}


\subsubsubsection{c-command and m-command}

\frame[label=c-Kommando]{
\frametitle{c-command and m-command}

\savespace
Popular formulations:
\begin{itemize}
\item c-command: upwards and at the next possibility downward again
\item m-command: upwards and downwards at any dominating node but not higher than the next XP
\end{itemize}

\pause
Exact version:
\begin{description}
\item[c-command] A c-commands B iff neither A dominates B nor B dominates A
and the first branching node dominating A also dominates B.
\item[m-command] A m-commands B iff neither A dominates B nor B dominates A and the first maximal
  projection XP dominating A also dominates B.
\end{description}
\nocite{AS83a}
}

\frame{
\frametitle{Examples}
\label{bsp-m-kommando}

%\small

\vfill
\begin{columns}[T]

\begin{column}{45mm}
\hfill
\begin{forest}
for tree={s sep=3em,plain content}
[XP,name=xp 
  [UP,name=up]
  [X\rlap{$'$},name=xbar2
    [VP,name=vp]
    [X\rlap{$'$},name=xbar1
      [WP,name=wp
        [YP,name=yp]
        [W\rlap{$'$},name=wbar
          [ZP,name=zp]
          [W,name=w]]]
      [X,name=x]]]]
\draw[->] (x.north) to [curve through={($(xbar1.south)+(0,.05)$) }] (wp.north) ;
% additional point for wp.north to geth the three lines together
\draw[->] (x.north) to [curve through={($(xbar1.south)+(0,.05)$) .. ($(wp.north)+(0,.4)$) .. ($(wp.west)+(-.4,0)$)}] (yp.north) ;
% yet another additional wp.north point to get two lines together
\draw[->] (x.north) to [curve through={($(xbar1.south)+(0,.05)$) .. ($(wp.north)+(0,.4)$) .. ($(wp.north)+(-.28,.2)$) .. ($(wp.west)+(0,0)$) .. ($(wbar.west)+(0,0)$) .. ($(zp.north)+(.1,.2)$)}] (zp.north) ;
\draw[->] (x.north) to [curve through={($(xbar1.south)+(0,.05)$) .. ($(wp.north)+(0,.4)$) .. ($(wp.north)+(-.28,.2)$) .. ($(wp.west)+(0,0)$) .. ($(wp.south)+(0,.05)$) .. ($(wbar.east)+(.3,0)$)}] (w.north) ;
\end{forest}
\hfill\mbox{}

\hfill c-command \hfill\mbox{}
\end{column}
\begin{column}{45mm}
\pause
\hfill
\begin{forest}
for tree={s sep=3em,plain content}
[XP,name=xp 
  [UP,name=up]
  [X\rlap{$'$},name=xbar2
    [VP,name=vp]
    [X\rlap{$'$},name=xbar1
      [WP,name=wp
        [YP,name=yp]
        [W\rlap{$'$},name=wbar
          [ZP,name=zp]
          [W,name=w]]]
      [X,name=x]]]]
\draw[->] (x.north) to [curve through={($(xbar1.east)+(.2,.0)$) .. ($(xbar2.east)+(.2,0)$) .. ($(xp.south)+(0,.05)$)}] (up.north) ;
\draw[->] (x.north) to [curve through={($(xbar1.east)+(.2,.0)$) .. ($(xbar2.south)+(.2,.05)$)}] (vp.north) ;
\draw[->] (x.north) to [curve through={($(xbar1.south)+(0,.05)$) }] (wp.north) ;
% additional point for wp.north to geth the three lines together
\draw[->] (x.north) to [curve through={($(xbar1.south)+(0,.05)$) .. ($(wp.north)+(0,.4)$) .. ($(wp.west)+(-.4,0)$)}] (yp.north) ;
% yet another additional wp.north point to get two lines together
\draw[->] (x.north) to [curve through={($(xbar1.south)+(0,.05)$) .. ($(wp.north)+(0,.4)$) .. ($(wp.north)+(-.28,.2)$) .. ($(wp.west)+(0,0)$) .. ($(wbar.west)+(0,0)$) .. ($(zp.north)+(.1,.2)$)}] (zp.north) ;
\draw[->] (x.north) to [curve through={($(xbar1.south)+(0,.05)$) .. ($(wp.north)+(0,.4)$) .. ($(wp.north)+(-.28,.2)$) .. ($(wp.west)+(0,0)$) .. ($(wp.south)+(0,.05)$) .. ($(wbar.east)+(.3,0)$)}] (w.north) ;
\end{forest}
\hfill\mbox{}

\hfill m-command \hfill\mbox{}
\end{column}
\end{columns}

\vfill
}

\subsubsubsection{Government}

\frame{
\frametitle{Government (definition)}

Government is a structural relation between a head \xnull and a phrase YP: 

\begin{description}
\item[Government] \xnull governs YP iff a), b) and c) hold simultaneously:
\begin{itemize}
\item[a)] \xnull has category V, N, A, P (=~lexical cateories) or finite I.
\item[b)] \xnull m-commands YP.
\item[c)] There is no barrier between \xnull and YP.
\end{itemize}
\end{description}

Barrier is defined on a language-particular basis.\\
Simplified: maximal projections except IP. % und AgrP%Deutsch: 

\pause
Clause c) makes sure that heads can assign neither case nor theta role to parts of NP or PP.

c) restricts government in depth.
\nocite{Chomsky86b,Sternefeld91a-u}
}

\frame{
\frametitle{Government (example)}

\begin{columns}[T]

\begin{column}{57mm}
\hfill
\begin{forest}
for tree={s sep=3em,plain content}
[XP,name=xp 
  [UP,name=up]
  [X\rlap{$'$},name=xbar2
    [VP,name=vp]
    [X\rlap{$'$},name=xbar1
      [WP,name=wp
        [YP,name=yp]
        [W\rlap{$'$},name=wbar
          [ZP,name=zp]
          [W,name=w]]]
      [X,name=x]]]]
\only<1->{\draw[->,color=green] (x.north) to [curve through={($(xbar1.south)+(0,.05)$) }] (wp.north) ;}
\only<2>{\draw[->,color=red]   (x.north) to [curve through={($(xbar1.south)+(0,.05)$) .. ($(wp.north)+(0,.4)$) .. ($(wp.north)+(-.28,.2)$) .. ($(wp.west)+(0,0)$) .. ($(wbar.west)+(0,0)$) .. ($(zp.north)+(.1,.2)$)}] (zp.north) ;}
\end{forest}
\hfill\mbox{}
\end{column}
\begin{column}{63mm}
\begin{itemize}
\item X can assign a theta role to WP.
\pause

\item X cannot assign a theta role to ZP,\\
      since WP is a barrier, provided WP $\neq$ IP.

\end{itemize}
\end{column}
\end{columns}


}





%      <!-- Local IspellDict: en_US-w_accents -->
