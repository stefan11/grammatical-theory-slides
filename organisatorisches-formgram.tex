\newsavebox{\veranstaltungsurl}
\savebox{\veranstaltungsurl}{\url{https://hpsg.hu-berlin.de/~stefan/Lehre/GT/}}

\newsavebox{\veranstaltungsbuch}
\savebox{\veranstaltungsbuch}[11.5cm][l]{%
\parbox{11.5cm}{\raggedright Textbook: Müller, Stefan (\citeyear{MuellerGT-Eng}), \emph{Grammatical Theory} (Textbooks in Language Science 1).
  Berlin: Language Science Press fourth edition.\newline\url{https://langsci-press.org/catalog/book/287}\newline\newline
A bit outdated: Müller, Stefan (\citeyear{MuellerGTBuch})
  \emph{Grammatiktheorie}, (Stauffenburg Einführungen 20).
  Tübingen: Stauffenburg Verlag zweite Auf"|lage.\newline\url{http://hpsg.hu-berlin.de/~stefan/Pub/grammatiktheorie.html}}
}

%\input ../organisatorisches-BA-VL-HU.tex

\input organisatorisches.tex

\subsection{Leistungen}
\frame{
\frametitle{Leistungen}

Master Linguistik, Modul 2: Theoretische Grundlagen II, 2 SWS

\begin{itemize}
\item Aktive Teilnahme, Vor- und Nachbereitung
\item Klausur (im Modul für Linguistik)
%Hausarbeit (\url{http://hpsg.fu-berlin.de/~stefan/Lehre/hausarbeiten.html})
\end{itemize}

Ideale Zeitaufteilung:

\begin{tabular}{@{}l@{~}r@{~}l}
Präsenzstudium Vorlesung  & 25 h \\
Vor- und Nachbereitung    & 95 h & (35/15 = 2 h 20 min für jede Sitzung + 60h Prüf)\\
Klausurvorbereitung       & 
\end{tabular}

Für die Veranstaltung gibt es 4 Leistungspunkte.


}

\subsection{Recapitulation}
\frame{
\frametitle{Recapitulation}


\begin{itemize}
\item Linguistics 101 in the BA (4 SWS)
\item Tutorial Linguistics 101
\end{itemize}

}

