\section{Konstruktionsgrammatik (CxG)}

\outline{

\begin{itemize}
\item Begriffe
\item Phrasenstrukturgrammatiken
\item Generalisierte Phrasenstrukturgrammatik (GPSG)
\item Lexikalisch-Funktionale Grammatik (LFG)
%\item Lexical Mapping Theory (LMT)
%\item PATR
\item Kategorialgrammatik (CG)
\item Kopfgesteuerte Phrasenstrukturgrammatik (HPSG)
\item \blaubf{Konstruktionsgrammatik (CxG)}
\item Baumadjunktionsgrammatik (TAG)
\end{itemize}
}

\frame{
\frametitle{Konstruktionsgrammatik (I)}

\begin{itemize}[<+->]
\item Konstruktionsgrammatik gehört auch zur West-Coast-Linguistik
\item Sie wurde maßgeblich von Charles Fillmore, Paul Kay und Adele Goldberg geprägt.
\citep*{FKoC88a,KF99a,Kay2002a,Kay2005a,Goldberg95a,Goldberg2006a}

\item Fillmore, Kay, Jakendoff und andere weisen darauf hin, dass Sprache zu großen Teilen aus
  komplexeren Einheiten besteht, die sich jedoch nicht ohne weiters mit den Mitteln beschreiben
  lassen, die wir bisher kennengelernt haben.

\item In Frameworks wie GB wird explizit zwischen Kerngrammatik und Peripherie unterschieden und die
  Peripherie wird weitestgehend ignoriert. Die Kritik der CxG an einer solchen Praxis ist
  berechtigt, da die Zuordnung zur Peripherie mitunter willkürlich scheint und auch nichts gewonnen
  ist, wenn man große Teile von Sprachen von der Beschreibung ausschließt, weil sie teilweise
  irregulär ist.
\end{itemize}

}

\frame{
\frametitle{Konstruktionsgrammatik (II)}

\begin{itemize}
\item Konstruktionsgrammatik gibt es in vielen Varianten:
\begin{itemize}
\item Construction Grammar (Berkeley)
\item Goldbergian/Lakovian Construction Grammar \citep{Goldberg95a,Goldberg2006a}
\item Cognitive Grammar \citep{Dabrowska2004a}
\item Radical Construction Grammar \citep{Croft2001a}
\item Embodied Construction Grammar \citep{BC2005a}
\item Fluid Construction Grammar \citep{SDB2006a-u}
\item Sign-Based Construction Grammar (HPSG-Variante) \citep{Sag2012a}
\end{itemize}
\pause
\item Phänomenbeschreibungen zielen meistens auf phrasale Muster ab.
\item Als Haupterklärungsmittel wird die Vererbung benutzt.\\
      \Zb \citew{Croft2001a,Goldberg2003a}
\pause
\item Deutschsprachiger Sammelband zum Thema: \citew{FS2006a-ed}.
\item Vergleich HPSG/CxG und generelle Diskussion, was CxG ist: \citew{MuellerCxG}
\end{itemize}

}

\frame{
\frametitle{Grundannahmen}

\begin{itemize}
\item kein angeborenes sprachspezifisches Wissen
\item keine Transformationen
\pause
\item keine leeren Elemente
\pause
\item lexikalische Integrität (wie in LFG, HPSG)
\pause
\item Vererbungshierarchien spielen eine große Rolle
\pause
\item Goldberg argumentiert für Konstruktionsstatus von Resultativkonstruktionen \citep{Goldberg95a,GJ2004a}:
\ea
Er lacht sich schlapp.
\z
\Dh, es gibt keinen Kopf, der die Zahl der Argumente festlegt.\\
Das wird durch die phrasale Resultativkonstruktion geregelt.

Das ist ein fundamentaler Unterschied zur GB, LFG und HPSG.

\end{itemize}

}

\frame{
\frametitle{Formalisierung}

\begin{itemize}
\item Es gibt nur sehr wenige Arbeiten zur Formalisierung der CxG.
\item Formalere Arbeiten sind:\\
\citew{KF99a,Kay2002a},\\
\citew{MR2001a},\\
\citew{Goldberg2003a},\\
\citew{BC2005a},\\
\citew{SDB2006a-u,vanTrijp2011a,Steels2013a}.
\item Eine durch Ideen der CxG beeinflusste Arbeit ist das Buch von Jean-Pierre Koenig (damals Berkeley)
  \citeyearpar{Koenig99a} im Rahmen der HPSG.
\item Fillmore und Kay haben eng mit Sag zusammengearbeitet,\\
  woraus sich eine HPSG"=Variante ergeben hat,\\
  die Sign"=Based Construction Grammar genannt wird \citep{Sag2012a}.
\end{itemize}

}



\section{Probleme phrasaler Ansätze}

\outline{

\begin{itemize}
\item Phrasale Konstruktionen
\item \blau{Probleme phrasaler Ansätze}
      \begin{itemize}
      \item {Probleme mit derivationeller Morphologie: {\it Ge- -e}-Derivation}
      \item {Lexikalische Integrität und phrasale Konstruktionen}
      \item {Aktiv/Passiv und phrasale Konstruktionen}
      \item {Einstöpseln und Kasus}
      \item {Einstöpseln und Koordination}
      \end{itemize}
\item Evidenz für phrasale Ansätze
\end{itemize}

}


\subsection{Probleme mit derivationeller Morphologie: {\it Ge- -e}-Derivation}

\frame{
\frametitlefit{Problem: Phrasale Konstruktionen und derivationelle Morphologie}


\begin{itemize}


\item Jackendoffs Rumpel-Konstruktion \citep{Jackendoff2008a}:\nocite{GJ2004a}
\ea
The bus \rot{rumble}d \gruen{around the corner}.\\
{}[\sub{VP} \rot{V} \gruen{PP}] = `go PP in such a way to make a V-ing sound'
\z

\pause
\item Die englische Konstruktion hat ein Ebenbild im Deutschen:
\ea
dass die Straßenbahnen \gruen{um} \gruen{die} \gruen{Ecke} \rot{quietsch}en
\z
\nocite{Mueller2006d}

\pause
\item Deutsch: diskontinuierliches Derivationsmuster interagiert mit der \emph{Quietsch}-Konstruktion:

\ea
wegen des \gruen{Um-die-Ecke}-Ge\rot{quietsch}es der Straßenbahnen
\z

\pause
 
\item Wenn die Bedeutung an der phrasalen Konstruktion hängt,\\
brauchen wir die folgende Allo-Konstruktion:
\ea
{}[\sub{N} \gruen{PP} [\sub{N} [\sub{N-stem} ge- \rot{V-stem} -e] -s]  ]
\z

\end{itemize}

}


\frame{
\frametitle{\geed}

Die Interaktion zwischen \emph{Quietsch}-Konstruktion und \gee ist aber in keiner Weise besonders. 
\nocite{Luedeling2001a,Mueller2002b,Mueller2003a}

Beispiele \geed:\footnote{
aus dem COW"=Korpus \url{http://hpsg.fu-berlin.de/cow/} 9,108,097,177 Token für Dt.
}

\eal
\ex 
an-den-Hals-Gewerfe
\ex 
aus-dem-Bett-Gequäle
%im-Kreis-Gelaufe
%in-die-Welt-Genöle
%in-extremo-Gerocke
%um-den-Brei-Gerede
%um-die-Probleme-herum-Gedeute
%um-sich-Geschmeiße
\ex 
von-oben-herab-Geschreibe
% \ex 
% \gll von-Start-bis-Ziel-im-Regen-Gelaufe\\
%      from-start-to-finnish-in.the-rain-running\\ 
\zl

% \pause

% Note that \emph{um den heißen Brei herumreden} (`to beat about the bush') is an idiom!
% \ea
% um-den-heißen-Brei-Gerede
% \z


% $\to$ lexical analysis seems to be required \citep*{NSW94a,Sag2007a}.

}


\frame{
\frametitle{Lexikonregelansatz für die \emph{quietsch}-Konstruktion}

\begin{itemize}
\item Lexikoneintrag für das intransitive Verb \stem{quietsch} (Listem \sliste{ NP } ) 
\pause
\item Lexikonregel lizenziert \stem{quietsch} (\sliste{ NP, PP })
\pause
\item Flexion oder Derivation gefolgt von Flexion
\pause
\item kombinatorisches System, das Köpfe mit Argumenten kombiniert und Dislokation von
  Konstituenten ermöglicht.
\end{itemize}

}

\subsection{Lexikalische Integrität und phrasale Konstruktionen}

\outline{

\begin{itemize}
\item Phrasale Konstruktionen
\item {Probleme phrasaler Ansätze}
      \begin{itemize}
      \item {Probleme mit derivationeller Morphologie: {\it Ge- -e}-Derivation}
      \item \blau{Lexikalische Integrität und phrasale Konstruktionen}
      \item {Aktiv/Passiv und phrasale Konstruktionen}
      \item {Einstöpseln und Kasus}
      \item {Einstöpseln und Koordination}
      \end{itemize}
\item Evidenz für phrasale Ansätze
\end{itemize}

}


\frame{
\frametitle{Lexikalische Integrität und phrasale Konstruktionen}

\begin{itemize}
\item Das Partizip ist nur bildbar, wenn es ein Objekt gibt:
\ea
\begin{tabular}[t]{@{}l@{~}l@{\hspace{3em}}l@{~}l}
a. & * Er tanzt die Schuhe. & c. & Er liebt das Buch.\\
b. & * die getanzten Schuhe & d. & das geliebte Buch\\
\end{tabular}
\z

\pause

\item Mit Resultativprädikat ist Partizip möglich:

\eal
\ex[]{
Er tanzt die Schuhe blutig / in Stücke.
}
\ex[]{\label{blutig-getanzte-schuhe}
die blutig / in Stücke getanzten Schuhe
}
\zl
Das legt nahe, dass die Resultativkonstruktion lexikalisch lizenziert ist. (\citealp[\page
412]{Dowty78a}; \citealp[\page 21]{Bresnan82a}; \citealp{Mueller2006d})

\pause
\item Lexikonbasierte Analysen für die Resultativkonstruktion:
\citew{Simpson83a,Wunderlich92a-u-kopiert,Verspoor97a,Wechsler97a,WN2001a,Mueller2002b,Kay2005a,Jacobs2009a,Welke2009a}.
\end{itemize}

}


\subsection{Aktiv/Passiv und phrasale Konstruktionen}

\outline{

\begin{itemize}
\item Phrasale Konstruktionen
\item {Probleme phrasaler Ansätze}
      \begin{itemize}
      \item {Probleme mit derivationeller Morphologie: {\it Ge- -e}-Derivation}
      \item {Lexikalische Integrität und phrasale Konstruktionen}
      \item \blau{Aktiv/Passiv und phrasale Konstruktionen}
      \item {Einstöpseln und Kasus}
      \item {Einstöpseln und Koordination}
      \end{itemize}
\item Evidenz für phrasale Ansätze
\end{itemize}

}


\frame{
\frametitle{Aktiv/Passiv und phrasale Konstruktionen}


\begin{itemize}
\item Fillmore/Kay und \citet{MR2001a} und \citet{CJ2005a}: Passiv ist eine alternative Unifikation von Beschränkungen 

\pause
\item Auch in anderen Theorien:
\begin{itemize}
\item HPSG: \citew[Kapitel~3]{Koenig99a} %\citew[Chapter~4]{MR2001a}; 
\citealp{Koenig99a,DK2000b-u,Kordoni2001b-u}
\item TAG: \citealp{Candito96a}; \citealp[\page 188]{CK2003a-u}
\item Simpler Syntax: \citew{CJ2005a}
\item LFG: \citew{AGT2014a}
\end{itemize}


\end{itemize}

}

\frame{
\frametitle{Passive in Simpler Syntax \citep{CJ2005a}}

\vfill
\scalebox{.7}{%
\begin{tabular}{ccccc}
DESIRE(&{~\rnode{b}{BILL$_2$}}, && & ~{\rnode{sw}{[SANDWICH; DEF]$_3$}})\\
\\[1ex]
       &{\rnode{gf2}{GF$_2$}}    && & {\rnode{gf3}{GF$_3$}}~\\
\\[1ex]
~~~~~~~~~\hfill{}[\sub{S} & {\rnode{np2}{NP$_2$}}  & [\sub{VP} & V$_1$ & ~~{\rnode{np3}{NP$_3$}}]] \\
\\
              & Bill           &  & desires & the sandwich.\\
\end{tabular}
\ncline{b}{gf2}\ncline{gf2}{np2}%
\ncline{sw}{gf3}\ncline{gf3}{np3}%
}
\pause
\vfill\vfill

\scalebox{.7}{%
\begin{tabular}{ccccc}
DESIRE(&~{\rnode{b}{BILL$_2$},} & & & ~{}{\rnode{sw}{[SANDWICH; DEF]$_3$}})\\
\\[1ex]
       &{\rnode{gf2}{GF$_2$}}    &&  & {\rnode{gf3}{GF$_3$}}\\
\\[1ex]
~~~~~~~~~\hfill{}[\sub{S} & {\rnode{np3}{NP$_3$}}  & [\sub{VP} & V$_1$  & by {\rnode{np2}{NP$_2$}}]] \\
\\
              & the sandwich             & & is desired & by Bill.\\
\end{tabular}
\ncline{b}{gf2}\ncline{gf2}{np2}%
\ncline{sw}{gf3}\ncline{gf3}{np3}%
}
\vfill

}


\frame{
\frametitle{Aktiv/Passiv und phrasale Konstruktionen}

\smallexamples


\begin{itemize}
\item Passiv interagiert mit anderen valenzverändernden Konstruktionen:
      \begin{itemize}
      \item Litauisch \citep{Timberlake82a,Wiemer2006a}
      \item Irisch \citep{Noonan94a}
      \item Türkisch \citep{Ozkaragoez86a}
      \item \ldots
      \end{itemize}

\eal\settowidth\jamwidth{(Türkisch)}
\ex%\label{ex-double-passivization-strangle}
\gll Bu şato-da boğ-ul-un-ur.\\
     dieses Schloss-{\sc loc} erwürg-{\sc pass}-{\sc imp}-{\sc aor}\\\jambox{(Türkisch)}
\glt `Man wird erwürgt (von jemandem) in diesem Schloß.'
\ex
\gll Bu oda-da döv-ül-ün-ür.\\
     dieser Raum-{\sc loc} schlag-{\sc pass}-{\sc imp}-{\sc aor}\\
\glt `Man wird geschlaggen (von jemadem) in diesem Raum.'
%% \ex
%% \gll Harp-te vur-ul-un-ur.\\
%%      war-{\sc loc} shoot-{\sc pass}-{\sc pass}-{\sc aor}\\
%% \glt `One is shot (by one) in war.'
\zl

%% \eal\settowidth\jamwidth{(Litauisch)}
%% \ex
%% \gll Vėjas nupūte tą lapelį.\\
%%      wind.\gruen{\nom}{} blies dieses Blatt.\gruen{\acc}\\\jambox{(Litauisch)}
%% \glt `Der Wind blies dieses Blatt herunter.' % The wind blew down that leaf.
%% \ex
%% \gll Tas lapelis vėjo nupūstas.\\
%%      dieses Blatt.\gruen{\nom}.\mas.\sg{} Wind.\gruen{\gen}{} blasen.\nom.\mas.\sg{}\\
%% \glt `Dieses Blatt wurde vom Wind heruntergeblasen.'
%% \ex
%% \gll To     lapelio               būta  vėjo        nupūsto.\\
%%      dieses Blatt.\gruen{\gen}.\mas.\sg{} wurde.\nom.\neu.\sg{} Wind.\gruen{\gen}{} blasen.\gen.\mas.\sg{}\\
%% \glt `Dieses Blatt wurde (wahrscheinlich) vom Wind heruntergeblasen.'
%% \zl

\item \citet{Blevins2003a}: Passiv + Impersonal

\pause
\item Lösung: Passivlexikonregel oder morphem-basierter Ansatz.

\end{itemize}

}

\subsection{Einstöpseln und Kasus}

\outline{

\begin{itemize}
\item Phrasale Konstruktionen
\item {Probleme phrasaler Ansätze}
      \begin{itemize}
      \item {Probleme mit derivationeller Morphologie: {\it Ge- -e}-Derivation}
      \item {Lexikalische Integrität und phrasale Konstruktionen}
      \item {Aktiv/Passiv und phrasale Konstruktionen}
      \item \blau{Einstöpseln und Kasus}
      \item {Einstöpseln und Koordination}
      \end{itemize}
\item Evidenz für phrasale Ansätze
\end{itemize}

}

\frame{
\frametitle{Einstöpseln und Kasus}

\begin{itemize}
\item \citet{Goldberg95a}: Ein Lexikoneintrag enthält die Semantik des Verbs und Angaben darüber, welche Rollen realisiert werden müssen.

\pause
\item Lexikoneinträge können in Argumentstrukturkonstruktionen eingesetzt werden (regulär oder mit Erzwingung = Coercion)

\pause
\item Problem: Kasusanforderungen lassen sich nicht auf semantische Unterschiede zurückführen:  
\eal
\ex Er hilft dem Mann.
\ex Er unterstützt den Mann.
\zl
\eal
\ex Er begegnet dem Mann.
\ex Er trifft den Mann.
\zl
\pause
\item Auch mit Coercion nicht grammatisch:
\ea[*]{
Er hilft den Mann.
}
\z
\end{itemize}
}


\subsection{Einstöpseln und Koordination}

\outline{

\begin{itemize}
\item Phrasale Konstruktionen
\item {Probleme phrasaler Ansätze}
      \begin{itemize}
      \item {Probleme mit derivationeller Morphologie: {\it Ge- -e}-Derivation}
      \item {Lexikalische Integrität und phrasale Konstruktionen}
      \item {Aktiv/Passiv und phrasale Konstruktionen}
      \item {Einstöpseln und Kasus}
      \item \blau{Einstöpseln und Koordination}
      \end{itemize}
\item Evidenz für phrasale Ansätze
\end{itemize}

}

\frame{%[shrink=5]{
\frametitle{Symmetrische Koordination}

\savespace
\smallexamples

\begin{itemize}
\item Wortgruppen mit gleichen syntaktischen Eigenschaften koordinierbar:
\eal
\ex Er [kennt und liebt] diese Schallplatte.
\ex Ich bin [froh und stolz auf meinen Sohn].\footnote{
    \url{http://www.lebenshilfebruchsal.de/index.php?option=com_content&task=view&id=227&Itemid=59}. 20.06.2012
}
%\ex dein [Freund und Helfer]
\zl

%% \eal
%% \ex[]{
%% \gll Ich kenne und unterstütze diesen Mann.\\
%%      I know and support this mann.\acc\\
%% }
%% \ex[*]{
%% \gll Ich kenne und helfe diesen Mann.\\
%%      I know and help this man.\acc\\
%% }
%% \ex[*]{
%% \gll Ich kenne und helfe diesem Mann.\\
%%      I know and help this man.\dat\\
%% }
%% \zl


\pause
\item Solche Koordinationen sind auch möglich, wenn ein Verb mit einfachem und eins mit erweitertem Valenzrahmen verwendet wird:
\ea
ich hab ihr jetzt diese Ladung Muffins mit den Herzchen drauf [gebacken und gegeben].\footnote{
\url{http://www.musiker-board.de/diverses-ot/35977-die-liebe-637-print.html}. 08.06.2012
}
\z
\pause
\item Problem: Wenn die GF erst nach dem Einsetzen in die Konstruktion
vorhanden wären, dann könnten wir die Koordination nicht durchführen.
\nocite{Wechsler2008a,Marantz97a}

\pause
\item Lösung: dreistelliges Verb \emph{gebacken} (Lexikonregel)
\end{itemize}
%~\\[-3\baselineskip]


}

\subsection{Phrasale Konstruktionen}

\frame{
\frametitle{Phrasale Konstruktionen}

\begin{itemize}
\item Es gibt Strukturen, die nicht der \xbart entsprechen.

\citet{Matsuyama2004a} und \citet{Jackendoff2008a}:
\eal
\ex Student after student left the room.
\ex
\label{ex-npn-iteration}
Day after day after day went by, but I never found the courage to talk to
her. \citep{Bargmann2015a}
\zl

\end{itemize}




}


\subsection{Allgemeines zum Repräsentationsformat}

\subsubsection{Boxen}

\frame{
\frametitle{Boxen statt Merkmal-Wert-Paaren}


HPSG: Dominanzverhältnisse werden wie andere Eigenschaften linguistischer Objekte durch
Merkmal"=Wert"=Paaren beschrieben.

Die BCG (Berkeley CxG) verwendet zwar im Allgemeinen Merkmal"=Wert"=Paare zur Beschreibung linguistischer Objekte,
die Dominanzverhältnisse werden aber über Boxen dargestellt:

\bigskip

\begin{tabular}{|ll|}\hline
\multicolumn{2}{|c|}{phon \phonliste{ der Mann }}\\
\begin{tabular}[t]{|l|}\hline
phon \phonliste{ der } \\\hline
\end{tabular}&%
\begin{tabular}[t]{|l|}\hline
phon \phonliste{ Mann } \\\hline
\end{tabular}\\
  &\\\hline
\end{tabular}


}

\subsubsection{Die Kopf"=Komplement"=Konstruktion}

\frame{
\frametitle{Die Kopf"=Komplement"=Konstruktion}

Head plus Complements Construction (HC)\\
\medskip
\begin{tabular}{|ll@{~}l|}\hline
\begin{tabular}[t]{|l|}\hline
\begin{tabular}{@{}l@{~}l@{}}
role  & head\\
lex   & $+$   \\
\end{tabular}\\
\hline
\end{tabular}&%
\begin{tabular}[t]{|l|}\hline
\begin{tabular}{@{}l@{~}l@{}}
role  & filler\\
loc   & $+$     \\
\end{tabular}\\
\hline
\end{tabular} & \begin{tabular}[t]{@{}l@{}}\\
$+$\\
\end{tabular}\\
& &\\\hline
\end{tabular}


Ein Kopf wird mit mindestens einem Komplement kombiniert.\\
(Das `+' hinter der Box steht für mindestens ein Zeichen,\\
 das zur Beschreibung in der Box passt.)

{\sc loc}+ bedeutet, dass das Element lokal realisiert werden muss.

Der Wert von {\sc role} sagt etwas über die Rolle aus,\\
die ein bestimmtes Element in einer Konstruktion spielt.

}


\frame{
\frametitle{Verbphrasen-Konstruktion}

\begin{tabular}{|ll|}\hline
\blau<1>{cat v} & \\
\begin{tabular}[t]{|l@{~}l|}\hline
role & head\\
lex  & $+$   \\\hline
\end{tabular}
&%
\begin{tabular}[t]{|l@{~}l|l}\cline{1-2}
role         & filler   \\
loc          & $+$        & $+$\\
\blau<2>{gf} & \blau<2>{$\neg$subj}\\
\cline{1-2}
\end{tabular}\\
& \\\hline
\end{tabular}


Die syntaktische Kategorie der gesamten Konstruktion ist V.

\pause
Die Komplemente dürfen nicht die grammatische Funktion Subjekt haben.

}


\frame{
\frametitle{Vererbung}

Die VP-Konstruktion ist eine bestimmte Art von Kopf"=Komplement"=Konstruktion.

\begin{tabular}{|ll|}\hline
\multicolumn{2}{|l|}{INHERIT HC}\\
cat v &  \\
% Box 1
\begin{tabular}[t]{|l|}\hline
\hspace{5em}\\\hline
\end{tabular}&%
% Box 2
\begin{tabular}[t]{|l|l}\cline{1-1}
% Inhalt 2
\begin{tabular}{@{}l@{~}l@{}}
gf   & $\neg$subj\\
\end{tabular} & $+$\\
% End Inhalt 2
\cline{1-1}
\end{tabular} \\
& \\\hline
\end{tabular}


}

\frame{
\frametitle{Valenz}

\begin{itemize}
\item Valenz wird in Mengen repräsentiert.
\item Valenzprinzip:
      Lokale Füllertöchter werden mit einem Element in der Valenzmenge der Mutter identifiziert.
\item Subset-Prinzip:
      Mengenwerte der Kopftochter sind Teilmengen der entsprechenden Mengen der Mutter.
\item Achtung: Das ist genau das Entgegengesetzte von HPSG.\\
      In HPSG"=Grammatiken werden Valenzlisten abgearbeitet, in CxG sind beim Mutterknoten
      mindestens so viele Elemente vorhanden, wie bei der Kopftochter.

\end{itemize}

}

\frame{
\frametitle{Valenz und Adjunkte}

\begin{itemize}
\item Kay und Fillmore gehen davon aus, dass Adjunkte auch etwas zum {\sc val}"=Wert des
  Mutterknotens beitragen. 
\item Im Prinzip ist {\sc val} dann nichts anderes als eine Menge aller Nicht"=Kopf"=Töchter in einem Baum.


\end{itemize}

}


\subsection{Passiv}

\frame{
\frametitle{Passiv}


\begin{itemize}[<+->]
\item Idee: Passiv über sogeannte Linking"=Konstruktionen, die in Vererbungshierarchien mit
  Lexikoneinträgen kombiniert werden.
\item Im Grundlexikoneintrag steht nur, welche semantischen Rollen ein Verb füllt, wie diese
  realisiert werden, wird von den jeweiligen Linking"=Konstruktionen bestimmt, mit denen der
  Grundeintrag kombiniert wird.
\item Die Idee geht auf Fillmore und Kay zurück, aber Varianten sind erst in \citew{Koenig99a} und
  \citew{MR2001a} veröffentlicht.
\item Varianten dieser Analyse wurden auch in HPSG vorgeschlagen.
\end{itemize}

}

\frame{
\frametitle{Passiv und Vererbung}

\vfill

\hfill
\begin{forest}
typehierarchy
[lexeme, for descendants={l sep+=5mm}
  [passive,name=passive, [passive $\wedge$ read, name=pr]]
  [active, name=active,  [active $\wedge$  read,  name=ar]]
  [read,   name=read     [passive $\wedge$ eat,  name=pe, no edge]]
  [eat,    name=eat,     [active $\wedge$  eat,   name=ae]] ]
\draw (passive.south)--(pe.north)
      (active.south) --(ae.north)
      (read.south)   --(pr.north)
      (read.south)   --(ar.north)
      (eat.south)    --(pe.north);
\end{forest}
\hfill\hfill\mbox{}


\vfill
}


\frame{
\frametitle{Linkingkonstruktionen}

\begin{tabular}{@{}ll@{}}
die \emph{Subject Construction}: & die \emph{Transitive Construction}:\\
\ms{
syn & \ms{ cat & v\\
       }\\
val & \menge{ \onems{ role \onems{ gf {\it subj} } }}\\
}
&
\ms{
syn & \ms{ cat & v\\
           voice & active\\
       }\\
val & \menge{ \onems{ role \ms{ gf & obj \\
                                %$\theta$ & {\sc da}$-$\\
                                {\sc da} & $-$\\
                              }\\
                    }}\\
}\\
%
die \emph{Passive Construction}:\\
\ms{
syn & \ms{ cat  & v\\
           form & PastPart\\
       }\\
val & \menge{ \ms{ role & \ms{ gf & obl\\
                               da & $+$\\
                             }\\
                   syn  & {\rm P[von]/}zero\\
                 }}\\
}
\end{tabular}

}

\frame{
\ea
Lexikoneintrag für \stem{schlag}:\\
\ms{
syn & \ms{ cat & v\\
       }\\
val & \menge{ \onems{ role \ms{ $\theta$ & agent\\
                                da       & $+$\\
                              }\\
                    }, 
              \onems{ role \ms{ $\theta$ & patient\\
                              }\\
                    }}\\
}
\z

}

\frame[shrink]{

\eal
\label{ex-schlagen-linking}
\ex 
\begin{tabular}[t]{@{}l@{}}
\stem{schlag} + Subjekt- und Transitiv"=Konstruktion:\\
\ms{
syn & \ms{ cat & v\\
           voice & active\\
       }\\
val & \menge{ \onems{ role \ms{ $\theta$ & agent\\
                                gf       & subj\\
                                da       & $+$\\
                              }\\
                    }, 
              \onems{ role \ms{ $\theta$ & patient\\
                                gf       & obj\\
                                da       & $-$\\
                              }\\
                    }}\\
}
\end{tabular}
\ex \stem{schlag} + Subjekt- und Passiv"=Konstruktion:\\
\ms{
syn & \ms{ cat & v\\
           form & PastPart\\
       }\\
val & \menge{ \ms{ role & \ms{ $\theta$ & agent\\
                                gf       & obl\\
                                da       & $+$\\
                              }\\
                   syn  & {\rm P[von]/}zero\\
                  }, 
              \onems{ role \ms{ $\theta$ & patient\\
                                gf       & subj\\
                              }\\
                    }}\\
}
\zl

\eal
\ex Er schlägt den Weltmeister.
\ex Der Weltmeister wird (von ihm) geschlagen.
\zl


}

\subsubsection{Kritik}

\frame{
\frametitle{Kritik: Mengenbegriff}


\begin{itemize}
\item Die Analyse ist formal inkonsistent, da die Mengenunifikation nicht funktioniert \citep{Mueller2006d}.
\item Man kann sie reparieren, indem man die HPSG"=Formalisierung von Mengen verwendet
 \citep{ps,PM90a}.
\item Die Subjekt-, Transitiv- und Passivkonstruktion muss man dann so abändern,
  dass die Konstruktionen etwas darüber aussagen, wie ein Element in {\sc val} aussieht, statt zu verlangen, dass
  {\sc val} genau ein Element enthält.
\end{itemize}

}


\frame[shrink]{
\frametitle{Grenzen der Vererbung}

\begin{itemize}
\item Das ist ein formales Problem, das sich lösen läßt. Schwerwiegender ist das folgende empirische Problem:

\pause
\item Vererbungsbasierte Ansätze scheitern an Phänomenen, bei denen ein argumentstrukturverändernder
  Prozess mehrfach angewendet werden kann.

Beispiel: Kausativierung im Türkischen 
\citep{Lewis67a-u}:
      \ea
      Öl-dür-t-tür-t- \\
      \glt `to cause somebody to cause somebody to kill somebody'
      \z

Wenn ich sage, dass ein Wort drei mal von der Causative Construction erbt, bekomme ich nichts
anderes, als wenn ich einmal erben würde.
%% \item Anderes Beispiel in \citew[S.\,14]{GM85a}:
%%       \ea
%% angutik anna- mik taku-$\varnothing$- kqu- ji- juk

\pause
\item Für solche Phänomene braucht man Regeln, die ein linguistisches Objekt zu einem anderen,
  komplexeren in Beziehung setzen. 
\pause
\item Diese Regeln können dann das Ursprungszeichen semantisch einbetten (also \zb \emph{cause} zu \emph{kill} hinzufügen).


\end{itemize}

}



\subsection{Verbstellung}


\frame{
\frametitle{Verbstellung}

\begin{itemize}
\item Es gibt keine Arbeiten zur Verbstellung im Deutschen im Rahmen der CxG. 
\item Da es weder leere Elemente noch Transformationen gibt,\\
  können die im Rahmen anderer Theorien gewonnenen Einsichten nicht direkt umgesetzt werden \citep{Mueller2004e}.
\end{itemize}

}


\subsection{Lokale Umstellung}



\frame{
\frametitle{Lokale Umstellung}


\begin{itemize}[<+->]
\item \citet{Kay2002a} nimmt eine phrasale Konstruktion für Heavy-NP-Shift an.
\item D.\,h. für die Umstellung schwerer NPen im Englischen gibt es eine neue Regel.
\item Solche Analysen stellen einen Rückfall hinter GPSG dar,\\
      wenn man keine Meta-Regeln annimmt. 
\item Alloconstructions (\citealp[\page 116]{Goldberg2014a}; \citealp{Cappelle2006a})
\end{itemize}




}


\subsection{Fernabhängigkeiten}

\frame{
\frametitle{Fernabhängigkeiten}



\begin{itemize}[<+->]
\item In der \emph{Left Isolation Construction} gibt es eine linke Tochter und eine rechte Tochter.
Die linke Tochter entspricht dem, was aus der rechten Tochter extrahiert wurde.

\item Das fehlende Element wird mit dem Operator VAL gesucht.
      VAL liefert alle Elemente der Valenzmenge eines linguistischen Objekts und
      alle Elemente in den Valenzmengen dieser Elemente usw.

\item Es ist somit möglich unbeschränkt tief in Argument und Adjunkttöchter hineinzuschauen.

\item Der Ansatz entspricht dem Ansatz von \citet{KZ89a} im Rahmen der LFG.
\end{itemize}

}

\frame{
\frametitle{Neue Entwicklungen}

\begin{itemize}[<+->]
\item In neueren Arbeiten wurden die Mengen aufgegeben. 
\item Aus der Berkeley-Variante der CxG wurde Sign"=Based Construction Grammar
      entwickelt \citep{Sag2012a,KSF2015a,KM2019a}
\item Sign"=Based Construction Grammar benutzt den formalen Apparat der HPSG (getypte
  Merkmalstrukturen). 
\item Valenz und Sättigung wird genauso wie in HPSG behandelt.
\item Valenzänderungen werden wie in HPSG über Lexikonregeln behandelt.
\item Fernabhängigkeiten werden genauso wie in HPSG behandelt.
\item Lediglich die Organisation der Merkmale in Merkmalstrukturen unterscheidet sich von HPSG"=Arbeiten.
\end{itemize}





}
