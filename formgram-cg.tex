\section{Categorial Grammar (CG)}

\subtitle{Categorial Grammar (CG)}

\huberlintitlepage[22pt]


\outline{

\begin{itemize}
\item Introduction and basic terms
\item Phrase structure grammar and \xbar Theory
\item Government \& Binding (GB)
\item {Generalized Phrase Structure Grammar (GPSG)}
\item {Feature descriptions, feature structures and models}
\item {Lexical Functional Grammar (LFG)}
%\item PATR
\item \alert{Categorial Grammar (CG)}
\item Head-Driven Phrase Structure Grammar (HPSG)
%\item Konstruktionsgrammatik (CxG)
\item Tree Adjoning Grammar (TAG)
\end{itemize}

%\tableofcontents
}

\frame{
\frametitle{Reading material}

\citew[Chapter~8]{MuellerGT-Eng} (without 8.1.2 on semantics)

}



\frame{
\frametitle{Categorial Grammar (CG)}

\begin{itemize}
\item Categorial Grammar is the second oldest of the approaches discussed here \citep{Ajdukiewicz35a-u}.
\item Hotspots: Edinburgh, Uetrecht and Amsterdam
\item Semanticists love CG since it syntactic combination goes hand in hand with semantic combination.
\item Important articles and books:\\
      \citet{Steedman91a,Steedman2000a-u,SB2006a-u}
\end{itemize}

}

\subsection{General remarks on the representational format}

\outline{
\begin{itemize}
\item general remarks on the representational format
\item verb position
\item local reordering (aka scrambling)
\item passive
\item long distance dependencies
\item summary and classification
\end{itemize}

}


\frame{
\frametitle{Representation of valence information}


\begin{itemize}
\item complex categories replace the \subcatf of GPSG

\medskip
\begin{tabular}[t]{@{}l@{\hspace{1cm}}l}
Rule                              & Category~in~the~lexicon\\
vp $\to$ v(ditrans) np~np         & (vp/np)/np  \\
vp $\to$ v(trans) np              & vp/np  \\
vp $\to$ v(np\_and\_pp) np~pp(to) & (vp/pp)/np  \\
\end{tabular}
\medskip


%
\pause
\item Very few, very abstract rules:

\ea
\alert{Forward application}\\
X/Y * Y = X 
\z

Combine an X looking for a Y with a Y, where Y occurs to the right of X/Y.
\pause

\item Valence is encoded just once, namely in the lexicon.

Until now we had two places for this:\\
the \subcatf and the grammar rules.
\end{itemize}
}

\frame{
\frametitle{Forward application}


\ea
\alert{Forward application}\\
X/Y * Y = X 
\z

Combine an X looking for a Y with a Y, where Y occurs to the right of X/Y.

\medskip
\deriv{2}{
chased  & Mary\\
\hr     & \hr\\
vp/np   & np\\
\multicolumn{2}{@{}c@{}}{\visible<2->{\forwardapp}} \\
\multicolumn{2}{@{}c@{}}{\visible<2->{vp}}\\
%\cgmc<2->{2}{vp}\\
%\end{tabular}
}
\medskip

\pause
%\item für `/' meist Linksassoziativität \dh (vp/pp)/np = vp/pp/np

\pause

The category v is not needed any longer.

}

\frame{
\frametitle{CG proofs vs.\ trees}

\begin{itemize}
\item CG derivations may seem strange on first encounter, but you can also depict them as trees.

\bigskip

\centerline{%
\begin{forest}
sm edges
[vp
	[vp/np
		[chased]]
	[np
		[Mary]]]
\end{forest}
}

\end{itemize}


}

\frame{
\frametitle{Backward application}

\begin{itemize}
\item vp can be eliminated as well: vp = s$\backslash$np

\ea
\alert{Backward application}\\
Y * X$\backslash$\kern-.1em Y = X 
\z

\pause
\deriv{4}{
the  & cat & chased         & Mary\\
\hr  & \hr & \hr            & \hr\\
np/n & n   & (s\bs np)/np   & np\\
\multicolumn{2}{@{}c}{\visible<3->{\forwardapp}} & \multicolumn{2}{c@{}}{\visible<4->{\forwardapp}}\\
\multicolumn{2}{c}{\visible<3->{np}}             & \multicolumn{2}{c@{}}{\visible<4->{s\bs np}}\\
\multicolumn{4}{@{}c@{}}{\visible<5->{\backwardapp}}\\
\multicolumn{4}{c@{}}{\visible<5->{s}}\\
}

\pause\pause\pause
\pause
\item no explicit distinction between words and phrases:
\begin{itemize}
\item intransitive verb = verb phrase = $(s \backslash np)$
\item similarly proper names = nominal phrases = np
\end{itemize}
\end{itemize}
}


\frame{

\frametitle{Modification}

\begin{itemize}
\item optional modification:

vp $\to$ vp~pp \\
noun $\to$ noun~pp

arbitrarily many PPs after a VP or a noun
\pause
\item modifiers in general: $X \backslash X$ or $X / X$
\pause
\item premodifier for nouns:

noun $\to$ adj~noun\\

adjective = $n/n$
\pause
\item postmodifier for nouns: $n \backslash n$
\pause
\item vp modifier: $\to$  X = $s \backslash np$
\pause
\item vp modifier: $(s \backslash np) \backslash (s \backslash np)$. 

\end{itemize}
}

\frame{
\frametitle{Derivation with a Categorial Grammar}
\vfill
% Eine typische Kategorialgrammatik benutzt nur s, np und n als Hauptkategorien. Manchmal werden noch pps als
% Komplementpräpositionalphrasen zugelassen.

% Eine Ableitung in der CG ist im wesentlichen ein binär verzweigender Baum, wird aber meistens wie folgt repräsentiert:
% Ein Pfeil unter einem Paar von Kategorien zeigt an, daß diese mit einer Kombinationsregel kombiniert werden.
% Die Richtung des Pfeils gibt die Richtung der Kombination an. Das Ergebnis wird unter den Pfeil geschrieben.
% Ein Beispiel zeigt Abbildung \ref{abb-cg}.

\oneline{%
\deriv{9}{
The  & small & cat & chased       & Mary & quickly                & round                     & the & garden\\
\hr  & \hr   & \hr & \hr          & \hr  & \hr                    & \hr                       & \hr & \hr\\
np/\gruen<3>{n} & n/\gruen<2>{n}   & \gruen<2>{n}   & (s\bs np)/\gruen<4>{np} & \gruen<4>{np}   & (s\bs np)\bs \gruen<5>{(s\bs np)} & (s\bs np)\bs (s\bs np)/\gruen<7>{np} & np/\gruen<6>{n} & \gruen<6>{n}\\
     & \multicolumn{2}{c}{\visible<2->{\forwardapp}} & \multicolumn{2}{c@{}}{\visible<4->{\forwardapp}}\\
     & \multicolumn{2}{c}{\visible<2->{\gruen<3>{n}}}           & \multicolumn{2}{c@{}}{\visible<4->{\gruen<5>{s\bs np}}}\\
\multicolumn{3}{@{}c}{\visible<3->{\forwardapp}}        & \multicolumn{3}{c@{}}{\visible<5->{\backwardapp}}\\
\multicolumn{3}{@{}c}{\visible<3->{\gruen<9>{np}}}                 & \multicolumn{3}{c@{}}{\visible<5->{\gruen<8>{s\bs np}}}\\
&&&&&&&\multicolumn{2}{c@{}}{\visible<6->{\forwardapp}}\\
&&&&&&&\multicolumn{2}{c@{}}{\visible<6->{\gruen<7>{np}}}\\
&&&&&&\multicolumn{3}{c@{}}{\visible<7->{\forwardapp}}\\
&&&&&&\multicolumn{3}{c@{}}{\visible<7->{(s\bs np)\bs \gruen<8>{(s\bs np)}}}\\
&&&\multicolumn{6}{c@{}}{\visible<8->{\backwardapp}}\\
&&&\multicolumn{6}{c@{}}{\visible<8->{(s\bs \gruen<9>{np})}}\\
\multicolumn{9}{@{}c@{}}{\visible<9->{\backwardapp}}\\
\multicolumn{9}{@{}c@{}}{\visible<9->{s}}\\
}}
% Eine Kategorialgrammatik mit den beiden Multiplikationsregeln, die oben angegeben wurden, ist schwach äquivalent
% zu einer kontextfreien Grammatik. Solche Grammatiken wurden zuerst von Ajdukiewicz diskutiert. Die Äquivalenz
% zu den CFG wurde von Bar-Hillel bewiesen. Deshalb wird ein solches System auch AB genannt. 

% Obwohl AB schwach äquivalent zu kontextfreien Grammatiken ist, ermöglicht das System die Beschreibung
% verschiedener linguistischer Phänomene auf elegante Weise. In der CFG müßte man für entsprechende
% Beschreibungen Merkmale benutzen.
\vfill
}



\subsection{Verb position}

\outline{
\begin{itemize}
\item general remarks on the representational format
\item \alert{verb position}
\item local reordering (aka scrambling)
\item passive
\item long distance dependencies
\item summary and classification
\end{itemize}

}


\subsubsection{Variable branching}

\frame{
\frametitle{Verb position}

\begin{itemize}
\item \citet[\page 159]{Steedman2000a-u} for Dutch:

\eal
\ex verb-final \emph{gaf} (`give'): (s\sub{+SUB}$\backslash$np)$\backslash$np
\ex verb-initial \emph{gaf} (`give'): (s\sub{$-$SUB}/np)/np
\zl

One item takes arguments to the left the other one to the right.

\pause
\item Lexical items are related by lexical rule.

\end{itemize}


}
\frame{
\frametitle{Comment on variable branching analysis}


Note that NPs are combined in different orders:\\
To get normal order, one would have to assume:

\eal
\ex verb-final: (s\sub{+SUB}$\backslash$np[nom])$\backslash$np[acc]
\ex verb-initial: (s\sub{$-$SUB}/np[acc])/np[nom]
\zl

\medskip
\hfill
\scalebox{.8}{%
\begin{forest}
[s\sub{$-$SUB}
  [{s\sub{$-$SUB}/\blau{np[acc]}}
    [{(s\sub{$-$SUB}/np[acc])/\gruen{np[nom]}}]
    [\gruen{np[nom]}]]
  [{np[acc]}]]
\end{forest}}
\hfill
\scalebox{.8}{%
\begin{forest}
[s\sub{$+$SUB}
  [\blau{np[nom]}]
  [{s\sub{$+$SUB}$\backslash$\blau{np[nom]}}
    [\gruen{np[acc]}]
    [{(s\sub{$+$SUB}$\backslash$np[nom])$\backslash$\gruen{np[acc]}}]]]
\end{forest}
}  
\hfill\mbox{}
\medskip

Two different branchings. So \citew{Mueller2005c} for criticism.


}


\subsubsection{Verb position with empty element}

\frame{
\frametitle{Verb position with empty element}

\citet{Jacobs91a}: empty element in final position\\
taking the arguments of the verb and the verb in initial position as arguments.


}


\subsection{Local reordering}

\outline{
\begin{itemize}
\item general remarks on the representational format
\item verb position
\item \alert{local reordering (aka scrambling)}
\item passive
\item long distance dependencies
\item summary and classification
\end{itemize}

}


\frame{
\frametitle{Local reordering}


\begin{itemize}
\item Until now: combinations either to the left or to the right.\\
      Combinations always in a fixed order from outside inwards.

\pause
\item \citet{SB2006a-u} distinguish: 
      \begin{itemize}
      \item languages in which the order of combination does not matter
\pause
      \item languages in which the direction of combination does not matter
      \end{itemize}

\pause
\medskip
\begin{tabular}{@{}lll@{}}
English   & (s$\backslash$np)/np     & S(VO)\\
Latin     & s\{$|$np[nom], $|$np[acc] \} & free order\\
Tagalog   & s\{/np[nom], /np[acc] \} & free order, verb-initial\\
Japanese  & s\{$\backslash$np[nom], $\backslash$np[acc] \} & free order, verb-final\\
\end{tabular}

\medskip
Elements in brackets can be combined with s in any order.

`$|$' instead of `$\backslash$' or `/' means that direction of combination is free.

\end{itemize}

}



\subsection{Passive}

\outline{
\begin{itemize}
\item general remarks on the representational format
\item verb position
\item local reordering (aka scrambling)
\item \alert{passive}
\item long distance dependencies
\item summary and classification
\end{itemize}

}


\frame{
\frametitle{Passive: A lexical rule}

\begin{itemize}
\item Lexical rule (\citealp[\page412]{Dowty78a}; \citealp[Section~3.4]{Dowty2003a}):

\ea
\label{Lexikonregel-Passiv-CG}
\begin{tabular}[t]{@{}l@{~$\to$~}l@{}}
%Syntax:   & 
$\alpha \in$ (s\bs np)/np & PST-PART($\alpha$) $\in$ PstP/np$_{by}$\\
%Semantics: & $\alpha'$                 & $\lambda y\lambda x \alpha'(y) (x)$
\end{tabular}
\z

For every (strictly) transitive verb $\alpha$, there is a past participle form with the category
PstP/np$_{by}$.

np$_{by}$ stands for the \emph{by}-PP.

\item example:
\eal
\ex touch:   (s\bs np)/np
\ex touched: PstP/np$_{by}$ 
\zl

\end{itemize}

}

\frame{
\frametitle{Passive: An example derivation}

\vfill
\centerline{%
\deriv{5}{%
John & was            & touched      & by         & Mary.\\
\hr  & \hr            & \hr_{\mathrm{LR}}  & \hr        & \hr\\
np   & (s\bs np)/\mathit{PstP} & \mathit{PstP}/np_{by} & np_{by}/np & np\\
     &                &              & \multicolumn{2}{c@{}}{\forwardapp}\\
     &                &              & \multicolumn{2}{c@{}}{np_{by}}\\
     &                & \multicolumn{3}{c@{}}{\forwardapp}\\
     &                & \multicolumn{3}{c@{}}{\mathit{PstP}}\\
     & \multicolumn{4}{c@{}}{\forwardapp}\\
     & \multicolumn{4}{c@{}}{s\bs np}\\
\multicolumn{5}{@{}c@{}}{\backwardapp}\\
\multicolumn{5}{@{}c@{}}{s}\\
}}
\vfill

}


\frame{
\frametitle{And German?}

\begin{itemize}
\item Well, due to the possibility of reordering items, we have sets:
\eal
\ex lieben `to love': s\sub{+SUB} \{ $\backslash$np[nom]$_i$, $\backslash$np[acc]$_j$ \}
\ex geliebt `loved': s\sub{pas} \{ $\backslash$np[nom]$_j$, $\backslash$pp[von]$_i$ \}
\zl
\pause
\item Passive rule would be different for German and English.
\end{itemize}



}



\subsection{Long distance dependencies}

\outline{
\begin{itemize}
\item general remarks on the representational format
\item verb position
\item local reordering (aka scrambling)
\item passive
\item \alert{long distance dependencies}
\item summary and classification
\end{itemize}

}


\frame{
\frametitle{Long distance dependencies}

\begin{itemize}
\item \citet[Section~1.2.4]{Steedman89a}: analysis of long distance dependencies without movement
  and empty elements.

\eal
\ex\label{Bsp-these-apples}
These apples, Harry must have been eating.
\ex apples which Harry devours
\zl
\pause
\item \emph{Harry must have been eating} and \emph{Harry devours} are just s/np.

\pause
\item But the missing np is missing at the end of the clause. We need an extension!

Type raising.

\end{itemize}




}


\subsubsection{Type Raising}

\frame{
\frametitle{Type Raising}



The category np can be transformed into the category (s/(s\bs np)) by \emph{type raising}. 
Combining this category with (s\bs np) yields the same result as combining np and (s\bs
np) with backward application.

\eal
\ex \blau<3>{np} * \gruen<2-3>{s$\backslash$np} $\to$ s 
\ex \gruen<4-5>{s/(s$\backslash$np)} * \blau<5>{s$\backslash$np} $\to$ s
\zl

\pause

Type raising simply reverses the direction of selection:\\
%\begin{itemize}
%\item[a:]
a: vp is the \gruen<2-3>{functor} and the np is the \blau<3>{argument}\\
\pause\pause
%\item[b:] 
b: type raised np is the \gruen<4-5>{functor}, and the vp is the \blau<5>{argument}.
\pause\pause
%\end{itemize}

The result is the same: s.



}


\subsubsection{Forward and backward composition}

\frame{
\frametitle{Forward and backward composition}

\begin{itemize}
\item Two additional means of combination: forward and backward composition:
\eal
\ex\label{Regel-Vorwaertskomposition}
 Forward composition (> B)\\
    X/Y $*$ Y\kern-.1em /Z = X/Z 
\ex Backward composition (< B)\\
    Y\bs Z $*$ X\bs\kern-.1em Y = X\bs Z
\zl
\bigskip 
\pause
\item Example forward composition:

\ea
 Forward composition (> B)\\
    \gruen<2>{X/\gruen<5>{Y}} $*$ \gruen<3,5>{Y}\kern-.1em/\gruen<4>{Z} = \gruen<5>{X/Z} 
\z

If I find a Y, then I am a complete X.
\pause
\item I have a Y, \pause but a Z is missing.
\pause
\item If I combine X/Y with Y\kern-.1em/Z despite the missing Z,\\
      I get something still lacking a Z.
\end{itemize}

}

\frame{
\frametitle{Forward and backward composition: Passing the np on}

\vfill
\centerline{%
\deriv{6}{
These\;apples  & Harry                & must & have & been & eating\\
\hr           & \forwardt            & \hr  & \hr  & \hr  & \hr\\
%
%
np            & s/\gruen<2>{(s\bs np)}        & \gruen<2>{(s\bs np)}/vp & \gruen<3>{vp}/vp\mathdash en & \gruen<4>{vp\mathdash en}/vp\mathdash ing & \gruen<5>{vp\mathdash ing}/np\\
              & \multicolumn{2}{c}{\visible<2->{\forwardc}}\\
              & \multicolumn{2}{c}{\visible<2->{{s/\gruen<3>{vp}}}}\\
              & \multicolumn{3}{c}{\visible<3->{\forwardc}}\\
              & \multicolumn{3}{c}{\visible<3->{{{s}/\gruen<4>{vp\mathdash en}}}}\\
              & \multicolumn{4}{c}{\visible<4->{\forwardc}}\\
              & \multicolumn{4}{c}{\visible<4->{{{s}/\gruen<5>{vp\mathdash ing}}}}\\
              & \multicolumn{5}{c@{}}{\visible<5->{\forwardc}}\\
              & \multicolumn{5}{c@{}}{\visible<5->{{{s}/np}}}\\
}}
\vfill

}

\frame{
\frametitle{The top of the dependency: The topicalization rule}

\citet{Steedman89a}:\\
rule for turning an X into a functor selecting a sentence lacking an X:
\ea
\label{Regel-Topikalisierung}
Topicalization ($\uparrow$\is{$\uparrow$}):\\
X $\Rightarrow$ st/(s/X)\\
where X $\in$ \{ np, pp, vp, ap, s$'$ \}
\z


}

\frame{
\frametitle{Topicalization long distance}

\centerline{%
\deriv{6}{
These\;apples  & Harry                & must & have & been & eating\\
\forwardtop   & \forwardt            & \hr  & \hr  & \hr  & \hr\\
%
%
\gruen<1>{st/\gruen<2>{(s/np)}}\;\;     & s/{(s\bs np)}        & {(s\bs np)}/vp & vp/vp\mathdash en & vp\mathdash en/vp\mathdash ing & vp\mathdash ing/np\\
              & \multicolumn{2}{c}{\forwardc}\\
              & \multicolumn{2}{c}{{{s}/vp}}\\
              & \multicolumn{3}{c}{\forwardc}\\
              & \multicolumn{3}{c}{{{s}/vp\mathdash en}}\\
              & \multicolumn{4}{c}{\forwardc}\\
              & \multicolumn{4}{c@{}}{{{s}/vp\mathdash ing}}\\
              & \multicolumn{5}{c@{}}{\forwardc}\\
              & \multicolumn{5}{c@{}}{\gruen<2>{s/np}}\\
\multicolumn{6}{@{}c@{}}{\visible<2>{\forwardapp}}\\
\multicolumn{6}{@{}c@{}}{\visible<2>{st}}\\
}}

}

\frame{
\frametitle{Topicalization across clause boundaries}

\centerline{%
\deriv{6}{
Apples        & I             & believe        & that & Harry         & eats\\
\forwardtop   & \forwardt     & \hr            & \hr  & \forwardt           & \hr\\
%
%
st/\gruen<6>{(s/np)}\;\; & s/\gruen<2>{(s\bs np)}  & \gruen<2>{(s\bs np)}/s'    & \gruen<4>{s'}/s & s/\gruen<3>{(s\bs np)} & \gruen<3>{(s\bs np)}/np\\
              & \multicolumn{2}{c@{}}{\visible<2->{\forwardc}}  &      & \multicolumn{2}{c@{}}{\visible<3->{\forwardc}}\\
              & \multicolumn{2}{c@{}}{\visible<2->{s/\gruen<4>{s'}}}       &      & \multicolumn{2}{c@{}}{\visible<3->{\gruen<5>{s}/np}}\\
              & \multicolumn{3}{c@{}}{\visible<4->{\forwardc}}\\
              & \multicolumn{3}{c@{}}{\visible<4->{s/\gruen<5>{s}}}\\
              & \multicolumn{5}{c@{}}{\visible<5->{\forwardc}}\\
              & \multicolumn{5}{c@{}}{\visible<5->{\gruen<6>{s/np}}}\\
\multicolumn{6}{@{}c@{}}{\visible<6->{\forwardapp}}\\
\multicolumn{6}{@{}c@{}}{\visible<6->{st}}\\
}}


}

\frame{
\frametitle{Extraction from the middle?}

\medskip

\hfill
\scalebox{.8}{%
\begin{forest}
sm edges
[S
  [NP [Fido]]
  [S/NP
    [NP [we]]
    [VP/NP
      [V$'$/NP
        [V [put]]
        [NP [\trace]]]
      [PP [downstairs]]]]]
\end{forest}}
\hfill
\visible<2->{%
\scalebox{.8}{%
\begin{forest}
sm edges
[st
  [st/pp
    [(st/\rot<3-4>{pp})/(s/\rot<3-4>{pp}/np) [Fido]]
    [s/pp/np
      [s/(s\bs np) [we]]
      [(s\bs np)/\rot<4>{pp}/np [put]]]]
  [pp [downstairs]]]
\end{forest}}}
\hfill\mbox{}
\begin{itemize}
\item Extraction from the middle is unproblematic in a GPSG-style analysis.
\pause
\item CG would look correspond to the tree on the right.
\pause
\item But we neither have the category for \emph{Fido}\pause{}
nor can we combine \emph{we} and \emph{put}. 
\end{itemize}

}


\frame{
\frametitle{Additional rules}

\begin{itemize}
\item We can combine Y with Y missing two things:
\ea
Forward composition for n=2 (> BB)\\*
X/Y $*$ (Y/Z1)/Z2 = (X/Z1)/Z2
\z

\pause
\item Topicalization turns X2 into a functor:

\ea
\label{Regel-Topikalisierung-zwei}
Topicalization\is{topicalization} for n=2 ($\uparrow\uparrow$\is{$\uparrow\uparrow$}):\\
\rot<4>{X2} $\Rightarrow$ \gruen<3>{(st/X1)}/\gruen<4>{((s/X1)/}\rot<4>{X2}\gruen<4>{)}\\
where X1 and X2 $\in$ \{ NP, PP, VP, AP, S$'$ \}
\z

\pause
The result of the combination is something that still needs the element from the right periphery of
the clause (X1). 

\pause
Something with the gap (X2) at the outside is selected.

\end{itemize}

}


\frame{
\frametitle{Analysis of fronting middle argument}

\centerline{%
\deriv{4}{
Fido                 & we            & put        & downstairs\\
\forwardtoptop       & \forwardt     & \hr        & \hr  \\
%
%
(st/pp)/\gruen<3>{((s/pp)/np)}  & s/\gruen<2>{(s\bs np)}  & (\gruen<2>{(s\bs np)}/pp)/np    & \gruen<4>{pp}\\
                     & \multicolumn{2}{c@{}}{\visible<2->{\forwardczwei}}  \\
                     & \multicolumn{2}{c@{}}{\visible<2->{\gruen<3>{(s/pp)/np}}}   \\
%
\multicolumn{3}{@{}c@{}}{\visible<3->{\forwardapp}}  \\
\multicolumn{3}{@{}c@{}}{\visible<3->{st/\gruen<4>{pp}}}   \\
\multicolumn{4}{@{}c@{}}{\visible<4->{\forwardapp}}\\
\multicolumn{4}{@{}c@{}}{\visible<4->{st}}\\
}}

}

\subsection{Summary and classification}

\outline{
\begin{itemize}
\item general remarks on the representational format
\item verb position
\item local reordering (aka scrambling)
\item passive
\item long distance dependencies
\item \alert{summary and classification}
\end{itemize}

}


\frame{
\frametitle{Summary and classification}

\begin{itemize}
\item lexical and phrasal approaches
\item headless constructions
\item relative clauses and nonlocal dependencies
\end{itemize}

}

\subsubsection{Lexical and phrasal approaches}

\frame{
\frametitle{Lexical and phrasal approaches}

\begin{itemize}
\item GPSG: approaches with valence in rules have problems with
\begin{itemize}
\item morphology
\pause
\item partial fronting
\end{itemize}
\pause
\item This also carries over to phrasal approaches in Construction Grammar.\\
 See \citew{MWArgSt} and \citew[Chapter 21]{MuellerGT-Eng} for extensive discussion.
\pause
\item Construction Grammarians often argue for phrasal approaches based on language acquisition,
  which is pattern-based, but look:

\end{itemize}

}

\frame{
\frametitle{Trees are determined lexically}

\hfill
\begin{forest}
sm edges
[s
  [np [the dolphin,roof]]
  [s\bs np 
    [(s\bs np)/np [attacks]]
    [np [the shark,roof]]]]
\end{forest}
\hfill
\visible<2->{\begin{forest}
sm edges
[s
  [np [the dolphin,roof]]
  [s\bs np
    [(s\bs np)/(s\bs np) [severely]]
    [s\bs np 
      [(s\bs np)/np [attacks]]
      [np [the shark,roof]]]]]
\end{forest}}
\hfill\mbox{}

The pattern [Subj Verb Obj] is completely determined by (s\bs np)/np.\\
The lexicon tells the syntax what to do!

\pause
And there is room for adjuncts!

}

\frame{
\frametitle{Headless constructions}

\begin{itemize}
\item CG has very few combinatorial schemata.\\
      They all assume a functor and an argument.
\pause
\item But there are constructions where it is difficult/impossible to argue for a head.

\citet{Matsuyama2004a} and \citet{Jackendoff2008a} discuss the NPN Construction:
\eal
\ex Student after student left the room.
\ex
\label{minimalism:ex-npn-iteration}
Day after day after day went by, but I never found the courage to talk to
her.\footnote{
\citet{Bargmann2015a}
}
\zl
\pause
\item This really seems to be a phrasal pattern.\\
      GPSG, CxG, HPSG, LFG, TAG can do this, Minimalism, CG, DG can't.\\ (but see \citealp{Hudson2021a} on a Word Grammar solution)

\end{itemize}






}


\subsubsection{Relative clauses and nonlocal dependencies}


\frame{
\frametitle{Relative clauses and nonlocal dependencies}

\citet[\page 614]{SB2006a-u}:
\ea
the man that Manny says Anna married
\z
\pause

Lexical entry for relative pronoun:
\ea
(n$\backslash$n)/(s/np)
\z
If I find a sentence missing an NP to the right of me,\\
I can form a noun modifier (n$\backslash$n) with it. 

The relative pronoun is the head (functor) in this analysis.

}


\frame{
\frametitle{Relative clauses and nonlocal dependencies}


\deriv{5}{
that                                & Manny                                                   & says                              & Anna                 & married\\
\hr                                 & \forwardt                                               & \hr                               & \forwardt            & \hr \\
%
%
(n\bs n)/\braun<5->{(s/np)} & s/\rot<2->{(s\bs np)}                           & \rot<2->{(s\bs np)}/s     & s/\gruen<3->{(s\bs np)} & \gruen<3->{(s\bs np)}/np\\
                                    & \multicolumn{2}{c}{\visible<2->{\forwardc}} & \multicolumn{2}{c@{}}{\visible<3->{\forwardc}}\\
%
%
                                    & \multicolumn{2}{c@{}}{\visible<2->{s/\blau<4->{s}}}                    & \multicolumn{2}{c@{}}{\visible<3->{\blau<4->{s}/np}}\\
                                    & \multicolumn{4}{c@{}}{\visible<4->{\forwardc}}\\
                                    & \multicolumn{4}{c@{}}{\visible<4->{\braun<5->{s/np}}}\\
\multicolumn{5}{@{}c@{}}{\visible<5->{\forwardapp}}\\
\multicolumn{5}{@{}c@{}}{\visible<5->{n\bs n}}\\
}



}


\frame{
\frametitle{Remark regarding this analysis}


\citet{Pollard88a}: relative pronoun = head? What about pied piping?


\eal
\ex Here's the minister [[in [the middle [of [whose sermon]]]] the dog barked].\footnote{
\citet[\page 212]{ps2}
}
\ex Reports [[the height of the lettering on the covers of which] the government prescribes] should be
abolished.\footnote{
\citet[\page 109]{Ross67}\nocite{Ross86a-u}
}
\zl

See \citew{Morrill95a,Steedman97a} for proposals.

}

\subsubsection{Summary}

\frame{
\frametitle{Summary}

\begin{itemize}[<+->]
\item simple combinatory rules
\item always functor-based
\item nonlocal dependencies without empty elements but with composition

Results in unusual constituents, but \citet{Steedman89a} argues that they are needed for coordination.
\end{itemize}


}

\subsection{Homework}

\frame{
\frametitle{Homework}


Analyze the sentence:
\ea
The children in the room laugh loudly.
\z

}


%      <!-- Local IspellDict: en_US-w_accents -->
