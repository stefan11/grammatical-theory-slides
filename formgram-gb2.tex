%% -*- coding:utf-8 -*-
\subsection{Verb position and nonlocal dependencies}

\subtitle{Government \& Binding: Verb position and long distance dependencies}

\huberlintitlepage[22pt]


\frame{
\frametitle{Reading material}

\citew[Section~3.2--3.3]{MuellerGT-Eng}


}


%\if 0


\subsubsection{Excursus: The English CP and IP}


\frame{
\frametitle{Excursus: The English CP/IP/VP system}

\begin{itemize}
\item Often the grammars of languages are modeled after suggestions for English.
\pause
\item Reasoning: Grammars are formed/limited by UG.\\
       We know that English has property X, hence\\
       all languages have property X.\\
       Caution: This is not a valid inference.
\pause
\item In order to understand the particular analysis discussed here,\\
      we first have to look at English.
\end{itemize}


}


\frame{
\frametitle{English clauses with complementizer}



\hfill\scalebox{0.73}{%
\begin{forest}
sm edges
[CP
[C$'$
	[C$^0$[that]]
	[IP
		[NP[Ann,roof]]
		[I$'$
			[I$^0$[will]]
			[VP
				[V$'$
					[V$^0$[read]]
					[NP[the newspaper, roof]]]]]]]]
\end{forest}}\hfill\hfill\mbox{}

\begin{itemize}
\item The complementizer (\emph{that}, \emph{because}, \ldots) requires an IP.
\end{itemize}

}

\subsubsubsection{The English CP, IP and VP}

\frame[shrink]{
%\frame{
\frametitle{The English CP, IP and VP: Questions}

% \savespace\small\smallexamples\parskip0pt\itemsep0pt
% \centerline{%
% \scalebox{.8}{
% \begin{forest}
% sm edges
% [IP
% 		[NP[Ann,roof]]
% 		[I$'$
% 			[I$^0$[\trace$_k$]]
% 			[VP
% 				[V$'$
% 					[V$^0$[read]]
% 					[NP[the newspaper,roof]]]]]]
% \end{forest}}}

\begin{itemize}
\item Ye/no questions are formed by fronting the auxiliary:
\ea
Will Ann read the newspaper?
\z
\item The auxiliary moves to the position of the complementizer.
\pause
\item \emph{wh} questions are formed by additionally preposing a constituent:
\ea
What will Ann read?
\z
\end{itemize}

}


\frame[shrink]{
\frametitle{English CP, IP and VP: Questions}

\vfill

\hfill
\scalebox{.85}{
\begin{forest}
sm edges
[CP
[C$'$
	[C$^0$[will$_k$]]
	[IP
		[NP[Ann,roof]]
		[I$'$
			[I$^0$[\trace$_k$]]
			[VP
				[V$'$
					[V$^0$[read]]
					[NP[the newspaper,roof]]]]]]]]
\end{forest}}
\hfill
\visible<2->{\scalebox{.85}{
\begin{forest}
sm edges
[CP
[NP$_i$ [what]]
[C$'$
	[C$^0$[will$_k$]]
	[IP
		[NP[Ann,roof]]
		[I$'$
			[I$^0$[\trace$_k$]]
			[VP
				[V$'$
					[V$^0$[read]]
					[NP[\trace$_i$]]]]]]]]
\end{forest}}}
\hfill\mbox{}

\vfill

\pause

}




\subsubsection{Topology of the German clause}

\frame{
\frametitle{Topology of the German clause (I)}

Before turning to the CP/IP system in grammars of German we have to sort out some terminology:

\begin{itemize}
\item Approaches to German constituent order often refer to topological fields.

\pause
\item Important works on topological fields are:\\
\citew{Drach37},  \citew{Reis80a} and \citew{HoehleTopo,Hoehle86}.

\pause
\item We will use \alert{Vorfeld}, \alert{linke/rechte Satzklammer},
\alert{Mittelfeld} and \alert{Nachfeld}.

\citew{Bech55a} introduced further fields for verbal complexes,\\
 but we will ignore them here.
\end{itemize}


}

\frame{
\frametitle{Verb positions and terminology}

\savespace
\begin{itemize}
\item Verb-final position
      \ea
\gll Peter hat erzählt, \rot<5->{dass} er das Eis \braun<5->{\gruen<4>{gegessen}} \braun<5->{\blauit<-3>{\gruen<4>{hat}}}.\\
     Peter has told that he the ice.cream eaten has\\
%\glt `Peter said that he has eaten the ice cream.'
      \z
\pause
\item Verb-initial position
        \ea
\gll      \rot<4->{\blauit<-3>{Hat}} Peter das Eis \braun<5->{\gruen<4>{gegessen}}?\\
	 has Peter the ice.cream eaten\\
%\glt `Has Peter eaten the ice cream?'
      \z
\pause
\item Verb-second poisiton
      \ea
\gll  Peter \rot<4->{\blauit<-3>{hat}} das Eis \braun<5->{\gruen<4>{gegessen}}.\\
	 Peter has the ice.cream eaten\\
%\glt `Peter has eaten the ice cream.'
      \z
\end{itemize}


\pause
\begin{itemize}[<+->]
\item verbal elements continuous in (\mex{-2}) only
\item \rot<5->{left} and \braun<5->{right} sentence bracket
\item complementizer (\emph{weil}, \emph{dass}, \emph{ob}) in left sentence bracket
\item complementizer and finite verb have complementary distribution \citep{Hoehle97a}
\item region before, between and after the brackets: \alert{Vorfeld}, \alert{Mittelfeld}, \alert{Nachfeld}
\end{itemize}


}

\frame{
\frametitle{Topology of German clauses}


\resizebox{\textwidth}{!}{
\begin{tabular}{@{}lllll@{}}
\rowcolor{structure!15}Vorfeld & left bracket & Mittelfeld                             & right bracket & Nachfeld                   \\ 
\\
\rowcolor{structure!10}Karl    & schläft.      &                                        &                &                            \\
                       Karl    & hat           &                                        & geschlafen.    &                            \\
\rowcolor{structure!10}Karl    & erkennt       & Maria.                                 &                &                            \\
                       Karl    & färbt         & den Mantel                             & um             & den Maria kennt.           \\
\rowcolor{structure!10}Karl    & hat           & Maria                                  & erkannt.       &                             \\
                       Karl    & hat           & Maria als sie aus dem Zug stieg sofort & erkannt.       &                             \\
\rowcolor{structure!10}Karl    & hat           & Maria sofort                           & erkannt        & als sie aus dem Zug stieg. \\
                       Karl    & hat           & Maria zu erkennen                      & behauptet.     &            \\
\rowcolor{structure!10}Karl    & hat           &                                        & behauptet      & Maria zu erkennen.         \\ 
\\
\rowcolor{structure!10}        & Schläft       & Karl?                                  &                &                            \\
                               & Schlaf!       &                                        &                &                             \\
\rowcolor{structure!10}        & Iß            & jetzt dein Eis                         & auf!           &                             \\
        & Hat           & er doch das ganze Eis alleine          & gegessen.            &      \\  \\
\rowcolor{structure!10}        & weil          & er das ganze Eis alleine               & gegessen hat   & ohne sich zu schämen.\\
        & weil          & er das ganze Eis alleine               & essen können will    & ohne gestört zu werden.    \\
\rowcolor{structure!10}wer     &               & das ganze Eis alleine                  & gegessen hat.  &                             \\
\end{tabular}
}

}

% \frame{
% \frametitle{Der Prädikatskomplex}
% %


% \begin{itemize}
% \item<+-> mehrere Verben in der rechten Satzklammer: Verbalkomplex
% \item<+-> manchmal wird auch von diskontinuierlichen Verbalkomplexen
%       gesprochen (Initialstellung das Finitums)
% \item<+-> auch prädikative Adjektive (\mex{1}a) und Resultativprädikate (\mex{1}b) werden zum Prädikatskomplex gezählt:
%       \eal
%       \ex dass Karl seiner Frau treu ist
%       \ex dass Karl das Glas leer trinkt
%       \zl
% \end{itemize}

% }

\frame{
\frametitle{The Rangprobe}
%


\begin{itemize}
\item<+-> Fields may be empty.
      \ea
      \field{Der Delphin}{VF} \field{gibt}{LS} \field{dem Kind den Ball,}{MF} \field{das er kennt}{NF}.
      \z
\item<+-> Test: Rangprobe \citep[\page 72]{Bech55a}
\eal
\ex[]{
\gll Der Delphin hat [\rot{dem} \rot{Kind}] den Ball gegeben, [\rot{das} \rot{er} \rot{kennt}].\\
     the dolphin has \spacebr{}the child the ball given \spacebr{}who he knows\\
\glt `The dolphin has given the ball to the child who it knows.'
}
\pause
\ex[*]{
\gll Der Delphin hat [\rot{dem} \rot{Kind}] den Ball, [\rot{das} \rot{er} \rot{kennt},] gegeben.\\
     the dolphin has \spacebr{}the child the ball \spacebr{}who he knows given\\
}
\zl
Replacing the finite verb by an auxiliary forces the main verb into the right sentence bracket.
\pause
\ea{
\gll Der Delphin hat [\rot{dem} \rot{Kind}, \rot{das} \rot{er} \rot{kennt},] den Ball gegeben.\\
     the dolphin has \spacebr{}the child who he knows the ball given\\
}
\z

\end{itemize}

}


\frame[shrink=10]{
\frametitle{Recursion}

\begin{itemize}
\item \citet*[\page 82]{Reis80a}: Recursion: Vorfeld can contain other topological fields:
\eal
\label{Beispiel-topologisch-komplexes-Vorfeld}
\ex
\gll Die Möglichkeit, etwas zu verändern, ist damit verschüttet für lange lange Zeit.\\
	 the possibility something to change is there.with buried for long long time\\
\glt `The possibility to change something will now be gone for a long, long time.'	  
\ex 
\gll {}[Verschüttet für lange lange Zeit] ist damit die Möglichkeit,      etwas zu ver"-ändern.\\
      \spacebr{}buried for long long time ist there.with the possibility  something to change\\
\pause
\ex 
\gll Wir haben schon seit langem gewußt, daß du kommst.\\
     we have \particle{} since long known that you come\\
\glt `We have known for a while that you are coming.'
\ex 
\gll {}[Gewußt, daß du kommst,] haben wir schon seit langem.\\
	 \spacebr{}known that you come have we \particle{} since long\\
\zl
% \pause
% \item Permutations occurring in the Mittelfeld can also take place within complex Vorfleds.

% \eal
% \ex {}[\gruen{Seiner Tochter} \blau{ein Märchen} erzählen] wird er wohl müssen.
% \ex {}[\blau{Ein Märchen} \gruen{seiner Tochter} erzählen] wird er wohl müssen.
% \zl
\end{itemize}

}


\frame{
\frametitle{Exercise}

Assign topological fields in the sentences in (\mex{1}):
\eal
\ex Der Mann hat gewonnen, den alle kennen.
\ex Sie gibt ihm das Buch, das Conny empfohlen hat.
\ex Maria hat behauptet, dass das nicht stimmt.
\ex Conny hat das Buch gelesen,\\das Maria der Schülerin empfohlen hat,\\die neu in die Klasse gekommen ist.
\ex Komm!
\zl

}

\subsubsection{The German CP and IP}

\frame{
\frametitle{The topological model paired with CP, IP, VP (I)}

\vfill
\centerline{
\scalebox{.7}{
        \begin{forest}
            sn edges original,empty nodes
            [CP
              [{}
                [XP,terminus
                  [SpecCP\\prefield, name=p1
                  ]
                ]
              ]
              [\hspaceThis{$'$}C$'$
                    [{}
                      [C, terminus
                        [C \\left SB, name=c0
                        ]
                      ]
                    ]
                    [IP
                      [{}
                        [XP, terminus
                          [{IP (without I, V )\\middle field}
                            [SpecIP\\subject position, set me left, name=specip
                            ]
                            [phrases inside\\the VP, name=p3
                            ]
                          ]
                        ]
                      ]
                      [\hspaceThis{$'$}I$'$
                              [VP, name=vp
                                [V, name=v0, terminus, no path, anchor=east
                                  [{V , I \\right SB}, name=p2, set me left
                                  ]
                                ]
                              ]
                              [{}
                                    [I , terminus, name=io
                                    ]
                              ]
                      ]
                    ]
              ]
            ]
            \draw [thick]
              (p1.north west) rectangle (io.east |- p3.south);
            \draw
              ($(c0.north east)!1/2!(specip.west |- c0.north east)$) coordinate (p6) -- (p6 |- p3.south)
              ($(p1.north east)!1/2!(c0.north west)$) coordinate (p4) -- (p3.south -| p4)
              ($(specip.north east)!1/2!(p3.north west)$) coordinate (p5) -- (p3.south -| p5)
              ($(p2.north west)!1/2!(p2.north west -| p3.east)$) coordinate (p7) -- (p3.south -| p7)
              (p6 |- p2.south) -- (p2.south -| p7)
              (vp.south) -- (v0.center -| p3.west) -- (v0.west)
              (v0.east) -- +(4.5pt,0) -- (vp.south)
              ;
        \end{forest}}}

}

% \frame{
% \frametitle{Das topologische Modell mit CP, IP, VP (II)}

% \footnotesize
% \resizebox{0.99\textwidth}{!}{
% \begin{tabular}{|l|l|l|l|l|}
% \hline
% %
% SpecCP    & \cnull      & \mc{2}{l|}{IP (ohne \inull, \vnull)} & \vnull, \inull\\
% Vorfeld   & Linke       & \mc{2}{l|}{Mittelfeld}                      & Rechte\\\cline{3-4}
%           & Satzklammer & SpecIP           & Phrasen innerhalb der VP & Satzklammer\\
%           &             & Subjektsposition &                          &\\\hline\hline
%           & dass         & Anna & [das Buch] [auf den Tisch] & legt$_k$ [ \_ ]$_k$\\
% \pause
%           & ob  & Anna & [das Buch] [auf den Tisch] & legt$_k$ [ \_ ]$_k$ \\\hline\hline
% \pause
% \ifthenelse{\boolean{gb-intro}}{
% wer$_i$      & [ \_ ] & [ t ]$_i$ & [das Buch] [auf den Tisch] & legt$_k$ [ \_ ]$_k$\\
% \pause
% was$_i$      & [ \_ ] & Anna & [ t ]$_i$ [auf den Tisch] & legt$_k$ [ \_ ]$_k$\\\hline\hline
% \pause
% }{}
%           & Legt$_k$ & Anna & [das Buch] [auf den Tisch]? & [ t ]$_k$ [ t ]$_k$\\
% \pause
%           & Legt$_k$ & Anna & [das Buch] [auf den Tisch], & [ t ]$_k$ [ t ]$_k$ \\\hline\hline
% \pause
% Anna$_i$     & legt$_k$ & [ t ]$_i$ & [das Buch] [auf den Tisch] & [ t ]$_k$ [ t ]$_k$\\
% \pause
% Wer$_i$      & legt$_k$ & [ t ]$_i$ & [das Buch] [auf den Tisch]? & [ t ]$_k$ [ t ]$_k$\\
% \pause
% {}[Das Buch]$_i$ & legt$_k$ & Anna & [ t ]$_i$ [auf den Tisch] & [ t ]$_k$ [ t ]$_k$\\
% \pause
% Was$_i$      & legt$_k$ & Anna & [ t ]$_i$ [auf den Tisch]? & [ t ]$_k$ [ t ]$_k$\\
% \pause
% {}[Auf den Tisch]$_i$ & legt$_k$ &Anna & [das Buch] [ t ]$_i$ & [ t ]$_k$ [ t ]$_k$\\\hline
% \end{tabular}
% }

% \pause
% \vfill
% Achtung: Die Bezeichner SpecCP u.\ SpecIP sind keine Kategoriensymbole. Sie kommen
% in Grammatiken mit Ersetzungsregeln nicht vor! Sie bezeichnen nur Positionen im Baum.

% }

\subsubsubsection{German as SOV language}

\frame{
\frametitle{German as SOV language}

\begin{itemize}

\item Heads of VP and IP (\vnull and \inull) are serialized to the right of their
  arguments. 

Together they form the right sentence bracket.


\pause
\item All other arguments and adjuncts are serialized to the left of them and form the Mittelfeld.

\pause
\item Typologically, German is a SOV language (basic order subject--object--verb), which is reflected at the D Structure level.

\begin{itemize}
\item SOV German, \ldots
\item SVO English, French, \ldots
\item VSO Welsh, Arabic, \ldots 
\end{itemize}
App.\ 40\,\% of all languages are SOV languages, app.\,35\,\% are SVO.

\item See \citew{MuellerGermanic} for discussion of Germanic and the classification of German.

\pause
\item Nice result of SOV structure: The closer a constituent is related to the verb, the closer it
  is to the right sentence bracket, even in sentences with inital finite verb and empty right
  sentence bracket.
\end{itemize}

}

\frame{
\frametitlefit{Motivation of SOV order as basic order: Particles}

\citew%[S.\,34--36]
{Bierwisch63a}: Verb particles form a close unit with the verb:
\eal
\ex 
\gll weil sie morgen \alert{an-fängt}\\
     because she tomorrow \textsc{part}-starts\\
\glt `because he is starting tomorrow'
\ex 
\gll Sie \alert{fängt} morgen \alert{an}.\\
     she starts tomorrow \textsc{part}\\
\glt `She is starting tomorrow.'
\zl
This unit can only be seen in verb"=final structures,\\
which speaks for the fact that this structure reflects the base order.
}

% \frame{
% \frametitle{Stellung der infiniten Verben}


% \eal
% \ex Dieses Buch sollte gelesen werden müssen.
% \ex This book should have been read.
% \zl

% }

\frame{
\frametitle{Sometimes SOV is the only option}

Sometimes SOV is the only option \citep[\page 370--371]{HoehleProjektionsstufen}:
\eal
\ex[]{
\gll weil sie das Stück heute ur-auf-führen\\
	 because they the play today \textsc{pref}-\textsc{part}-lead\\
\glt `because they are performing the play for the first time today'
}
\ex[*]{
\gll Sie ur-auf-führen heute das Stück.\\
     they \textsc{pref}-\textsc{part}-lead today the play\\
}
\ex[*]{
\gll Sie führen heute das Stück ur-auf.\\
     they lead today the play \textsc{pref}-\textsc{part}\\
}
\zl

This is backformation.\\
\emph{Ur-auf-führung} is wrongly assumed to be derived from the verb \emph{uraufführen}.

}

\frame{
\frametitle{Order in subordinated sentences}

Verbs in non-finite subordinated clauses and in finite subordinated clauses introduced by a
conjunction are positioned at the end (ignoring extraposition):
\eal
\ex 
\gll Der Clown versucht, Kurt-Martin die Ware \alert{zu} \alert{geben}.\\
     the clown tries Kurt-Martin the goods to give\\
\glt `The clown is trying to give Kurt-Martin the goods.'
\ex 
\gll dass der Clown Kurt-Martin die Ware \alert{gibt}\\
	 that the clown Kurt-Martin the goods gives\\
\glt `that the clown gives Kurt-Martin the goods'
\zl
}

\frame{
\frametitle{Order of verbs in SVO and SOV languages}

\citet{Oersnes2009b}: 
\eal
\ex 
\gll dass er ihn gesehen$_3$ haben$_2$ muss$_1$\\
	 that he him seen have must\\\hfill(German)
\ex 
\gll at han må$_1$ have$_2$ set$_3$ ham\\
     that he must have seen him\\\hfill(Danish)
\glt `that he must have seen him'
\zl

OV: embedding verbs go to the end\\
VO: embedding verbs go to the beginning

(ignore the Dutch for the moment \ldots)


}

\frame{
\frametitle{Scope}

%\citew[Section~2.3]{Netter92}:
\citew{Netter92}:
Adverbs outscope material to their right (preference only?):

\eal
\ex 
\gll dass er [absichtlich [nicht lacht]]\\
     that he \spacebr{}intentionally \spacebr{}not laughs\\
\glt `that he is intentionally not laughing'
\ex 
\gll dass er [nicht [absichtlich lacht]]\\
     that he \spacebr{}not \spacebr{}intentionally laughs\\
\glt `that he is not laughing intentionally'
\zl
\pause
The scoping does not change if the verb is in initial position:
\eal
\ex 
\gll Er lacht$_i$ [absichtlich [nicht \_$_i$]].\\
     he laughs \spacebr{}intentionally \spacebr{}not\\
\glt `He is intentionally not laughing.'
\ex 
\gll Er lacht$_i$  [nicht [absichtlich \_$_i$]].\\
     he laughs \spacebr{}not \spacebr{}intentionally\\
\glt `He is not laughing intentionally.'
\zl
}


\subsubsubsection{C -- The left sentence bracket}


\frame{
\frametitle{\cnull{} -- The left sentence bracket in embedded clauses}

\cnull corresponds to the left sentence bracket and is filled as follows:
\begin{itemize}
\item In embedded sentences with subordinating conjunction\\ the conjunction (the complementizer) is
  placed in \cnull, as in English. 

The verb stays in the right sentence bracket.
\ea
\gll dass jeder diese Frau kennt\\
     that everybody this woman knows\\
\glt `that everybody knows this woman'
\z

\pause
\item The verb moves from V to I.

\end{itemize}

}


\frame{
\frametitle{V to I movement in embedded clauses}

\centerline{%
\scalebox{.77}{%
\begin{forest}
sm edges
[CP
[\hspaceThis{$'$}C$'$
	[C [dass;that]]
	[IP
		[NP [jeder;everybody,roof]]
		[\hspaceThis{$'$}I$'$
			[VP
				[\hspaceThis{$'$}V$'$
					[NP[diese Frau;this woman, roof]]
					[V [\trace$_j$]]]]
			[I [kenn-$_j$ -t;know- -s]]]]]]
\end{forest}
}
}

}


\frame{
\frametitle{\cnull{} -- The left sentence bracket in V1 and V2 clauses}

\begin{itemize}
\item The finite verb is moved via \inull to \cnull in verb-first and verb-second clauses: \\
\vnull $\to$  \inull $\to$ \cnull. 
\eal
\settowidth\jamwidth{(verb in \vnull)}
\ex 
\gll dass jeder diese Frau kenn- -t\\
     that everybody this woman know- -s\\ \jambox{(verb in \vnull)}
\ex 
\gll dass jeder     diese Frau \_$_i$ [kenn-$_i$ -t]\\
     that everybody this  woman {}   \spacebr{}know- -s\\ \jambox{(verb in \inull)}
\ex 
\gll {}[Kenn-$_i$ -t]$_j$ jeder diese Frau \_$_i$ \_$_j$?\\
     \spacebr{}know- -s   everybody this woman\\ \jambox{(verb in \cnull)}
\zl

\end{itemize}

}

\frame{
\frametitle{V to I to C movement in V1/V2 clauses}

\centerline{%
\scalebox{.77}{%
\begin{forest}
sm edges
[CP
[\hspaceThis{$'$}C$'$
	[C [(kenn-$_j$ -t)$_k$;knows]]
	[IP
		[NP [jeder;everybody,roof]]
		[\hspaceThis{$'$}I$'$
			[VP
				[\hspaceThis{$'$}V$'$
					[NP [diese Frau; this woman, roof]]
					[V [\trace$_j$]]]]
			[I  [\trace$_k$]]]]]]
\end{forest}
}}


}

\subsubsubsection{SpecCP -- The Vorfeld}

\exewidth{(235)}

\frame{
\frametitle{SpecCP -- The Vorfeld in declarative clauses (I)}

The position SpecCP corresponds to the Vorfeld and is filled as follows:
\begin{itemize}
\item Declarative clauses: XP is moved to the Vorfeld.
\ea
\gll Gibt der Mann dem Kind jetzt den Mantel?\\
     gives the.\NOM{} man the.\DAT{} child now the.\ACC{} coat\\
\glt `Is the man going to give the child the coat now?'
\z

\eal
\ex 
\gll Der Mann gibt dem Kind jetzt den Mantel.\\
     the.\NOM{} man gives the.\DAT{} child now the.\ACC{} coat\\
\glt `The man is giving the child the coat now.'
\pause
\ex 
\gll Dem Kind gibt der Mann jetzt den Mantel.\\
     the.\DAT{} child gives the.\NOM{} man now the.\ACC{} coat\\
\pause
\ex 
\gll Den Mantel gibt der Mann dem Kind jetzt.\\
	 the.\ACC{} coat gives the.\NOM{} man the.\DAT{} child now\\
\pause
\ex 
\gll Jetzt gibt der Mann dem Kind den Mantel.\\
	 now gives the.\NOM{} man the.\DAT{} child the.\ACC{} coat\\
\zl
\end{itemize}

}


\frame{

\frametitle{Verb movement and movement to SpecCP}

\vfill
\centerline{\scalebox{0.77}{
\begin{forest}
sm edges
[CP
[NP$_i$ [diese Frau;this woman, roof]]
[\hspaceThis{$'$}C$'$
	[C [(kenn-$_j$ -t)$_k$; know- -s]]
	[IP
		[NP [jeder;everybody,roof]]
		[\hspaceThis{$'$}I$'$
			[VP
				[\hspaceThis{$'$}V$'$
					[NP[\trace$_i$]]
					[V [\trace$_j$]]]]
			[I  [\trace$_k$]]]]]]
\end{forest}}}
\vfill


}

\frame[shrink]{
\frametitle{SpecCP -- The Vorfeld in declarative clauses (II)}

\begin{itemize}
\item The crucial factor for deciding which phrase to move is the \emph{information structure} of the sentence. Material connected to previously mentioned or otherwise"=known information is 
placed further left (preferably in the prefield) and new information tends to occur to the right. Fronting to the
prefield in declarative clauses is often referred to as
\alert{topicalization}. 

\pause
\item But this is rather a misnomer, since the focus (informally: the constituent being asked for)
  can also occur in the prefield. Expletives as well.

\pause
\item Caution:\\
      Movement to the Vorfeld does not have the same status as fronting in English!
\end{itemize}

}

\frame{
\frametitle{Nonlocal dependencies}

\begin{itemize}
\item Analysis also works for nonlocal dependencies:
\ea
  \gll {}[Um zwei Millionen Mark]$_i$ soll er versucht haben,~~~~ [eine Versicherung \_$_i$ zu betrügen].\footnotemark\\
     {}\spacebr{}around two million Deutsche.Marks should he tried have \spacebr{}an insurance.company {} to deceive\\
\footnotetext{%
         taz, 04.05.2001, p.\,20.
}
\glt `He apparently tried to cheat an insurance company out of two million Deutsche Marks.'
      \z
Step-wise movement: the fronted constituent first moves to the specifier position of the phrase it
originates from than to the next specifier of the next maximal projection and so on until it reaches
the uppermost SpecCP position.
\end{itemize}

}






%      <!-- Local IspellDict: en_US-w_accents -->
