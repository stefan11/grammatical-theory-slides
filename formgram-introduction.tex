



%\if 0


\subtitle{Motivation of (formal) syntax and basic terminology}

\section{Motivation of (formal) syntax and basic terminology}


\huberlintitlepage[22pt]




\frame{
\frametitlefit{Reading material}

\begin{itemize}
\item Literature: English version of the grammatical theory textbook: \citew{MuellerGT-Eng}
\pause
\item There is also a German and a Chinese version.\\
      The fourth edition of the English book is the most recent one.
\pause
\item For this session, please read \citew[Chapter~1]{MuellerGT-Eng}.

Topological fields are covered in Section~1.8. They are not part of the slides of this session but
will be needed later on (chapter~3 and onwards).

\end{itemize}

\vspace{1cm}

%%\rotbf{Achtung, wichtiger Hinweis: Diese Literaturangabe hier bedeutet,\\dass Sie die Literatur zum
%%   nächsten Mal lesen sollen!!!!}
%% 
}


\subsection{Goals of this course}



\frame{
\frametitle{Goals of this course}


\begin{itemize}[<+->]
\item conveyance of basic ideas about grammar
\item introduction to various grammatical theories and approaches
\item enlightenment and attainment of supernatural powers
\end{itemize}


}




\frame{
\frametitle{Ancient wisdom}

{}[Grammar is] the gate to freedom, the medicine for the diseases of language, the purifier
of all sciences; it spreads its light over them; \ldots{} it is
the first rung on the ladder which leads to the realization of supernatural powers and
straight, royal road for those who seek freedom. (Bhartrhari, poet of sayings,
died before 650 AD, from \emph{Vakyapadiya}, found by Gabriele Knoll)

% {}[Grammatik ist] das Tor zur Freiheit, die Medizin für die Krankheiten der Sprache, der Reiniger
% aller Wissenschaften; sie verbreitet ihr Licht über ihnen; \ldots{} sie ist
% die erste Sprosse auf der Leiter, die zur Realisierung übernatürlicher Kräfte führt und der
% gerade, königliche Weg für diejenigen, die die Freiheit suchen. (Bhartrhari, Spruchdichter,
% gest.\ vor 650 n. Chr., aus \emph{Vakyapadiya}, gefunden von Gabriele Knoll)

}





\subsection{Why syntax?}



\frame{
\frametitle{Why syntax?}

\begin{itemize}
\item Literature: \citew[Chapter~1]{MuellerLehrbuch} or \citew[Chapter~1]{MuellerGTBuch}
\medskip

\item signs: form-meaning pairs \citep{Saussure16a}
\pause
\item words, word groups, sentences
\pause
\item language $\stackrel{?}{=}$ finite enummeration of word sequences\\
\pause
      language is finite, if onw assumes a maximal sentence length
      \eal
      \ex This sentence goes on and on and on \ldots
\pause
      \ex {}[A sentence is a sentence] is a sentence.
      \zl
\pause
      We can form enourmously many sentences.\\
A restriction on complexity would be arbitrary.

\item One distinguishes between \alert{competence} (knowledge about what is possible) and
  \alert{performance} (useage of this knowledge)

\end{itemize}
}


\frame{
\frametitle{The Six Bullerby Children}

Und wir beeilten uns, den Jungen zu erzählen, wir hätten von Anfang an gewußt, daß es nur eine
Erfindung von Lasse gewesen sei. Und da sagte Lasse, die Jungen hätten gewußt, daß wir gewußt
hätten, es sei nur eine Erfindung von ihm. Das war natürlich gelogen, aber vorsichtshalber sagten
wir, wir hätten gewußt, die Jungen hätten gewußt, daß wir gewußt hätten, es sei nur eine Erfindung
von Lasse. Und da sagten die Jungen -- ja -- jetzt schaffe ich es nicht mehr aufzuzählen, aber es
waren so viele "`gewußt"', daß man ganz verwirrt davon werden konnte, wenn man es hörte. (p.\,248)

\bigskip

We are capable of forming long, complex sentences (competence), but at some level of complexity we
get confused since our brains cannot deal with the complexity anymore (performance).


}




\frame{
\frametitle{Creativity}


\begin{itemize}
\item We can form sentences we never heard before $\to$\\
      There has to be structure, patterns.\\
      It cannot be just sequences learned by heart.

\end{itemize}


}

\frame{
\frametitle{Direct evidence for syntactic structures?}

\begin{itemize}
\item We can show that we are following rules by observing children.
      
      Children often use rules wrongly (or rather use their own rules).

\pause
\item Example from morphology: 

German has an unmarked Plural for some nouns: \emph{Bagger} `digger',
  \emph{Ritter} `knight'.

\item Children apply the \suffix{s} ending to such unmarked plurals instead:
\eal
\ex[*]{
die Baggers
}
\ex[*]{
die Ritters
}
\zl

\vfill
\pause
\item Side remark: We will use German examples throughout this course, since English is sooooo
  boring. I gloss whatever I can, but sometimes stuff would not fit onto the slide. Please refer to
  the textbook in such cases.

\end{itemize}
}


\frame[shrink=10]{
\frametitlefit{Why syntax? Computation of meaning from utterance parts}

\begin{itemize}
\item The meaning of an utterance can be computed from the meaning of its parts.
      \ea
      \gll Der Mann kennt diese Frau.\\
           the man  knows this woman\\
      \z
\pause
\item Syntax: the way parts are combined, the utterance is structured
      \eal
\ex 
\gll Die Frau kennt die Mädchen.\\
     the woman know.\textsc{3sg} the girls\\
\glt `The woman knows the girls.'
\ex 
\gll Die Frau kennen die Mädchen.\\
     the woman know.\textsc{3pl} the girls\\
\glt `The girls know the woman.'
\pause
\ex 
\gll Die Frau schläft.\\
     the woman sleep.\textsc{3sg}\\
\glt `The woman sleeps.'
\ex 
\gll Die Mädchen schlafen.\\
     the girls sleep.\textsc{3pl}\\
\glt `The girls sleep.'
      \zl
        Subject-verb agreement $\to$ meaning of (\mex{0}a,b) is unambiguous

\end{itemize}

}

\subsection{Why formal?}
\frame[shrink=20]{
\frametitle{Why formal?}


Precisely constructed models for linguistic structure can play an
important role, both negative and positive, in the process of discovery 
itself. By pushing a precise but inadequate formulation to
an unacceptable conclusion, we can often expose the exact source
of this inadequacy and, consequently, gain a deeper understanding
of the linguistic data. More positively, a formalized theory may 
automatically provide solutions for many problems other than those
for which it was explicitly designed. Obscure and intuition-bound
notions can neither lead to absurd conclusions nor provide new and
correct ones, and hence they fail to be useful in two important respects. 
I think that some of those linguists who have questioned
the value of precise and technical development of linguistic theory
have failed to recognize the productive potential in the method
of rigorously stating a proposed theory and applying it strictly to
linguistic material with no attempt to avoid unacceptable conclusions by ad hoc adjustments or loose formulation.
\citep[\page 5]{Chomsky57a}


As is frequently pointed out but cannot be overemphasized, an important goal
of formalization in linguistics is to enable subsequent researchers to see the defects
of an analysis as clearly as its merits; only then can progress be made efficiently.
\citep[\page 322]{Dowty79a}


\bigskip

\begin{itemize}
\item What does an analysis mean?
\item Which predictions does it make?
\item exclusion of alternative proposals
\end{itemize}


}

% has to be set elsewhere since this file is included into the syntax vorlesung
%\exewidth{(35)}


\subsection{Constituency}

\subsubsection{Constituency tests}

\frame{
\frametitle{Grouping words}

\begin{itemize}
\item Sentences may contain sentences containing sentences die \ldots:
\ea
that Max thinks [that Julius knows [that Otto claims [that Karl suspects [that Richard confirms [that Friederike is laughing]]]]]
\z

This works like a Russian doll or like an onion.

\pause

\item The words in (\mex{1}) can be grouped into units as well:
\ea
\gll Alle Studenten lesen während dieser Zeit Bücher.\\
     all  students  read  during  this   time books\\
\glt `All the students are reading books at this time.'
\z

Which ones?

\end{itemize}


}

\frame{
\frametitle{Boxes}

\oneline{%
\begin{pspicture}(0,0)(12,1.8)
     \rput[bl](0,0){%
\psset{fillstyle=solid, framearc=0.25,framesep=5pt}
\psframebox{%
\psframebox{%
       \psframebox{alle}
       \psframebox{Studenten}}
\psframebox{lesen}
\psframebox{%
       \psframebox{während}
       \psframebox{%
           \psframebox{dieser}
           \psframebox{Zeit}}}
\psframebox{Bücher}}}
%\psgrid
    \end{pspicture}}

We put all words belonging together into a box.

Such boxes can be put into other boxes.

It is intuitively clear what belongs into a box in the example at hand,\\
but are there tests?

}


\frame{
\frametitle{Constituency}

Terminology:
\begin{description}
\item[Word sequence]  An arbitrary linear sequence of words which do not necessarily need to have any syntactic or semantic relationship.
\item[Word group, constituent, phrase] One or more words forming a structural unit.
\end{description}


}

\frame{
\frametitle{Constituency tests}

Which ones do you know?
\pause

\begin{itemize}
\item substitution/pronominalization/question formation
\item omission
\item permutation
\item fronting
\item coordination
\end{itemize}


}


\frame{
\frametitle{Constituency tests (I)}


\begin{description}
\item[Substitution]
If it is possible to replace a sequence of words in a sentence with a different sequence of words\is{substitution test} and the acceptability of the sentence 
remains unaffected, then this constitutes evidence for the fact that each sequence of words forms a constituent.
        \eal
\ex 
\gll Er kennt [den Mann].\\
     he knows \spacebr{}the man\\
\glt `He knows the man.'
\ex 
\gll Er kennt [eine Frau].\\
     he knows \spacebr{}a woman\\
\glt `He knows a woman.'
        \zl
\end{description}

}


\frame{
\frametitle{Constituency tests (II)}


\begin{description}
\item[Pronominalization]
Everything\is{pronominalization test} that can be replaced by a pronoun forms a constituent.
        \eal
%        \ex {}[Der Mann] schläft.
%        \ex Er schläft.
\ex 
\gll {}[Der Mann] schläft.\\
	 {}\spacebr{}the man sleeps\\
\glt `The man is sleeping.'
\ex 
\gll Er schläft.\\
	 he sleeps\\
\glt `He is sleeping.'
        \zl
%
\end{description}

}

\frame{
\frametitle{Constituency tests (III)}


\begin{description}
\item[Question formation]
A sequence of words that can be elicited by a question forms a constituent.

\eal
\ex 
\gll {}[Der Mann] arbeitet.\\
	 \spacebr{}the man works\\
\glt `The man is working.'
\ex 
\gll Wer arbeitet?\\
	 who works\\
\glt `Who is working?'
\zl

\end{description}

}

\frame{
\frametitle{Constituency tests (IV)}

\begin{description}
\item[Permutation test] If a sequence of words can be moved without adversely affecting the acceptability of the sentence
in which it occurs, then this is an indication that this word sequence forms a constituent.

\eal
\ex[]{
\gll dass keiner [dieses Kind] kennt\\
     that nobody \spacebr{}this child knows\\
  }
\ex[]{
\gll dass [dieses Kind] keiner kennt\\
	 that this child nobody knows\\
\glt `that nobody knows this child'
  }
\zl

\end{description}

}

\frame{
\frametitle{Constituency tests (V)}

\begin{description}
\item[Fronting]

Fronting is a further variant of the movement test. In German declarative sentences, only a single constituent may normally precede the finite verb:
\eal
\label{bsp-v2}
\ex[]{
\gll [Alle Studenten] lesen während der vorlesungsfreien Zeit Bücher.\\
      \spacebr{}all students read.\textsc{3pl} during the lecture.free time books\\
\glt `All students read books during the semester break.'
}
\ex[]{
\gll [Bücher] lesen alle Studenten während der vorlesungsfreien Zeit.\\
     \spacebr{}books read all students during the lecture.free time\\
}
\ex[*]{
\gll [Alle Studenten] [Bücher] lesen während der vorlesungsfreien Zeit.\\
     \spacebr{}all students \spacebr{}books read during the lecture.free time\\
}
\ex[*]{
\gll [Bücher] [alle Studenten] lesen während der vorlesungsfreien Zeit.\\
     \spacebr{}books \spacebr{}all students read during the lecture.free time\\
}
\zl 
\end{description}

}

\frame{
\frametitle{Constituency tests (VI)}


\begin{description}
\item[Coordination test]
If two sequences of words can be conjoined\is{coordination!test|(} then this suggests that each sequence
forms a constituent.

\ea
\gll {}[Der        Mann] und [die          Frau] arbeiten.\\
     \spacebr{}the man   and \spacebr{}the woman work.3PL\\
\glt `The man and the woman work.'
\z

\end{description}

}






\frame{
\frametitle{Warning}

Danger! 

These tests are not 100\,\% reliable. See \citew[Section~1.3.2]{MuellerGT-Eng} for details.

For more on the tests see also \citew[Section~2]{MuellerEvaluating}.


}



\subsection{Heads}



\frame{
\frametitle{Heads}

A head determines the most important properties of a phrase.
\eal
\ex 
\gll \alert{Träumt} dieser Mann?\\
     dreams this.\NOM{} man\\
\glt `Does this man dream?'
\ex 
\gll \alert{Erwartet} er diesen Mann?\\
	 expects he.\NOM{} this.\ACC{} man\\
\glt `Is he expecting this man?'
\ex 
\gll \alert{Hilft} er diesem Mann?\\
	 helps he.\NOM{} this.\DAT{} man\\
\glt `Is he helping this man?'
\ex 
\gll \alert{in} diesem Haus\\
	 in this.\DAT{} house\\
\ex 
\gll ein \alert{Mann}\\
	 a.\NOM{} man\\
\zl

}

\frame{
\frametitle{Projection}

The combination of a head with other material is called
\alert{projection of the head}.

\pause
A complete projection is a \alert{maximal projection}.

\pause
A maximal projection of a finite verb is a sentence.
}

\frame{
\frametitle{Labeled boxes}


Those of you who moved to a new flat know that is is good to label your boxes.

\pause

\medskip

\centerline{%
\begin{pspicture}(0,0)(7.8,3.4)
     \rput[bl](0,0){%
\psset{fillstyle=solid, framearc=0.25,framesep=5pt}
\psframebox{%
\begin{tabular}{@{}l@{}}
VP\\
\psframebox{%
\begin{tabular}{@{}l@{}}
NP\\[2mm]
       \psframebox{\begin{tabular}{@{}l@{}}
                   Det\\der
                   \end{tabular}}
       \psframebox{\begin{tabular}{@{}l@{}}
                   N\\Mann
                   \end{tabular}}
\end{tabular}}
\psframebox{\begin{tabular}{@{}l@{}}
                   V\\liest
                   \end{tabular}}
\psframebox{%
\begin{tabular}{@{}l@{}}
NP\\[2mm]
           \psframebox{\begin{tabular}{@{}l@{}}
                   Det\\einen
                   \end{tabular}}
           \psframebox{\begin{tabular}{@{}l@{}}
                   N\\Aufsatz
                   \end{tabular}}
\end{tabular}}
\end{tabular}}}
%\psgrid
    \end{pspicture}}

\medskip

The label on a box indicates the most important element in the box.

}

\frame{
\frametitle{Boxes are replaceable}


\begin{itemize}
\item It does not matter what exactly is in the box:
\eal
\ex 
\gll er\\
     he\\
\ex 
\gll der Mann\\
     the man\\
\ex 
\gll der Mann aus Stuttgart\\
     the man  from Stuttgart\\
\ex 
\gll der Mann aus Stuttgart, den wir kennen\\
     the man  from Stuttgart who we know\\
\zl
The only thing that matters:\\
all words or phrases in (\mex{0}) are nominal and complete: NP.

They can be substituted for each other within bigger boxes.

\end{itemize}


}

\frame{
\frametitle{Boxes are replaceable. Well, hm.}


\begin{itemize}
\item This does not work with all NPs:

\eal
\ex[]{ 
\gll Der Mann liest einen Aufsatz.\\
     the man  reads an    essay\\
} 
\ex[*]{ 
\gll Die Männer liest einen Aufsatz.\\
     the men    reads an    essay\\
} 
\ex[*]{ 
\gll Des Mannes liest einen Aufsatz.\\
     the man.\GEN{} reads an essay\\
} 
\zl 

\item Certain properties are important for the distribution of phrases.

\end{itemize}


}


\frame{
\frametitle{More carefully labeled boxes}

~\medskip

\oneline{%
\begin{pspicture}(0,0)(16.4,3.4)
     \rput[bl](0,0){%
\psset{fillstyle=solid, framearc=0.25,framesep=5pt}
\psframebox{%
\begin{tabular}{@{}l@{}}
VP, fin\\[2mm]
\psframebox{%
\begin{tabular}{@{}l@{}}
NP, nom, 3, sg, mas\\[2mm]
       \psframebox{\begin{tabular}{@{}l@{}}
                   Det, nom, sg, mas\\der
                   \end{tabular}}
       \psframebox{\begin{tabular}{@{}l@{}}
                   N, nom, sg, mas\\Mann
                   \end{tabular}}
\end{tabular}}
\psframebox{\begin{tabular}{@{}l@{}}
                   V, fin, 3, sg\\liest
                   \end{tabular}}
\psframebox{%
\begin{tabular}{@{}l@{}}
NP, akk, 3, sg, mas\\[2mm]
           \psframebox{\begin{tabular}{@{}l@{}}
                   Det, akk, sg, mas\\einen
                   \end{tabular}}
           \psframebox{\begin{tabular}{@{}l@{}}
                   N, akk, sg, mas\\Aufsatz
                   \end{tabular}}
\end{tabular}}
\end{tabular}}}
%\psgrid
    \end{pspicture}}

All features that are important for the distribution of the whole phrase are projected.

Such feature are called \alert{head features}.

}


% \frame[shrink=10]{
% \frametitle{Projizierte Merkmale}


% \hfill%
% \begin{tabular}{|l|l|}\hline
% Kategorie    & projizierte Merkmale\\\hline
% Verb         & Kategorie, Verbform ({\it fin\/}, {\it bse\/}, \ldots)\\
% Nomen        & Kategorie, Kasus ({\it nom\/}, {\it gen\/}, {\it dat\/}, {\it acc\/})\\
% %Präposition  & Kategorie, Form der Präposition ({\it an\/}, {\it auf\/}, \ldots)\\
% Adjektiv     & Kategorie, bei flektierten Formen Kasus\\\hline
% \end{tabular}\hfill\hfill\mbox{}
% \pause
% ~
% \bigskip

% Beispiel:
% Wenn \emph{stolzer} den Kasus Genitiv hat,\\
% dann hat auch die gesamte Adjektiv-Phrase Genitiv. 

% \ea
% \emph<3>{einiger \emph<2>{auf ihren Sohn \blau<2>{stolzer}} \blau<3>{Männer}}
% \z

% Das ist wichtig, da die Adjektiv-Phrase mit dem Determinierer und\\
% dem Nomen im Kasus übereinstimmen muß.

% \pause
% Wenn \emph{Männern} in (\mex{1}) Dativ ist, hat die gesamte NP diese Eigenschaft.
% \ea
% den Männern
% \z

% }

%\fi



\subsection{Arguments and adjuncts}

\frame[shrink=10]{
\frametitle{Arguments}

\begin{itemize}
\item Constituents are in different relations with their head.
\pause
\item There are \alert{arguments} and \alert{adjuncts}.
\pause
\item Certain elements are part of the meaning of a verb.

For example in situations described by the verb \emph{love},\\
there is a lover and a \emph{lovee}.

\eal
\ex Kim loves Sandy.
\ex $love'(Kim', Sandy')$
\zl

(\mex{0}b) is a logical representation of (\mex{0}a).

\relation{Kim} and \relation{Sandy} are \alert{logical arguments} of \relation{love}.

\pause
\item Syntactic arguments usually correspond to logical arguments (more on this later).
\pause
\item The term for such relations between head and arguments is \alert{selection} or \alert{valence}.
\pause
\item \citet{Tesniere59a-Eng} transferred the concept of valence from chemistry to linguistics.
\end{itemize}

}


%\BackgroundPicture{periodensystem}{1}{1}
\frame{
\frametitle{Valency in chemistry}

\begin{itemize}
\item Atoms can form more or less stable molecules with other atoms.

\pause
\item The number of electrons on an electron shell is important for the stability of the molecule.

\pause
\item If atoms combine with other atoms this can lead to completely filled electron layers, which
  would result into a stable compound.

\pause
\item The valency of an atom is the number of hydrogen atoms that can be combined with an atom of a
  certain element.

\pause
\item Oxygen has the valency 2 since it can be combined with two hydrogen atoms: H$_2$O.

\pause
\item The elements can be grouped into valence classes.\\
Elements with a certain valence are represented in a column in the periodice system of Mendeleev.

\end{itemize}

}

\frame{
\frametitle{Valence in linguistics}

\begin{itemize}
\item A head needs certain arguments to enter a stable compound.
\item Words having the same valence (same number and type of arguments) are grouped into valence
  classes, since they behave alike with respect to the combinations they enter.


\bigskip

\centerline{
\begin{forest}
[O
  [H] 
  [H] ]
\end{forest}
\hspace{5em}
\begin{forest}
[love
 [Kim]
 [Sandy] ]
\end{forest}
}

\bigskip

Combining oxygen with hydrogen and combining a verb with its arguments

\end{itemize}

\vfill

}

\frame{
\frametitle{Optional arguments}

\begin{itemize}
\item Sometimes arguments may be omitted:
\eal
\ex I am waiting for my man.
\ex I am waiting.
\zl
\pause
The prepositional object of \emph{wait} is an \alert{optional argument}.

\pause
\item All arguments are optional in nominal environments.
\eal
\ex 
\gll Jemand liest diese Bücher.\\
     somebody reads these books\\
\ex 
\gll das Lesen dieser Bücher\\
     the reading of.these books\\
\ex 
\gll das Lesen\\
     the reading\\
\zl

\end{itemize}
}

\frame{
\frametitle{Syntactic arguments that are not logical ones}

\begin{itemize}
\item Syntactic arguments correspond to logical arguments in our example above:
\eal
\ex Kim loves Sandy.
\ex $love'(Kim', Sandy')$
\zl

\pause
\item There are also arguments not contributing semantically:

\eal
\ex 
\gll Es regnet.\\
     it rains\\
\ex 
\gll Kim erholt sich.\\
     Kim recreates \self\\
\zl
\emph{es} and \emph{sich} are \alert{syntactic arguments}, without being \alert{logical arguments}.

\end{itemize}
}

\frame{
\frametitle{Arguments and adjuncts}


\begin{itemize}
\item Adjuncts do not fill a semantic role
\item Adjuncts are optional
\item Adjuncts can be iterated
\end{itemize}


}


\frame{
\frametitle{Adjuncts do not fill a semantic role}

\begin{itemize}
\item In a \emph{loving} situation there is a lover and a lovee.

\emph{since three years} in (\mex{1}) is of a different type:
\ea
Kim loves Sandy since three years.
\z
This phrase provides information about the span in which\\
the relation between Kim and Sandy holds.

\end{itemize}

}

\frame{
\frametitle{Adjuncts are optional}


\begin{itemize}
\item Adjuncts are optional:
\eal
\ex Kim loves Sandy.
\ex Kim loves Sandy since three years.
\ex Kim loves Sandy honestly.
\zl
\pause
\item Be aware! Arguments may also be optional:
\eal
\ex Er gibt den Armen Geld.
\ex Er gibt den Armen.
\pause
\ex Er gibt Geld.
\pause
\ex Er gibt gerne.
\pause
\ex Du gibst. (beim Skat)
\pause
\ex Gib!
\zl
\end{itemize}


}



\frame{
\frametitle{Adjuncts can be iterated}


\begin{itemize}
\item Arguments can be combined with their head once:
\ea[*]{
The man  the man sleeps\\
}
\z

The respective slot of the head (\emph{sleeps}) is filled. 

\pause
\item But adjuncts are different:

\ea
\label{Beispiel-Iteration-Adjektive}
A: All grey squirrels are big.\\
B: No, I saw a small grey squirrel.\\
A: But all small grey squirrels are ill.\\
B: No, I saw a healthy small grey squirrel.\\
\hspaceThis{A:~}\ldots
\z

\end{itemize}

}

\frame{
\frametitle{Some further examples for adjuncts}


Adverbially used adjective (not all adjectives):
\ea
\gll Karl schnarcht \emph{laut}.\\
     Karl snores    loudly\\
\z


Relative clauses (not all of them):
\ea
\gll das Kind, \emph{dem} \emph{der} \emph{Delphin} \emph{hilft}\\
     the child who the dolphin helps\\
\z


Prepositional phrases (not all of them):
\eal
\ex 
\gll Die Frau arbeitet \emph{in} \emph{Berlin}.\\
     the woman works   in Berlin\\
\ex 
\gll die Frau \emph{aus} \emph{Berlin}\\
     the woman from      Berlin\\
\zl



}

%\if 0



% \frame{
% \frametitle{Andere Bezeichnungen}

% \begin{itemize}
% \item Argument: Ergänzung

% \pause
% \item Adjunkt: (freie) Angabe

% \pause
% \item Argumente werden mitunter in Subjekt und Komplemente aufgeteilt.

% \pause
% \item auch Aktant für Subjekte und Objekte\\
%       (aber nicht Prädikative und Adverbialien)
% \pause
% \item Zirkumstant für Adverbialien
%       \begin{itemize}
%       \item Adverbiale des Raumes (Lage, Richtung/Ziel, Herkunft, Weg)
%       \item Adverbiale der Zeit (Zeitpunkt, Anfang, Ende, Dauer)
%       \item Adverbiale des Grundes.\\
%             Hierher werden traditionellerweise auch Adverbialien gestellt,\\
%                    die einen Gegengrund oder eine Bedingung ausdrücken.
%       \item Adverbiale der Art und Weise. 
%       \end{itemize}
% \end{itemize}

% }



\subsection[Grammatical theories]{Various grammatical theories}

\frame{
\frametitle{Various grammatical theories (I)}

\begin{itemize}
\item Dependency Grammar (DG)\\\citep{Tesniere80a-u,Tesniere2015a-u,Kunze75a-u,Weber97a,Heringer96a-u,Eroms2000a}
\item Categorial Grammar (CG)\\\citep{Ajdukiewicz35a-u,Steedman2000a-u}
\item Phrase structure grammar (PSG)\nocite{Bloomfield33a-u}
\item Transformational Grammar and its successors
      \begin{itemize}
      \item Transformational grammar \\\citep{Chomsky57a,Bierwisch63a}
      \item Government \& Binding \\\citep{Chomsky81a,SS88a,Grewendorf88a}
      \item Minimalism \\\citep{Chomsky95a-u,Grewendorf2002a}
      \end{itemize}
\end{itemize}


}
\frame{
\frametitle{Various grammatical theories (II)}

\begin{itemize}
\item Tree Adjoning Grammar\\
      \citep*{JLT75a-u,Joshi87a-u,KJ85a}
\item Generalized Phrase Structure Grammar (GPSG)\\\citep*{GKPS85a,Uszkoreit87a}
\item Lexical Functional Grammar (LFG)\\\citep{Bresnan82a-ed,Bresnan2001a,BF96a-ed,Berman2003a}
\item Head-Driven Phrase Structure Grammar (HPSG)\\\citep{ps,ps2,Mueller99a,Mueller2002b,MuellerLehrbuch}
\item Construction Grammar (CxG)\\\citep*{FKoC88a,Goldberg95a,Goldberg2006a,FS2006a-ed}

\bigskip
\item We will deal with most of these in this course.
%See \citew{MuellerGT-Eng} for an overview.
\end{itemize}


}







%      <!-- Local IspellDict: en_US-w_accents --> 