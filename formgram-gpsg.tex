%% -*- coding:utf-8 -*-




\section{Generalized Phrase Structure Grammar (GPSG)}

\subtitle{Generalized Phrase Structure Grammar (GPSG)}

\huberlintitlepage[22pt]


\outline{

\begin{itemize}
\item Introduction and basic terms
\item Phrase structure grammar and \xbar Theory
\item Government \& Binding (GB)
\item \alert{Generalized Phrase Structure Grammar (GPSG)}
\item {Feature descriptions, feature structures and models}
\item Lexical Functional Grammar (LFG)
%\item PATR
\item Categorial Grammar (CG)
\item Head-Driven Phrase Structure Grammar (HPSG)
%\item Konstruktionsgrammatik (CxG)
\item Tree Adjoning Grammar (TAG)
\end{itemize}

%\tableofcontents
}

\frame{
\frametitle{Reading material}

\citew[Chapter~5]{MuellerGT-Eng} without Section~5.1.4 about semantics.


}

\frame{
\frametitle{Generalized Phrase Structure Grammar (GPSG)}

\begin{itemize}
\item GPSG was developed as an answer
to Transformational Grammar at the end of the 1970s.
\pause
\item Main publication: \citet*{GKPS85a}

\pause
\item \citet{Uszkoreit87a} developed large GPSG fragment of German.
\pause
\item Chomsky showed PSGs to be inadequate.\\
      GPSG extends PSG in ways that make it possible to address Chomsky's monita:
\begin{itemize}
\item categories may be complex \citep{Harman63a}
\pause
\item different treatment of local reordering
\pause
\item passive as metarule
\pause
\item non-local dependencies as a series of local dependencies
\end{itemize}
\pause
\item We will deal with each of these innovations in what follows.
\end{itemize}


}

\subsection{General remarks on the representational format}

\subsubsection{Categories and \xbar Theory}

\frame[shrink=0]{
\frametitle{General remarks on the representational format}

\begin{itemize}
\item Categories are sets of feature value pairs.
\pause
\item Lexical entries have a feature \subcat. The value is a number
which says something about the kind of grammatical rules in which the word can be used. 
\pause
\item Examples from \citew{Uszkoreit87a}:

\medskip
\begin{tabular}{@{}l@{~$\to$~}ll@{}}
V2  & H[5]                              & (kommen `come', schlafen `sleep')\\
V2  & H[6], N2[Case Acc]                & (kennen `know', suchen `search')\\
V2  & H[7], N2[Case Dat]                & (helfen `help', vertrauen `trust')\\
V2  & H[8], N2[Case Dat], N2[Case Acc]  & (geben `give', zeigen `show')\\
V2  & H[9], V3[+dass]                   & (wissen `know', glauben `believe')\\
\end{tabular}
\medskip

These rules license VPs: the combination verb \& complements, but not subject.

\pause
\item The numbers following the category symbols (V or N) indicate the
\xbar~level.\\
      The maximum level of a verbal projection is three rather than two.
\pause
\item H stands for Head.

\end{itemize}

}

\subsubsection{Principles: The Head Feature Convention}

\frame{
\frametitle{Principles: The Head Feature Convention}

Head Feature Convention:\\
The mother node and the head daughter must bear the same head features unless indicated otherwise.

}

\subsubsection{Metarules and ID/LP format}

\frame{
\frametitle{Metarules and ID/LP format}

Two further innovations of GPSG:
\begin{itemize}
\item Metarules: Additional phrase structure rules are licensed via metarules.
\item ID/LP format: Constraints on linearization are separated from immediate dominance.
\end{itemize}

These two tools will be discussed with respect to our set of phenomena.


}




\subsection{Local reordering \& Verb position}

\frame{
\frametitle{Local reordering}


\begin{itemize}
\item Arguments can appear in almost any order in the German \mf.
\eal
\ex 
\gll {}[weil] \gruen{der} \gruen{Mann} \rot{dem} \rot{Kind} \blau{das} \blau{Buch} gibt\\
     \spacebr{}because the.\NOM{} man the.\DAT{} child the.\ACC{} book gives\\
\glt `because the man gives the book to the child'
\ex 
\gll {}[weil] \gruen{der} \gruen{Mann} \blau{das} \blau{Buch} \rot{dem} \rot{Kind} gibt\\
     \spacebr{}because the.\NOM{} man the.\ACC{} book the.\DAT{} child gives\\
\ex\label{ex-das-buch-der-mann-der-frau-gibt} 
\gll {}[weil] \blau{das} \blau{Buch} \gruen{der} \gruen{Mann} \rot{dem} \rot{Kind} gibt\\
     \spacebr{}because the.\ACC{} book the.\NOM{} man the.\DAT{} child gives\\
\ex 
\gll {}[weil] \blau{das} \blau{Buch} \rot{dem} \rot{Kind} \gruen{der} \gruen{Mann} gibt\\
     \spacebr{}because the.\ACC{} book the.\DAT{} child the.\NOM{} man gives\\
\ex 
\gll {}[weil] \rot{dem} \rot{Kind} \gruen{der} \gruen{Mann} \blau{das} \blau{Buch} gibt\\
     \spacebr{}because the.\DAT{} child the.\NOM{} man the.\ACC{} book gives\\
\ex 
\gll {}[weil] \rot{dem} \rot{Kind} \blau{das} \blau{Buch} \gruen{der} \gruen{Mann} gibt\\
     \spacebr{}because the.\DAT{} child the.\ACC{} book the.\NOM{} man gives\\
\zl
\end{itemize}
}



\frame{
\frametitle{Motivation for linearization rules (I)}

%\smallframe
\savespace
Motivation: Permutation with phrase structure rules $\to$\\
we need six phrase structure rules for ditransitive verbs in verb-final position:
\ea
\begin{tabular}[t]{@{}l@{ }l@{ }l@{ }l@{ }l@{ }}
S  & $\to$ NP[nom]& NP[dat] & NP[acc] & V\\
S  & $\to$ NP[nom]& NP[acc] & NP[dat] & V\\
S  & $\to$ NP[acc]& NP[nom] & NP[dat] & V\\
S  & $\to$ NP[acc]& NP[dat] & NP[nom] & V\\
S  & $\to$ NP[dat]& NP[nom] & NP[acc] & V\\
S  & $\to$ NP[dat]& NP[acc] & NP[nom] & V\\
\end{tabular}
\z
}

\frame{
\frametitle{Motivation for linearization rules (II)}


Plus six rules for verb-initial position:

\ea
\begin{tabular}[t]{@{}l@{ }l@{ }l@{ }l@{ }l}
S  & $\to$ V NP[nom]& NP[dat] & NP[acc]\\
S  & $\to$ V NP[nom]& NP[acc] & NP[dat]\\
S  & $\to$ V NP[acc]& NP[nom] & NP[dat]\\
S  & $\to$ V NP[acc]& NP[dat] & NP[nom]\\
S  & $\to$ V NP[dat]& NP[nom] & NP[acc]\\
S  & $\to$ V NP[dat]& NP[acc] & NP[nom]\\
\end{tabular}
\z

A generalization is missed.

Similarly for transitive verbs and other valence frames.

}

\frame{
\frametitle{Abstraction from linear order: Dominance}

\begin{itemize}
\item \citet*{GKPS85a}:\\
      Separation of \alert{immediate dominance} = ID and \alert{linear precedence} = LP.
\pause
\item Dominance rules do not constrain the order of the daughters.
\ea
\begin{tabular}[t]{@{}l@{ }l}
S  & $\to$ V, NP[nom], NP[acc], NP[dat]\\
\end{tabular}
\z

The only thing (\mex{0}) says is that S dominates the other nodes:
\medskip
\centerline{%
\begin{forest}
sm edges
[S
  [V]
  [{NP[nom]}]
  [{NP[dat]}]
  [{NP[acc]}] ]
\end{forest}}
\medskip

\pause
\item Since there are no constraints on the order of the elments of the right-hand side, we need one
  rule rather than twelve:

\end{itemize}
}


\frame{
\frametitle{Abstraction from linear order: Linear order}


\begin{itemize}
\item LP rules hold for local trees, that is, trees of depth one:


\begin{table}[H]
\centerline{%
\begin{forest}
sm edges
[S
  [V]
  [{NP[nom]}]
  [{NP[dat]}]
  [{NP[acc]}] ]
\end{forest}}
\end{table}

~\medskip

      $\to$ We can say something about order of V, NP[nom], NP[dat] and NP[acc].

\pause

An LP constraint holds for the whole grammar.\\
If we claim that NP[nom] precedes NP[acc],\\
this holds for rules for strictly transitive verbs as well as for rules for ditransitive verbs.


\end{itemize}
}


\frame{
\frametitle{Getting more restrictive again}


\begin{itemize}
\item Without restriction for the order $\to$ too much freedom

\medskip
\begin{tabular}[t]{@{}l@{ }l}
S  & $\to$ V, NP[nom], NP[dat], NP[acc]\\
\end{tabular}

\medskip
The rule admits the following order:
\ea[*]{
\gll \rot{Dem} \rot{Kind} \gruen{der} \gruen{Mann} gibt ein Buch.\\
     the.\DAT{} child the.\NOM{} man gives the.\ACC{} book\\
}
\z
\pause
\item Linearization rules rule out such orders.

\ea
\begin{tabular}[t]{@{}l@{~$<$~}l@{}}
V[+MC]  & X\\
X       & V[$-$MC]\\
\end{tabular}
\z
\textsc{mc} stand for \emph{main clause}. 

LP rule states: verb must be placed before all other constituents in main clauses
(+\textsc{mc}) and after all other constituents in dependent clauses ($-$\textsc{mc}).

\end{itemize}
}











\subsection{Passive}

\subsubsection{Passive pre-theoretically}

\frame[shrink=15]{
\frametitle{Passive pre-theoretically (I)}
\savespace\smallexamples

German passive theory-neutrally:
\begin{itemize}
\item The subject is suppressed. 
\item If there is an accusative object, this becomes the subject.
\end{itemize}

This holds for all verb classes forming a passive. Independent of the arity of the verb:
\eal
\label{beispiel-arbeiten}
\ex 
\gll weil er noch gearbeitet hat\\
	 because he.\nom{} still worked has\\
\glt 'because he has still worked'
\ex 
\gll weil noch gearbeitet wurde\\
	 because still worked was\\
\glt `because there was still working there'
\zl
\pause
\eal
\label{beispiel-denken}
\ex 
\gll weil er an Maria gedacht hat\\
	 because he.\nom{} on Maria thought has\\
\glt `because he thought of Maria'
\ex 
\gll weil an Maria gedacht wurde\\
	 because on Maria thought was\\
\glt `because Maria was thought of'
\zl

}


\frame[shrink=15]{
\frametitle{Passive pre-theoretically (II)}
\savespace\smallexamples

German passive theory-neutrally:
\begin{itemize}
\item The subject is suppressed. 
\item If there is an accusative object, this becomes the subject.
\end{itemize}

\eal
\ex 
\gll weil Judit den Weltmeister geschlagen hat\\
     because Judit.\nom{} the.\acc{} world.champion beaten has\\
\glt `because Judit has beaten the world champion'
\ex 
\gll weil der Weltmeister geschlagen wurde\\
	 because the.\nom{} world.champion beaten was\\
\glt `because the world champion was beaten'
\zl
\pause
\eal
\ex 
\gll weil er ihm den Aufsatz gegeben hat\\
     because he.\nom{} him.\dat{} the.\acc{} essay given has\\
\glt `because he has given him the essay'
\ex 
\gll weil ihm der Aufsatz gegeben wurde\\
     because him.\dat{} the.\nom{} essay given was\\
\glt `because he was given the essay'
\zl

}



\subsubsection{Metarules}

\frame{
\frametitle{Passive and phrase structure grammars}

\begin{itemize}
\item One would have to write down two rules for every active/passive pair in PSG.
\pause
\item GPSG is a non-transformational theory.
\pause
\item Metarule derives passive rules from active rules.
\pause
\medskip
\item These are explained with respect to the subject introduction metarule.
\end{itemize}


}

\frame[shrink=5]{
\frametitle{Introduction of the subject via a metarule (I)}

Our rules look like this:

\ea
\begin{tabular}[t]{@{}l@{~$\to$~}ll@{}}
V2  & H[7], N2[Case Dat]                & (helfen `help', vertrauen `trust')\\
V2  & H[8], N2[Case Dat], N2[Case Acc]  & (geben `give', zeigen `show')\\
\end{tabular}
\z

\pause

The rules in (\mex{0}) can be used to analyze VPs but not sentences with subject.

\pause

We use a metarule saying: ``If there is a rule of the form `V2 consists of something', then there is
also a rule stating `V3 consists of whatever V2 consists of + an NP in the nominative'{}''.  

\pause

Formally:
\ea
V2  $\to$ W $\mapsto$\\
V3  $\to$ W, N2[Case Nom]
\z

W stands for an arbitrary number of categories (whatever).

}

\frame{
\frametitle{Introduction of the subject via a metarule (II)}



\ea
V2  $\to$ W $\mapsto$\\
V3  $\to$ W, N2[Case Nom]
\z



\pause

This metarule takes the rules in (\mex{1}) as input and produces the rules in (\mex{2}):
\ea
\oneline{%
\begin{tabular}[t]{@{}l@{~$\to$~}ll@{}}
V2  & H[7], N2[Case Dat]                & (helfen `help', vertrauen `trust')\\
V2  & H[8], N2[Case Dat], N2[Case Acc]  & (geben `give', zeigen `show')\\
\end{tabular}
}
\z

\ea
\begin{tabular}[t]{@{}l@{~$\to$~}l@{}}
V3  & H[7], N2[Case Dat], N2[Case Nom]                \\
V3  & H[8], N2[Case Dat], N2[Case Acc], N2[Case Nom]  \\
\end{tabular}
\z

\pause

Subject and other arguments are on the same right-hand side of a rule and hence can be permuted,
provided no LP rule is violated.

}

\subsubsection{Passive as metarule}



\frame[shrink=5]{
\frametitle{Passive as metarule}

\begin{itemize}
\item For each active rule with subject and accusative object, a passive rule will be licensed with
  the subject suppressed. The relation between the rules is captured.
\pause
\item Differences between Transformational Grammar/GB and GPSG:\\
      It is not the case that there are several trees that are related to each other,\\
      but rather active rules are related to passive rules.

      The active and passive rules can be used to derive two structures independently:\\
      (\mex{1}b) is not derived from (\mex{1}a).

\eal
\ex 
\gll weil Judit den Weltmeister geschlagen hat\\
     because Judit.\nom{} the.\acc{} world.champion beaten has\\
\glt `because Judit has beaten the world champion'
\ex 
\gll weil der Weltmeister geschlagen wurde\\
     because the.\nom{} world.champion beaten was\\
\glt `because the world champion was beaten'
\zl

The generalization regarding active/passive alternations is captured nevertheless.
\end{itemize}

}

\frame{
\frametitle{Passive in English}

\citet*{GKPS85a} suggest the following metarule:
\ea
VP  $\to$ W, NP $\mapsto$\\
VP[PAS]  $\to$ W, (PP[\emph{by}])
\z
This rule says that verbs selecting an object can be realized without this object in a passive VP.
Optionally a \emph{by} PP may appear in passive VPs.

(VP corresponds to V2)

}

\frame{
\frametitle{Problems of the passive metarule operating on VP}

\begin{enumerate}
\item Rule does not refer to the type of the verb (not all verbs have a passive).
\pause
% \item Es ist unklar, wie die Semantik parallel zur Syntax aufgebaut werden soll:
%       Die Regel in (\mex{0}) unterdrückt ein NP-Argument in der VP. 
%       Dieses Argument ist aber ein Objekt. Rein syntaktisch stellt das System der GPSG-Metaregeln
%       die richtige Menge von ID-Regeln zur Verfügung, in der Semantik muss man aber sicherstellen,
%       dass das durch eine Metaregel eingeführte Subjekt mit dem unterdrückten Objekt übereinstimmt.
% \pause
\item Impersonal passive cannot be derived by suppressing an object.
\ea
\begin{tabular}[t]{@{}l@{\hspace{2cm}}l@{}}
V2 $\to$ H[5] & (\emph{arbeiten} `work')\\
V2 $\to$ H[13], PP[\emph{an}] & (\emph{denken} `think')\\
\end{tabular}
\z      

So, if the analysis of the passive in English is not revised,\\
the analyses of the passive in English and German will differ.

\pause
\item The German passive metarule could apply to rules including the subject.

\end{enumerate}


}


\subsection{Long"=distance dependencies}


\frame[shrink=5]{
\frametitlefit{Long"=distance dependencies as the result of local dependencies}

\begin{itemize}
\item Until now: verb-initial and verb-final placement of the verb:
\eal
\ex 
\gll {}[dass] der Mann dem Kind das Buch \alert{gibt}\\
	 {}\spacebr{}that the.\NOM{} man the.\DAT{} child the.\ACC{} book gives\\
%\glt `that the man gives the book to the child'
\ex 
\gll \alert{Gibt} der Mann dem Kind das Buch?\\
	 gives the.\NOM{} man the.\DAT{} child the.\ACC{} book\\
%\glt `Does the man give the book to the child?'
\zl
\pause
\item What about verb second placement:
\eal
\ex 
\gll Der Mann \alert{gibt} dem Kind das Buch.\\
     the.\NOM{} man  gives the.\DAT{} child the.\ACC{} book\\
%\glt `The man gives the child the book.'
\ex 
\gll Dem Kind \alert{gibt} der Mann das Buch.\\
     the.\DAT{} child gives the.\NOM{} man the.\ACC{} book\\
%\glt `The man gives the child the book.'
\zl

\pause
\item V2 is analyzed as a nonlocal dependency via a sequence of local dependencies.

One of the main innovations of GPSG:\\
transformationless analysis of nonlocal dependencies (but also \citew{Harman63a}).
\end{itemize}

}

\subsubsection{Metarules for the introduction of nonlocal dependencies}


\frame{
\frametitle{Metarules for the introduction of nonlocal dependencies}

We take an arbitrary category X out of the set of categories on the right-hand side of the rule and
represent it on the left-hand side after a slash (`/'):
\ea
V3  $\to$ W, \gruen{X} $\mapsto$\\
V3/\gruen{X}  $\to$ W
\z
\pause

Given the input in (\mex{1}), the rule creates the rules in (\mex{2}):
\ea
\begin{tabular}[t]{@{}l@{~$\to$~}l@{}}
V3  & H[8], N2[Case Dat], N2[Case Acc], N2[Case Nom] 
\end{tabular}
\z
\ea
\begin{tabular}[t]{@{}l@{~$\to$~}l@{}}
V3/N2[Case Nom] &  H[8], N2[Case Dat], N2[Case Acc]\\
V3/N2[Case Dat] &  H[8], N2[Case Acc], N2[Case Nom]\\
V3/N2[Case Acc] &  H[8], N2[Case Dat], N2[Case Nom]\\
\end{tabular}
\z

}

\subsubsection{Rule for binding off nonlocal dependencies}

\frame{
\frametitle{Rule for binding off nonlocal dependencies}


\ea
V3[+Fin] $\to$ X[+Top], V3[+MC]/X
\z
X stands for arbitrary category marked as missing in V3 by `/'.

\pause
Example instantiations of the rule are given in (\mex{1}):
\ea
\begin{tabular}[t]{@{}l@{~$\to$~}l@{~}l@{}}
V3[+Fin] & N2[+Top, Case Nom], & V3[+MC]/N2[Case Nom]\\
V3[+Fin] & N2[+Top, Case Dat], & V3[+MC]/N2[Case Dat]\\
V3[+Fin] & N2[+Top, Case Acc], & V3[+MC]/N2[Case Acc]\\
\end{tabular}
\z

\pause
LP rule: X in (\mex{-1}) is serialized left of anything else
(\eg V3), since it is [+Top].
\ea
{}[+Top] $<$ X
\z


}


\subsubsection{An example analysis}

\frame{
\frametitle{An example analysis}


\centerline{%
\scalebox{0.85}{
\begin{forest}
sm edges
% [{V3[+\textsc{fin}, $+$\textsc{mc}]}
%   [\alert<beamer:4>{N2[dat,+\textsc{top}]} [dem Kind;the child,roof] ]
%   [{V3[+\textsc{mc}]/\alert<beamer:4>{N2[dat]}}, alert on=<beamer:2>
%     [\alert<3>{V[8,+\textsc{mc}]},alert on=<beamer:3> [gibt;gives] ]
%     [{N2[nom]} [er;he] ] 
%     [{N2[acc]} [das Buch;the book, roof] ] ] ]
[{V3[+\textsc{fin}, $+$\textsc{mc}]},s sep+=1em
  [{\gruen<4>{N2[dat},+\textsc{top}\gruen<4>{]}} [dem Kind;the child,roof] ]
  [\gruen<2>{V3[+\textsc{mc}]/\gruen<1,4>{N2[dat]}}
    [\gruen<2>{\gruen<3>{V[}8,\gruen<3>{+\textsc{mc}]}} [{gibt};gives] ]
    [\gruen<2>{N2[nom]} [{er};he] ] 
    [\gruen<2>{N2[acc]} [{das Buch};the book, roof] ] ] ]
\end{forest}
}}

\begin{itemize}
\item Metarule licenses rule introducing dative object into \slasch.
\pause
\item This rule is applied and licenses the subtree for \emph{gibt er das Buch}.
\pause
\item The linearization rule orders the verb left of other constituents (V[+MC] $<$ X).
\pause
\item The constituent following the slash is bound off in the last step.
\end{itemize}

}

\subsubsection{An example with nonlocal dependencies}

\frame{%[shrink=20]{
\frametitle{An example with nonlocal dependencies (I)}

%\savespace
All NPs in (\mex{1}) depend on the same verb:
\ea
\gll Dem Kind gibt er das Buch.\\
     the.\dat{} child gives he.\nom{} the.\acc{} book\\
\glt `He gives the child the book.'
\z
Complicated system of linearization rules $\to$ analyze (\mex{0}) with a flat structure.
\pause

But this would not work for:


\ea \gll Wen$_i$ glaubst du, daß ich \_$_i$ gesehen habe?\footnotemark\\
     who believe you that I {} seen have\\
\footnotetext{%
    \citew[\page84]{Scherpenisse86a}.
    }
\glt `Who do you think I saw?'
%\ex
% {\raggedright
% \gll {}[Gegen ihn]$_i$ falle es den Republikanern hingegen schwerer, [~[~Angriffe~\_$_i$] zu lancieren].\footnotemark\\
% 	 {}\spacebr{}against him fall it the Republicans however more.difficult
%          \hspaceThis{[~[~}attacks to launch\\
% \par}
% \footnotetext{%
%   taz, 08.02.2008, p.\,9.
% }
% \glt `It is, however, more difficult for the Republicans to launch attacks against him.'
\z

(\mex{0}) cannot be explained by local reordering since \emph{wen} does not depend on \emph{glaubst}
but on \emph{gesehen} and \emph{gesehen} is located in a different local subtree.

}

\frame{
\frametitle{An example with nonlocal dependencies (II)}


\begin{itemize}
\item (\mex{1}) is analyzed in several steps: introduction, percolation and finally binding off of information about the long"=distance dependency
\ea
\gll Wen glaubst du, daß ich gesehen habe?\\
     who believe you that I seen have\\
\z

\pause

\item \emph{ich gesehen habe} is V3/NP[acc]\\
(grammar rule licensed by a metarule)

\pause
\item \emph{dass ich gesehen habe} is V3/NP[acc]\\
      (percolation of \slasch information)

\pause
\item \emph{glaubst du, dass ich gesehen habe} is V3/NP[acc]\\
      (percolation of \slasch information)

\pause
\item \emph{Wen glaubst du, dass ich gesehen habe} is V3\\
      (binding off of \slasch information in grammar rule)

\end{itemize}

}

\frame{
\frametitle{An example with nonlocal dependencies (III)}


\vfill
\centerline{%
\scalebox{0.75}{%
\begin{forest}
sm edges,empty nodes
[{V3[+\textsc{fin},+\textsc{mc}]}
  [{\gruen<3>{N2[acc},+\textsc{top}\gruen<3>{]}} [wen;who] ]
  [{V3[+\textsc{mc}]/\gruen<2-3>{N2[acc]}}
    [{V[9,+\textsc{mc}]} [glaubst;believes] ]
    [{N2[nom]} [du;you] ] 
    [{V3[+dass,$-$\textsc{mc}]/\gruen<2>{N2[acc]}} 
      [{}[dass;that] ]
      [{V3[$-$dass,$-$\textsc{mc}]/\gruen<1>{N2[acc]}} 
         [{N2[nom]} [ich;I] ]
         [{V[6,$-$\textsc{mc}]} [gesehen habe;seen have,roof] ] ] ] ] ]
\end{forest}
}
}

\vfill
Simplifying assumption: \emph{gesehen habe} behaves like a simplex transitive verb.
\vfill

}

\subsection{Summary and Classification}

\subsubsection{Highlights}

\frame{
\frametitle{Highlights: Across the Board Extraction}

\begin{itemize}
\item Gazdar's \citeyearpar{Gazdar81a} \slasch-based analysis can account for so-called Across the Board extraction \citep{Ross67a}:

\eal\settowidth\jamwidth{(= S/NP \& S/NP)}
\label{ex-atb-gazdar}
\ex[]{ The kennel     which Mary made and Fido sleeps in has been stolen.	 \jambox{(= S/NP \& S/NP)}
}
\ex[]{ The kennel in which Mary keeps drugs and Fido sleeps has been stolen.	\jambox{(= S/PP \& S/PP)}
}
\ex[*]{The kennel (in) which Mary made and Fido sleeps has been stolen.     \jambox{(= S/NP \& S/PP)}
}
\zl

Conjuncts have to have the same element in \slasch and this information is percolated further and
then bound off.

\pause

\item Such sentences are a miracle for transformational analyses:\\
      Why must two transformations move something of the same category?\\
      How can two different things land in the same position?

\end{itemize}

}


\subsubsection{Problems}

\frame{
\frametitle{Problems}

\begin{itemize}
\item representation of valence and morphology
\item partial fronting
\item generative capacity
\end{itemize}


}


\subsubsubsection{Representation of valence and morphology}

\frame{
\frametitle{Representation of valence and morphology}

\begin{itemize}
\item Morphology has to access valence information:
\eal\settowidth\jamwidth{(nominative, accusative, PP[mit])}
\ex[]{
\gll lös-bar\\
     solv-able\\  	\jambox{(nominative, accusative)}
}
\ex[]{
\gll vergleich-bar\\ 
     compar-able\\ \jambox{(nominative, accusative, PP[mit])}
}
\ex[*]{
\gll schlaf-bar\\ 
	 sleep-able\\ \jambox{(nominative)}
}
\ex[*]{
\gll helf-bar\\  
	 help-able\\\jambox{(nominative, dative)}
}
\zl
\item Generalization: \emph{bar} adjectives can be formed from verbs governing an accusative.
\pause
\item This information is inaccessable in GPSG. Only valence numbers and this number does
  not even tell us whether there is an accusative. There may be a bunch of different rules
  (active/passive) with or without the accusative.
\pause
\item Valence must contain detailed descriptions of arguments (CG, LFG, HPSG).

\end{itemize}


}

\subsubsubsection{Partial fronting (I)}

\frame{
\frametitle{Partial fronting}

German allows the fronting of (partial) VPs:
\eal
\ex 
\gll [Erzählen] wird \gruen{er}        \rot{seiner}     \rot{Tochter}  \blau{ein}      \blau{Märchen} können.\\
     \spacebr{}tell     will he.\NOM{} his.\DAT{} daughter a.\ACC{} fairy.tale can\\
\glt `He will be able to tell his daughter a fairy tale.'
\ex 
\gll [\blau{Ein} \blau{Märchen} erzählen] wird \gruen{er} \rot{seiner} \rot{Tochter} können.\\
     \spacebr{}a.\ACC{} fairy.tale tell will he.\NOM{} his.\ACC{} daughter can\\
\ex 
\gll [\rot{Seiner} \rot{Tochter} \blau{ein} \blau{Märchen} erzählen] wird \gruen{er} können.\\
     \spacebr{}his.\DAT{} daughter a.\ACC{} fairy.tale tell will he.\NOM{} can\\
\zl

Arguments not realized in the fronted VP have to be realized in the \mf.

}

\frame[shrink=7]{
\frametitle{Partial fronting (II)}

\begin{itemize}
\item Arguments missing in initial position have to be realized in the \mf.\\
      The case in the \mf has to match the requirement of the verb in the \vf:
\eal
\ex[]{
\gll Verschlungen hat er es nicht.\\
     devoured     has he.\nom{} it.\acc{} not\\
\glt `He did not devour it.'
}
\ex[*]{
\gll Verschlungen hat er nicht.\\
     devoured     has he.\nom{} not\\
}
\ex[*]{
\gll Verschlungen hat er ihm nicht.\\
     devoured     has he.\nom{} him.\dat{} not\\
}
\zl
\pause
\item But this is impossible to do with the standard treatment of valence in GPSG. 
\pause
\item Combinations of verbs with arguments are licensed by PSG rules referring to numbers.

\item But the objects can only be missing when they are realized in the \mf.\\
      How is this connection established?

\pause
\item \citet{Nerbonne86a} and \citet{Johnson86a}: different representation of valence.\\
One similar to Categorial Grammar.

\end{itemize}


}


\frame{
\frametitle{Generative capacity}

\begin{itemize}
\item The generative capacity of GPSG corresponds to those of context free grammars.
\pause
\item Being restrictive was one of the goals of GPSG.
\pause
\item But \citet{Shieber85a} and \citet{Culy85a}:\\
      there are languages that cannot be described with context free grammars.\\
 (see also \citew{Pullum86a} for historical remarks)
\pause
\item This means that GPSG is not powerful enough to describe all languages.
\pause

\bigskip
\item All mentioned problems are fixed in HPSG, the successor of GPSG.
\end{itemize}


}





%      <!-- Local IspellDict: en_US-w_accents -->
