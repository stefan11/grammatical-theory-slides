\subsection{Passive}


\subtitle{Government \& Binding: Passive and local reordering}

\huberlintitlepage[22pt]


\frame{
\frametitle{Reading material}

\citew[Section~3.4--3.5]{MuellerGT-Eng}


}


\subsubsection{Case of arguments: Structural and lexical case}
\label{sec-struk-lex-kas}
\label{sec-struc-lex-kas}

\frame{
\frametitle{Case and case principles}

\begin{itemize}
\item What types of case exist?
\pause
\item In which way does case depend on syntactic context?
\pause
\item One way to capture case requirements is to list them in valence representations.\\
      If we understand the regularities, we can avoid this.

We capture regularities and
need just one lexical item for verbs like \emph{lesen} `read':
\eal
\ex 
\gll \alert{Er} möchte das Buch lesen.\\
     he.\NOM{}        wants  the book read\\
\ex 
\gll Ich sah \alert{ihn} das Buch lesen.\\
     I   saw him.\ACC{}        the book read\\
\zl
The case of the subject (and the object) is determined by the principle.
\end{itemize}

}



\subsubsubsection{Structural case}

\frame{
\frametitle{Structural case: The subject}

\begin{itemize}
\item If case depends on the syntactic environment, it is called \alert{structural case}.\\
      Otherwise it is \alert{lexical case}.
\pause
\item Subject (nominative in the active) can be realized as accusative and genitive:
\eal
\ex 
\gll \alert{Der} \alert{Installateur} kommt.\\
	 the.\NOM{} plumber comes\\
\glt `The plumber is coming.'
\pause
\ex 
\gll Der Mann lässt \alert{den} \alert{Installateur} kommen.\\
	 the man lets the.\ACC{} plumber come\\
\glt `The man is getting the plumber to come.'
\pause
\ex 
\gll das Kommen \alert{des} \alert{Installateurs}\\
	 the coming of.the plumber\\
\glt `the plumber's visit'
\zl
\end{itemize}

}

\frame{
\frametitle{Structural: The object}

\begin{itemize}
\item Object (accusative in the active) can be realized as nominative and genitive:

\eal
\ex 
\gll Judit schlägt \alert{den} \alert{Weltmeister}.\\
     Judit beats the.\ACC{} world.champion\\
\glt `Judit beats the world champion.'
\ex 
\gll \alert{Der} \alert{Weltmeister} wird geschlagen.\\
     the.\NOM{} world.champion is beaten\\
\glt `The world champion is being beaten.'
\ex 
\gll  das Schlagen \alert{des} \alert{Weltmeisters}\\
      the beating  of.the world.champion\\
\zl
\end{itemize}
}

\subsubsubsection{Lexical case}

%\subsubsubsection{Genitiv}

\frame{
\frametitle{Lexical case}

\begin{itemize}
\item genitive depending on the verb is lexical case:\\
The case of the genitive object does not change in passivization.
\eal
\ex[]{
\gll Wir gedenken \alert{der} \alert{Opfer}.\\
     we remember the.\GEN{} victims\\
}
\ex[]{ 
\gll \alert{Der} \alert{Opfer} wird gedacht.\\
	 the.\GEN{} victims are remembered\\
\glt `The victims are being remembered.'
}
\ex[*]{
\gll \alert{Die} \alert{Opfer} wird / werden gedacht.\\
     the.\NOM{}  victims       is   {} are      remembered\\    
}
\zl
\pause
(\mex{0}b) = impersonal passive, there is no subject.
\end{itemize}

}

%\subsubsubsection{Dativ}

\frame{
\frametitle{Is the dative a lexical case?}

\begin{itemize}
\item Similarly there is no change in the passive with dative objects:
\eal
\ex 
\gll Der Mann hat \alert{ihm} geholfen.\\
	 the man has him.\DAT{} helped\\
\glt `The man has helped him.'
\ex 
\gll \alert{Ihm} wird geholfen.\\
	 him.\DAT{} is helped\\
\glt `He is being helped.'
\zl
\pause
\item But what about (\mex{1})?
\eal
\ex 
\gll Der Mann  hat   den Ball \alert{dem} \alert{Jungen} geschenkt.\\
     the man   has   the ball the.\DAT{} boy given\\
\ex 
\gll \alert{Der} \alert{Junge} bekam den Ball            geschenkt.\\
     the.\NOM{} boy  got the ball given\\
\zl
\end{itemize}

}

\frame{
\frametitle{Dative structural or lexical?}

\begin{itemize}
\item The status of the dative is controversial.\\
Three options:
\begin{enumerate}[<+->]
\item All datives are lexical.
\item Some datives are lexical, some structural.
\item All datives are structural.
\end{enumerate}
\end{itemize}

}

\frame[shrink=5]{
\frametitle{1. The dative as lexical case}

\begin{itemize}
\item If the dative is treated as a lexical case,\\
      the dative has to change in the dative passive from lexical to structural.
\pause
\item Haider's examples in (\mex{1}) are immediately explained \citeyearpar[\page 20]{Haider86}:
\begin{exe}
\ex \begin{tabular}[t]{@{}l@{~}l@{\hspace{1em}}l@{~}l@{~}l@{}}
a. & {\gll Er  streichelt \alert{den} \alert{Hund}.\\
          he  strokes the dog\\
     } & \visible<3->{d.} & & \visible<3->{\gll Er hilft \alert{den} \alert{Kindern}.\\
                    he helps the.\DAT{} children\\
               }\\
b. & {\gll \alert{Der} \alert{Hund} wurde gestreichelt.\\
           the         dog          was   stroked\\
     } & \visible<3->{e.} & & \visible<3->{\gll \alert{Den} \alert{Kindern} wurde geholfen.\\
                    the.\DAT{}         children was helped\\
              }\\
c. & {\gll sein Streicheln \alert{des}    \alert{Hundes}\\
           his stroking    of.the dog\\
     } & \visible<3->{f.} &  & \visible<3->{\gll das Helfen \alert{der} \alert{Kinder}\\
                    the helping of.the children\\}\\
   &  &     &  & \visible<3->{(children agent only)}
              \\
   &  &  \visible<3->{g.} & \visible<3->{*} & \visible<3->{\gll sein Helfen \alert{der} \alert{Kinder}\\
                      his  helping of.the children\\
 }
\end{tabular}
\end{exe}
\pause
\pause
\item Dative can only be expressed prenominally:
\ea
\gll das \alert{Den-Kindern}-Helfen\\
     the the-children-helping\\
\z

\end{itemize}
}

\frame{
\frametitle{All datives structural? Structural case and bivalent verbs}

\begin{itemize}
\item If structural/lexical is the only distinction available,\\
      there is a problem with bivalent verbs:
\eal
\ex 
\gll Er hilft ihm.\\
     he helps him.\DAT\\
\ex 
\gll Er unterstützt ihn.\\
     he supports him.\ACC\\
\zl
There has to be a difference between \emph{helfen} and \emph{unterstützen}.\\
Just saying the verbs require structural case, would not be enough.
\pause
\item For ditransitive verbs one can derive the dative case from general principles (Nom, Dat, Acc),
  but this does not work for bivalent verbs.

$\to$ Dative of \emph{helfen} is assumed to be lexical (mixed approach).

Prediction: dative passive is not possible with two-place verbs.

\end{itemize}
}

\frame{
\frametitle{Dative passive with bivalent verbs}

\savespace
\eal
\ex Er kriegte von vielen geholfen / gratuliert / applaudiert.
\ex Man kriegt täglich gedankt.
\zl

\pause
Attested data:
\eal
\ex "`Da kriege ich geholfen."'\footnote{
Frankfurter Rundschau, 26.06.1998, S.\,7.%
}
\ex
% auch nach applaudiert geholfen + bekommen und kriegen gesucht 21.09.2003
Heute morgen bekam ich sogar schon gratuliert.\footnote{%
Brief von Irene G.\ an Ernst G.\ vom 10.04.1943, Feldpost-Archive mkb-fp-0270}
%Branich IG-Vorsitzender Friedel Schönel meinte deshalb, 
\ex
"`Klärle"' hätte es wirklich mehr als verdient, auch mal zu einem "`unrunden"' Geburtstag gratuliert zu bekommen.\footnote{
Mannheimer Morgen, 28.07.1999, Lokales; "`Klärle"' feiert heute Geburtstag.%
}
\ex
Mit dem alten Titel von Elvis Presley "`I can't help falling in love"' bekam Kassier Markus Reiß zum Geburtstag gratuliert, [\ldots]\footnote{
%der dann noch viel später bekannte: "Ich hab' immer noch Gänsehaut, das war der schönste Teil meines Geburtstages." Doch auch die anderen Abteilungen des Bürgervereins können auf ein erfolgreiches Jahr 1998 zurückblicken.
Mannheimer Morgen, 21.04.1999, Lokales; Motor des gesellschaftlichen Lebens.%
}
\zl

Hence: Haider' approach: all datives have lexical case + trick for dative passive.

}


\subsubsection{Case assignment and passive as movement}


\frame{
\frametitle{Case assignment}



\begin{itemize}
\item Lexical case is assigned by the verb.
\pause
\item Verbs assign object case (accusative),\\
      if the object has structural case.
\pause
\item Finite Infl (or T in more recent versions) assigns nominative to the subject.
\bigskip
\pause
\item Case filter: Every NP has to have case.

\bigskip
\pause
\item Case is assigned under government, that is,\\
      only NPs in certain tree positions may get case.
\end{itemize}

}



\frame{
\frametitle{Case and passive as movement}


Assumptions regarding case and passive:
\begin{itemize}[<+->]
\item The subject gets case from I, the other arguments get case from V.
\item The passive blocks the subject (in the lexicon).
\item The accusative object gets a theta role but no case.
\item Therefore it has to move to a position where it gets case (move to SpecIP).
\end{itemize}


}

\frame{
\frametitle{Case and theta role assignment in the active}

\resizebox{.9\linewidth}{!}{
\begin{forest}
sm edges
[IP
  [{NP[nom]}, name=subject [der Mann;the man, roof]]
  [\hspaceThis{$'$}I$'$
	[VP
		[\hspaceThis{$'$}V$'$
			[{NP[dat]}, name=dobject [der Frau;the woman, roof]]
			[\hspaceThis{$'$}V$'$
				[{NP[acc]},   name=aobject [den Jungen;the boy, roof]]
				[V ,name=verb    [zeig-;show-]]]]]
	[I , name=Infl [-t;-s]]]]
\draw[->,dotted] (Infl.north) .. controls (3.5,-0.2) and (-1.5,0.4)  .. ($(subject.north)+(-.1,.1)$);
\draw[->]        (verb.north) .. controls (2.8,-2.9) and (-.4,.5)   .. ($(subject.north)+(0,.1)$);
\draw[->,dashed] (verb.north) .. controls (2.8,-3.0) and (0,-3.3)   .. ($(dobject.north)+(0,.1)$);
\draw[->,dashed] (verb.north) .. controls (2.3,-4.2) and (1.8,-4.6) .. ($(aobject.north)+(0,.1)$);
%\draw (-4,-7) to[grid with coordinates] (4,0.5);
\end{forest}\hspace{1cm}
\begin{tabular}[b]{ll@{}}
\tikz[baseline]\draw[dotted](0,1ex)--(1,1ex);&just case\\
\tikz[baseline]\draw(0,1ex)--(1,1ex);&just theta"=role\\
\tikz[baseline]\draw[dashed](0,1ex)--(1,1ex);&case and theta"=role
\\
\\
\end{tabular}
}

}

\frame{
\frametitle{Case and theta role assignment in the passive}



\resizebox{.9\linewidth}{!}{
\begin{forest}
sm edges
[IP
[{NP[nom]}, name=subject [der Junge$_i$;the boy ,roof]]
[\hspaceThis{$'$}I$'$
	[VP
		[\hspaceThis{$'$}V$'$
			[{NP[dat]}, name=dobject [der Frau;the woman, roof]]
			[\hspaceThis{$'$}V$'$
				[NP,   name=aobject [\_$_i$]]
				[V ,name=verb [gezeigt wir-;shown is, roof]]]]]
	[I  ,name=Infl [-\/d]]]]
\draw[->,dotted] (Infl.north) .. controls (2.5,.3)   and (-1.5,-.05) .. ($(subject.north)+(0,.1)$);
\draw[->,dashed] (verb.north) .. controls (2.2,-3.3)  and (0,-3.0)    .. ($(dobject.north)+(0,.1)$);
\draw[->]        (verb.north) .. controls (2.0,-4.3) and (1.2,-4.3) .. ($(aobject.north)+(0,.1)$);
%\draw (-3,-7) to[grid with coordinates] (3.6,0.5);
\end{forest}\hspace{1cm}
\begin{tabular}[b]{ll@{}}
\tikz[baseline]\draw[dotted](0,1ex)--(1,1ex);&just case\\
\tikz[baseline]\draw(0,1ex)--(1,1ex);&just theta"=role\\
\tikz[baseline]\draw[dashed](0,1ex)--(1,1ex);&case and theta"=role
\\
\\
\end{tabular}
}
}


\frame[shrink=15]{
\frametitle{Remarks on passive as movement analyses}


\begin{itemize}
\item The analysis works for English: the object has to move.

\eal
\ex[]{
The mother gave [the girl] [a cookie].
}
\ex[]{
{}[The girl] was given [a cookie] (by the mother).
}
\zl

\pause
\item But this is not the case for German:

\eal
\ex 
\gll weil das Mädchen dem Jungen \alert{den} \alert{Ball} schenkte\\
     because the.\NOM{} girl the.\DAT{} boy the.\ACC{} ball gave\\
\glt `because the girl gave the ball to the boy'
\ex 
\gll weil dem Jungen \alert{der} \alert{Ball} geschenkt wurde\\
	 because the.\DAT{} boy the.\NOM{} ball given was\\
\glt `because the ball was given to the boy'
\ex 
\gll weil \alert{der} \alert{Ball} dem Jungen geschenkt wurde\\
     because the.\NOM{} ball the.\DAT{} boy given was\\
\zl

(\mex{0}b) is the unmarked order \citep{Hoehle82a}, not (\mex{0}c).
\pause
That is: nothing has to be moved.

\item Solution: abstract movement. (empty expletive in subject position)

\pause
\item We will learn about alternative analyses not relying on such complicated mechanisms.


\end{itemize}


}



\subsection{Local reordering}

\frame[shrink=20]{
\frametitle{Local reordering}

The arguments of verbs can appear in any order in German.\\
So for verbs with three arguments, there are six possible orders for the arguments:

\eal
\label{ex-gb-umstellung}
\ex 
\gll {}[weil] \gruen{der} \gruen{Mann} \rot{dem} \rot{Kind} \blau{das} \blau{Buch} gibt\\
     \spacebr{}because the.\NOM{} man the.\DAT{} child the.\ACC{} book gives\\
\glt `because the man gives the book to the child'
\ex 
\gll {}[weil] \gruen{der} \gruen{Mann} \blau{das} \blau{Buch} \rot{dem} \rot{Kind} gibt\\
     \spacebr{}because the.\NOM{} man the.\ACC{} book the.\DAT{} child gives\\
\ex\label{ex-das-buch-der-mann-der-frau-gibt} 
\gll {}[weil] \blau{das} \blau{Buch} \gruen{der} \gruen{Mann} \rot{dem} \rot{Kind} gibt\\
     \spacebr{}because the.\ACC{} book the.\NOM{} man the.\DAT{} child gives\\
\ex 
\gll {}[weil] \blau{das} \blau{Buch} \rot{dem} \rot{Kind} \gruen{der} \gruen{Mann} gibt\\
     \spacebr{}because the.\ACC{} book the.\DAT{} child the.\NOM{} man gives\\
\ex 
\gll {}[weil] \rot{dem} \rot{Kind} \gruen{der} \gruen{Mann} \blau{das} \blau{Buch} gibt\\
     \spacebr{}because the.\DAT{} child the.\NOM{} man the.\ACC{} book gives\\
\ex 
\gll {}[weil] \rot{dem} \rot{Kind} \blau{das} \blau{Buch} \gruen{der} \gruen{Mann} gibt\\
     \spacebr{}because the.\DAT{} child the.\ACC{} book the.\NOM{} man gives\\
\zl

(\mex{0}a) is the so-called \alert{unmarked order} \citep{Hoehle82a}.\\
The number of contexts in which
sentences can be used is restricted for all other sentences in (\mex{0}).

}

\frame{
\frametitle{Movement or base-generation}

\begin{itemize}
\item Two suggestions:
\begin{itemize}
\item Assumption of a base order and derivation of all other orders by movement \citep{Frey93a}.
\pause
\item Base generation: all orders are derived in the phrase structure component without movement \citep{Fanselow2001a}.



\end{itemize}



\end{itemize}



}

\subsubsection{Movement}

\subsubsubsection{Adjunction to IP}

\frame{
\frametitle{Movement}

\centerline{%
\scalebox{.7}{%
\begin{forest}
sm edges
[IP
  [{NP[acc]$_i$} [das Buch;the book, roof]]
  [IP
    [{NP[nom]} [der Mann;the man, roof]]
    [\hspaceThis{$'$}I$'$
 	[VP
		[\hspaceThis{$'$}V$'$
			[{NP[dat]} [dem Kind;the child, roof]]
			[\hspaceThis{$'$}V$'$
				[NP   [\trace$_i$]]
				[V   [gib-;give-]]]]]
	[I , name=Infl [-t;-s]]]] ]
\end{forest}}
\hfill
\pause
\scalebox{.7}{%
\begin{forest}
sm edges
[IP
  [{NP[dat]}$_i$ [dem Kind;the child, roof]]
  [IP
    [{NP[acc]$_j$} [das Buch;the book, roof]]
    [IP
      [{NP[nom]} [der Mann;the man, roof]]
      [\hspaceThis{$'$}I$'$
   	[VP
		[\hspaceThis{$'$}V$'$
			[NP [\trace$_i$]]
			[\hspaceThis{$'$}V$'$
				[NP   [\trace$_j$]]
				[V   [gib-;give-]]]]]
	[I , name=Infl [-t;-s]]]] ]]
\end{forest}}}


}

\subsubsubsection{Problems of movement approaches}

\frame[shrink=10]{
\frametitle{Problems of movement approaches: Quantifier scope}

\begin{itemize}
\item Quantifier scope as motivation for movement-based approaches \citep{Frey93a}:%\citep[\page 185]{Frey93a}:

\ea 
\gll Es ist nicht der Fall, daß er mindestens einem Verleger fast jedes Gedicht anbot.\\
     it is not the case that he at.least one publisher almost every poem offered\\
\glt `It is not the case that he offered at least one publisher almost every poem.'
\z

(\mex{0}) has only one reading in which \emph{at least one} scopes over \emph{almost every}.

\pause
\ea 
\gll Es ist nicht der Fall, daß er fast jedes Gedicht$_i$ mindestens einem Verleger \_$_i$ anbot.\\
	 it is not the case that he almost every poem at.least one publisher {} offered\\
\glt `It is not the case that he offered almost every poem to at least one publisher.'
\z
(\mex{0}) has two readings.\\
One corresponds to the surface realization and one to the reading of (\mex{-1}).
\end{itemize}

}


\frame[shrink=10]{
\frametitle{Quantifier scope: Movement and recreation}

\begin{itemize}
\item Idea: Reconstruction of the moved items at D structure position.
\ea 
\gll Es ist nicht der Fall, daß er fast jedes Gedicht$_i$ mindestens einem Verleger \_$_i$ anbot.\\
	 it is not the case that he almost every poem at.least one publisher {} offered\\
\glt `It is not the case that he offered almost every poem to at least one publisher.'
\z
\pause

\item But this causes problems with two moved NPs \parencites{Kiss2001a}{Fanselow2001a}:%\parencites[\page 146]{Kiss2001a}[Section~2.6]{Fanselow2001a}:
\ea
\gll Ich glaube, dass mindestens einem Verleger$_i$ fast jedes Gedicht$_j$ nur dieser Dichter \_$_i$ \_$_j$ angeboten hat.\\
	 I believe that at.least one publisher almost every poem only this poet {} {} offered has\\
\glt `I think that only this poet offered almost every poem to at least one publisher.'
\z
Reconstructing \emph{mindestens einem Verleger} corresponds to a non-exiting reading. If two items
are moved. Their relative scope is fixed. They cannot reconstruct independently.
\end{itemize}

}

\frame{
\frametitle{Fix involving additional movements, some at PF}

\begin{itemize}
\item \citet{SE2002a} discuss the same problem in movement-based approaches to Japanese (in the
  Minimalist Program).
\pause
\item They suggest solving the problem by assuming additional movements some of them optionally
  taking place at PF without having semantic effects.
\pause
\item The resulting analysis is highly complex and involves additional assumptions, which begs the
  question as how such complex systems should be acquirable.

\end{itemize}


}

\subsubsection{Base generation}

\frame{
\frametitle{Base generation}

\begin{itemize}
\item Alternative: allow for the verb to combine with its arguments in any order.
 \citet{Fanselow2001a}: a base generation analysis (in Minimalism)% for the two objects
\pause
\item No account for (\mex{1}) in IP approach, since objects are before \gruen{subject}:

\eal
\ex
\gll {}[weil] \blau{das} \blau{Buch} \gruen{der} \gruen{Mann} \rot{dem} \rot{Kind} gibt\\
     \spacebr{}because the.\ACC{} book the.\NOM{} man the.\DAT{} child gives\\
\ex 
\gll {}[weil] \blau{das} \blau{Buch} \rot{dem} \rot{Kind} \gruen{der} \gruen{Mann} gibt\\
     \spacebr{}because the.\ACC{} book the.\DAT{} child the.\NOM{} man gives\\
\ex 
\gll {}[weil] \rot{dem} \rot{Kind} \gruen{der} \gruen{Mann} \blau{das} \blau{Buch} gibt\\
     \spacebr{}because the.\DAT{} child the.\NOM{} man the.\ACC{} book gives\\
\ex 
\gll {}[weil] \rot{dem} \rot{Kind} \blau{das} \blau{Buch} \gruen{der} \gruen{Mann} gibt\\
     \spacebr{}because the.\DAT{} child the.\ACC{} book the.\NOM{} man gives\\
\zl

\pause
\item IP-less base generation approach %\citet{Haider93a} 
seems to be the best option.\\
(also adopted in Categorial Grammar and HPSG)

\pause
\item Theta roles are assigned in tandem with argument selection. Not to positions.

\end{itemize}

}




\subsection{Summary}

\frame{
\frametitle{Summary}

Goals:
\begin{itemize}
\item Capture relations between certain structures, for example:
      \begin{itemize}
      \item active/passive
      \item verb last/verb initial/verb second position
      \item almost free order of constituents in the Mittelfeld and a certain base order
      \end{itemize}
      mapping from D Structure to S Structure
\pause
\item Explanation of language acquisition by
      \begin{itemize}
      \item assumption of a general rule schema holding for all languages and all structures\\
            (\xbar Theory)
      \item general principles holding for all languages but parameterizable
      \end{itemize}

\end{itemize}

}


\frame{
\frametitle{Exercise}

\smallframe
Draw the syntax trees for the fowllowing sentences:
\eal
\ex 
\gll dass der Delphin dem Kind hilft\\
     that the.\NOM{} dolphin the.\DAT{} child helps\\
\glt `that the dolphin helps the child'
\ex 
\gll dass der Delphin den Hai attackiert\\
     that the.\NOM{} dolphin the.\ACC{} shark attacks\\
\glt `that the dolphin attacks the shark'
\ex 
\gll dass der Hai attackiert wird\\
     that the.\NOM{} shark attacked is\\
\glt `that the shark is attacked'
\ex 
\gll Der Hai wird attackiert.\\
     the.\NOM{} shark is attacked\\
\glt `The shark is attacked.'
\ex 
\gll Der Delphin hilft dem Kind.\\
     the dolphin.\NOM{} helps the.\DAT{} child\\
\glt `The dolphin is helping the child.'
\zl

}


%      <!-- Local IspellDict: en_US-w_accents -->
