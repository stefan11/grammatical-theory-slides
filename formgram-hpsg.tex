\section{Head-Driven Phrase Structure Grammar (HPSG)}

\subtitle{Head-Driven Phrase Structure Grammar (HPSG)}

\huberlintitlepage[22pt]


\outline{

\begin{itemize}
\item Introduction and basic terms
\item Phrase structure grammar and \xbar Theory
\item Government \& Binding (GB)
\item {Generalized Phrase Structure Grammar (GPSG)}
\item {Feature descriptions, feature structures and models}
\item {Lexical Functional Grammar (LFG)}
%\item PATR
\item {Categorial Grammar (CG)}
\item \alert{Head-Driven Phrase Structure Grammar (HPSG)}
%\item Konstruktionsgrammatik (CxG)
\item Tree Adjoning Grammar (TAG)
\end{itemize}

%\tableofcontents
}

\frame{
\frametitle{Reading material}

\citew[Chapter~9]{MuellerGT-Eng}

}



\frame{
\frametitle{Head-Driven Phrase Structure Grammar (HPSG)}


\begin{itemize}[<+->]
\item developed by Carl Pollard and Ivan Sag in the mid-80s in Stanford and in the Hewlett"=Packard
research laboratories in Palo Alto\\
      \citep{ps,ps2}; see \citep*{FWPEvolution} for history
\item Ivan Sag was one of the developers of GPSG, Pollard worked in a version of CG.
\item HPSG is part of West-Coast linguistics (LFG, BCG).
\item Hotspots: Columbus (Ohio), Buffalo, Germany, Paris, Seoul
\item Teaching material and overviews:\\
      \citew{MuellerLehrbuch,MuellerArten,LM2006a,MuellerCurrentApproaches}\nocite{Mueller99a,Mueller2002b}
\item 1500+ page handbook on HPSG: \citet*{HPSGHandbook}
\end{itemize}




}


\subsection{General remarks on representational format}

\outline{
\begin{itemize}
\item General remarks on the representational format
\item Passive
\item Verb position
\item Local reordering (aka scrambling)
\item Long distance dependencies
\item Summary and classification
\end{itemize}

}


\frame{

\frametitle{General remarks on HPSG}
\small
\begin{itemize}
\item lexicalized (head-driven)
\pause
\item sign-based \citep{Saussure16a-de}
\pause
%\item unifikationsbasiert
\item typed feature structures (lexical items, phrases, principles)
\pause
\item multiple inheritance
\pause
\item monostratal theory\\~\\
\begin{minipage}[t]{2.5cm}
~\\[-20mm]
\begin{itemize}
\item \blau<6>{phonologie}\rnode{1}{}
\item \blau<7>{syntax}\rnode{2}{}
\item \blau<8>{semantics}\rnode{3}{}
\end{itemize}
\end{minipage}%
\parbox[t]{3cm}{
\resizebox{5cm}{!}{
\(
\ms[word]{
\rnode{4}{phon}   & \blau<6>{\phonliste{ Grammatik }} \\[1mm]
synsem$|$loc & \ms[loc]{ \rnode{5}{cat}  & \blau<7>{\ms[cat]{ head & \ms[noun]{ case & \ibox{1}\\
                                                       }\\[6mm]
                                       spr & \liste{ DET[{\sc case}~\ibox{1}] } \\
                                     }} \\[6mm]
              \rnode{6}{cont} & \blau<8>{\ldots \ms[grammatik]{ inst & X \\
                                   }}\\
            }\\
}
\)
}
}
\end{itemize}
}

\frame{
\frametitle{Influences}

\begin{itemize}[<+->]
\item Categorial Grammar\\
      (functor-argument structures, valence, argument composition)
\item GPSG\\
      (ID/LP format, Slash mechanism for nonlocal dependencies)
\item Government \& Binding\\
      (for example analysis of verb position in German)
\item Construction Grammar\\
      (increased use of inheritance hierarchies for phrasal aspects, \citealp{Sag97a,Sag2010b,Sag2012a})
\end{itemize}



}


\frame{
\frametitle{Valence and grammar rules: PSG}


\begin{itemize}
\item lage number of rules:\\
\begin{tabular}[t]{@{}l@{~$\to$~}l@{\hspace{3em}}l@{}}
      S & NP[\type{nom}], V                             & \emph{X schläft} `X is sleeping'\\
      S & NP[\type{nom}], NP[\type{acc}], V                         & \emph{X Y erwartet} `X expects Y'\\
      S & NP[\type{nom}], PP[\type{über}], V           & \emph{X über Y spricht} `X talks about Y'\\
      S & NP[\type{nom}], NP[\type{dat}], NP[\type{acc}], V                     & \emph{X Y Z gibt} `X gives Z to Y'\\
      S & NP[\type{nom}], NP[\type{dat}], PP[\type{mit}], V        & \emph{X Y mit Z dient} `X serves Y with Z'\\
      \end{tabular}
\pause
\item Verbs have to be used with an appropriate rule.
\end{itemize}

}

\frame{

\frametitle{Valence and grammar rules: HPSG}

\begin{itemize}
\item Arguments are represented as complex categories in the lexical representation of the head (as
  in Categorial Grammar).
\pause
\item \begin{tabular}[t]{@{}lll}
      verb             & \comps\\
      \emph{schlafen} `to sleep'  & \sliste{ NP[\type{nom}] }\\
      \emph{erwarten} `to expect' & \sliste{ NP[\type{nom}], NP[\type{acc}] }\\
      \emph{sprechen} `to speak'  & \sliste{ NP[\type{nom}], PP[\type{über}] }\\
      \emph{geben}    `to give'   & \sliste{ NP[\type{nom}], NP[\type{dat}], NP[\type{acc}] }\\
      \emph{dienen}   `to serve'  & \sliste{ NP[\type{nom}], NP[\type{dat}], PP[\type{mit}] }\\  
      \end{tabular}

\end{itemize}


}

\frame{
\frametitle{Example with valence information: Intransitive verb}

\vfill
\hfill
\begin{forest}
sm edges
[V{[\comps \eliste]}
	[{\ibox{1} NP[\type{nom}]}
		[Peter;Peter]]
	[V{[\comps \sliste{ \ibox{1} }]}
		[schläft;sleeps]]]
\end{forest}
\hfill\hfill\mbox{}
\vfill
V[\comps \sliste{ }] corresponds to a fully saturated phrase (VP or S)
\vfill
}

\frame{
\frametitle{Example with valence information: Transitive verb}

\vfill
\hfill
\begin{forest}
sm edges
[V{[\comps \eliste]}
	[\gruen<2>{\ibox{1} NP[\type{nom}]}
		[Peter;Peter]]
	[V{[\comps \sliste{ \gruen<2>{\ibox{1}} }]}
		[\gruen<1>{{\ibox{2} NP[\type{acc}]}}
			[Maria;Maria]]
		[V{[\comps \sliste{ \ibox{1}, \gruen<1>{\ibox{2}} }]}
			[erwartet;expects]]]]
\end{forest}\hfill\hfill\mbox{}
\vfill

}

\frame{
\frametitle{SOV vs.\ SVO: Representation of subjects}

\begin{itemize}
\item Researchers working on German assume that the subject of finite verbs behaves like the other
  arguments. (\citealp{Pollard90a-Eng}; \citealp[\page 376]{Eisenberg94b})

HPSG: subjects and complements are listed in one valence list (\comps).

\pause
\item English: subjects are different.
\pause
\item \argst as a underlying representation containing all arguments. \citep{DKW2021a}
\pause
\item Language dependent mapping to valence features \spr and \comps.

\medskip
\oneline{%
\begin{tabular}[t]{@{}llll}
      verb          & \spr                      & \comps                                     & \argst\\
      \emph{sleep}  & \sliste{ NP[\type{nom}] } & \sliste{}                                  & \sliste{ NP[\type{nom}] }\\
      \emph{expect} & \sliste{ NP[\type{nom}] } & \sliste{ NP[\type{acc}] }                  & \sliste{ NP[\type{nom}], NP[\type{acc}] }\\
      \emph{speak}  & \sliste{ NP[\type{nom}] } & \sliste{ PP[\type{about}] }                & \sliste{ NP[\type{nom}], PP[\type{about}] }\\
      \emph{give}   & \sliste{ NP[\type{nom}] } & \sliste{ NP[\type{acc}], NP[\type{acc}] }  & \sliste{ NP[\type{nom}], NP[\type{acc}], NP[\type{acc}] }\\
      \emph{serve}  & \sliste{ NP[\type{nom}] } & \sliste{ NP[\type{acc}], PP[\type{with}] } & \sliste{ NP[\type{nom}], NP[\type{acc}], PP[\type{with}] }\\  
      \end{tabular}}

\end{itemize}


}

\frame{
\frametitle{Example analysis with \spr and \comps}

\centerline{%
\scalebox{.8}{%
\begin{forest}
sm edges
[V{\feattab{\spr \eliste,\\
            \comps \eliste}}
  [\ibox{1} NP [Kim]]
  [V{\feattab{\spr \sliste{ \ibox{1} },\\
              \comps \eliste}}
    [V{\feattab{\spr \sliste{ \ibox{1} },\\
                \comps \sliste{ \ibox{2} }}}
      [talks]]
    [\ibox{2} P{\feattab{\spr \sliste{ },\\
                \comps \sliste{ }}}
      [P{\feattab{\spr \sliste{ },\\
                \comps \sliste{ \ibox{3} }}} [about]]
      [\ibox{3} N{\feattab{\spr \sliste{ },\\
                     \comps \sliste{ }}}
        [\ibox{4} Det [the]]
        [N{\feattab{\spr \sliste{ \ibox{4} },\\
                     \comps \sliste{ }}} [summer] ]]]]]
\end{forest}}}

}


\subsubsection{Representation of constituent structure}

\frame{
\frametitle{Representation of constituent structure}

\centerline{%
\begin{forest}
sm edges
[NP
	[Det
		[dem;the]]
	[N
		[Mann;man]]]
\end{forest}
}

The tree can be represented in feature descriptions:

\ea
\ms{ 
  phon     & \phonliste{ dem Mann }\\[1mm]
  head-dtr & \onems{ phon \phonliste{ Mann }
                 }\\
  non-head-dtrs & \sliste{ \onems{ phon \phonliste{ dem }
                            }}
}
\z

}

\subsubsection{Feature geometry}

\frame{
\frametitle{Complete feature geometry}

\ea
\label{LE-Grammatik}
\scalebox{.7}{%
\ms[word]{
phon   & \phonliste{ Grammatik } \\[1mm]
synsem & \ms{ loc & \ms[local]{ cat  & \ms[category]{ head & \ms[noun]{ case & \ibox{1}
                                                                      }\\[3mm]
                                                     spr & \sliste{ Det[\textsc{case}~\ibox{1}] }\\
                                                     comps & \eliste\\[1pt]
                                                    } \\[6mm]
                                cont & \ms[mrs]{
                                       ind & \ibox{2} \ms{ per & third\\
                                                           num & sg\\
                                                           gen & fem\\
                                                         }\\
                                       rels & \sliste{ \ms[grammatik]{ inst & \ibox{2} 
                                                                    } }
                                        }
                              }\\
               nonloc & \ms{ inher$|$slash   & \eliste{}\\
                             to-bind$|$slash & \eliste{}\\
                           }
            }
}}
\z

Information that is needed for structure sharing is grouped together.


}

\subsubsection{ID schemata}

\frame{
\frametitle{The Head-Complement Schema (preliminary)}


\type{head-complement-phrase}\istype{head"=complement"=phrase} \impl\\
\onems{
      synsem$|$loc$|$cat$|$comps \ibox{1} \\
      head-dtr$|$synsem$|$loc$|$cat$|$comps \ibox{1} $\oplus$ \sliste{ \ibox{2} } \\
      non-head-dtrs \sliste{ [ \synsem \ibox{2} ] }
      }

\pause

\ea
\onems[head-complement-phrase]{
phon \phonliste{ Peter schläft }\\
synsem$|$loc$|$cat$|$comps \eliste\\
head-dtr \onems{ phon \phonliste{ schläft }\\
                 synsem$|$loc$|$cat$|$comps \sliste{ \ibox{1} NP[\type{nom}] }
               }\\
non-head-dtrs \sliste{ \onems{ phon \phonliste{ Peter }\\
                               \synsem \ibox{1}
                             } }
}
\z


}

\subsubsection{LP rules}

\frame{
\frametitle{Linearization rules}

\eal
\ex\label{lp-ini-arg} 
Head[\initial$+$] $<$ Complement
\ex 
Complement $<$ Head[\initial --]
\zl

\pause
Prepositions have an \initialv `$+$' and therefore have to precede arguments. 
\eal
\ex[]{
\gll {}[in [den Schrank]]\\
     \spacebr{}in \spacebr{}the cupboard\\
}
\ex[*]{
\gll {}[[den Schrank] in]\\
     \hspaceThis{[[}the cupboard in\\
}
\zl
\pause
Verbs in final position bear the value `$-$' and have to follow
their arguments.
\eal
\ex[]{
\gll {}dass [er [ihn umfüllt]]\\
     {}that \spacebr{}he \spacebr{}it decants\\
}
\ex[*]{
\gll {}dass [er [umfüllt ihn]]\\
     {}that \spacebr{}he \spacebr{}decants it\\
}
\zl


}


\subsubsection{Head features}

\frame{
\frametitle{Head features}

\begin{itemize}
\item Information about verb form has to be present at the top-most node of a projection:
\eal
\label{bsp-projektion-v-merkmale}
\ex[]{
\gll {}[Dem Mann helfen] will er nicht.\\
 {}\spacebr{}the man help wants he not\\
\glt `He doesn't want to help the man.'
}
\ex[]{
\gll {}[Dem Mann geholfen] hat er nicht.\\
{}\spacebr{}the man helped has he not\\
\glt `He hasn't helped the man.'
}
\ex[*]{
\gll {}[Dem Mann geholfen] will er nicht.\\
{}\spacebr{}the man helped wants he not\\
}
\ex[*]{
\gll{}[Dem Mann helfen] hat er nicht.\\
{}\spacebr{}the man help has he not\\
}
\zl
\end{itemize}
}


\frame{
\frametitle{Projection of features along the head path}

\settowidth{\offset}{V[\type{fi}}
\settowidth{\offsetup}{V[\type{fin}}
\centerline{
\begin{forest}
sm edges, for tree={l+=\baselineskip}
[\gruen{V}{[\gruen{\type{fin}}, \comps \eliste]}, name=fin1
	[\ibox{1} NP{[\type{nom}]}
		[jemand;somebody]]
	[\gruen{V}{[\gruen{\type{fin}}, \comps \sliste{ \ibox{1} }]}, name=fin2
		[\ibox{2} NP{[\textit{dat}]}
			[dem Kind;the child,roof]]
		[\gruen{V}{[\gruen{\type{fin}}, \comps \sliste{ \ibox{1}, \ibox{2} }]}, name=fin3
			[\ibox{3} NP{[\textit{acc}]}
				[das Buch;the book,roof]]
			[\gruen{V}{[\gruen{\type{fin}}, \comps \sliste{ \ibox{1}, \ibox{2}, \ibox{3} }]}, name=fin4
				[gibt;gives]]]]]	
tikz={\draw[<->] ($(fin1.south west)+(\offsetup,0)$) to ($(fin2.north west)+(\offset,0)$);
      \draw[<->] ($(fin2.south west)+(\offsetup,0)$) to ($(fin3.north west)+(\offset,0)$);
      \draw[<->] ($(fin3.south west)+(\offsetup,0)$) to ($(fin4.north west)+(\offset,0)$);}
\end{forest}
}

}

\frame{
\frametitle{Structure sharing of \head values}

\centerline{
\scalebox{.8}{%
\begin{forest}
sm edges
[\ms{head & \gruen{\ibox{1}}\\
     comps & \sliste{ }
     }
	[{\ibox{2} NP{[\type{nom}]}}
		[jemand;somebody]]
	[\ms{
             head & \gruen{\ibox{1}}\\
             comps & \sliste{ \ibox{2} }
             }
		[\ibox{3} NP{[\textit{dat}]}
			[dem Kind;the child, roof]]
		[\ms{
                                                                                   head & \gruen{\ibox{1}}\\
                                                                                   comps & \sliste{ \ibox{2}, \ibox{3} }
                                                                                    }
			[\ibox{4} NP{[\textit{acc}]}
				[das Buch;the book, roof]]
			[\ms{
                                                                                   head & \gruen{\ibox{1} \ms[verb]{
                                                                                                  vform & fin
                                                                                                  }}\\
                                                                                   comps & \sliste{ \ibox{2}, \ibox{3}, \ibox{4} }
                                                                                    }
				[gibt;gives]]]]]	
\end{forest}}}
}


\subsubsection{Type hierarchies and inheritance}


\frame{
\frametitle{Type hierarchies and inheritance}


\centerline{%
\begin{forest}
type hierarchy
[sign
  [word]
  [phrase 
    [non-headed-phrase]
    [headed-phrase [head-complement-phrase]]]]
\end{forest}}

\begin{itemize}
\item All feature structures are typed in HPSG. 
\pause
\item Types are ordered in hierarchies.
\pause
\item Subtypes inherit constraints from supertypes.

\pause

\item Example: \type{headed-phrase}
\ea
\type{headed"=phrase}\istype{headed"=phrase} \impl
\ms{ 
synsem$|$loc$|$cat$|$head \ibox{1}\\
head-dtr$|$synsem$|$loc$|$cat$|$head \ibox{1}\\
} 
\z


\end{itemize}


}


\frame{
\frametitle{Inheritance of constraints}

\begin{itemize}
\item
\ea
\label{head-arg-schema-hfp}
Head-Complement Schema + Head Feature Principle:\\
\onems[head-complement-phrase~]{
synsem$|$loc$|$cat  \ms{ \visible<2->{\gruen{head}   & \gruen{\ibox{1}}} \\
                          comps & \ibox{2}
                        }\\
head-dtr$|$synsem$|$loc$|$cat \ms{ \visible<2->{\gruen{head}   & \gruen{\ibox{1}}} \\
                                   comps & \ibox{2} $\oplus$ \sliste{ \ibox{3} }
                                 } \\
non-head-dtrs   \sliste{ [ synsem \ibox{3} ] }
}
\z
\medskip

Constraints on \type{head-complement-phrase} \pause
and inherited constraints from \type{headed-phrase}
\pause
\item Inheritance hierarchies are important for capturing generalizations.\\
They have been used in the lexicon since \citew*{FPW85a}.
\end{itemize}


}



\subsection{Passive}

\outline{
\begin{itemize}
\item General remarks on the representational format
\item \alert{Passive}
\item Verb position
\item Local reordering (aka scrambling)
\item Long distance dependencies
\item Summary and classification
\end{itemize}

}


\frame{
\frametitle{Passive}

\begin{itemize}[<+->]
\item HPSG follows Bresnan's argumentation that passive should be treated lexically.

\item A lexical rule takes a verb stem as input and licenses a participle form.\\
The most prominent argument (the designated argument) is suppressed.

\item Since grammatical functions are not parts of the theory,\\
      mapping principles mapping objects onto subjects are not needed.

\item But the change of case in passives has to be explained.
\end{itemize}



}

\subsubsection{Structural case}

\frame{
\frametitle{Structural and lexical case}

\begin{itemize}
\item Case depending on the syntactic environment is called \alert{structural case}.
      Otherwise the case is \alert{lexical case}.
\pause
\item Examples of structural case:
\eal
\ex 
\gll \alert{Der} \alert{Installateur} kommt.\\
	 the.\NOM{} plumber comes\\
\glt `The plumber is coming.'
\pause
\ex 
\gll Der Mann lässt \alert{den} \alert{Installateur} kommen.\\
	 the man lets the.\ACC{} plumber come\\
\glt `The man is getting the plumber to come.'
\pause
\ex 
\gll das Kommen \alert{des} \alert{Installateurs}\\
	 the coming of.the plumber\\
\glt `the plumber's visit'
\zl

\end{itemize}

}

\frame{
\frametitle{Structural case: The object}

\begin{itemize}
\item Object (accusative in the active) can be realized as nominative and genitive:

\eal
\ex 
\gll Judit schlägt \alert{den} \alert{Weltmeister}.\\
     Judit beats the.\ACC{} world.champion\\
\glt `Judit beats the world champion.'
\pause
\ex 
\gll \alert{Der} \alert{Weltmeister} wird geschlagen.\\
     the.\NOM{} world.champion is beaten\\
\glt `The world champion is being beaten.'
\pause
\ex 
\gll  das Schlagen \alert{des} \alert{Weltmeisters}\\
      the beating  of.the world.champion\\
\zl
\end{itemize}
}

\subsubsection{Lexical case}

\frame{
\frametitle{Lexical case}

\begin{itemize}
\item genitive depending on the verb is lexical case:\\
The case of the genitive object does not change in passivization.
\eal
\ex[]{
\gll Wir gedenken \alert{der} \alert{Opfer}.\\
     we remember the.\GEN{} victims\\
}
\ex[]{ 
\gll \alert{Der} \alert{Opfer} wird gedacht.\\
	 the.\GEN{} victims are remembered\\
\glt `The victims are being remembered.'
}
\ex[*]{
\gll \alert{Die} \alert{Opfer} wird / werden gedacht.\\
     the.\NOM{}  victims       is   {} are      remembered\\    
}
\zl
\pause
(\mex{0}b) = impersonal passive, there is no subject.

\pause
\item I count the dative of dative objects of verbs among the lexical cases.\\
      See \citew{MuellerLehrbuch}.

\end{itemize}

}


\subsubsection{Valence information and the Case Principle}


\frame{
\frametitle{Valence information and the Case Principle}



%\begin{principle-break}[\hypertarget{case-p}{Case Principle (simplified)}]
\label{case-p}
Case Principle (simplified)
\begin{itemize}
\item The first element with structural case in the argument structure list of a verb receives nominative.
\item All other elements in the argument structure list of a verb with structural case receive accusative.
\item In nominal environments, elements with structural case are assigned genitive\is{case!genitive}.
\end{itemize}
%\end{principle-break}

\bigskip
Based on \citet*{YMJ87}.

Also works for Icelandic and other Germanic languages and also for Hindi.


}

\subsubsubsection{Active}

\frame{
\frametitle{Active}

prototypical valence lists for finite verbs:
\ea
\label{ex-verben-active}
\begin{tabular}[t]{@{}l@{~}l@{~}l}
a. & \emph{schläft} `sleeps':       & \argst \sliste{ NP[\type{str}]$_j$ }\\[2pt]
b. & \emph{unterstützt} `supports': & \argst \sliste{ NP[\type{str}]$_j$, NP[\type{str}]$_k$ }\\[2pt]
c. & \emph{hilft} `helps':          & \argst \sliste{ NP[\type{str}]$_j$, NP[\type{ldat}]$_k$ }\\[2pt]
d. & \emph{schenkt} `gives':        & \argst \sliste{ NP[\type{str}]$_j$, NP[\type{ldat}]$_k$, NP[\type{str}]$_l$ }\\
\end{tabular}
\z
\emph{str} stands for \emph{structural} and \emph{ldat} for lexical dative. 

\pause
The first element of the \argstl with structural case gets nominative.\\
All others with structural case get accusative.

}

\subsubsubsection{Case assignment in the passiv}

\frame[shrink]{
\frametitle{Passive}

\ea
\begin{tabular}[t]{@{}l@{~}l@{~}l}
a. & \emph{schläft} `sleeps':       & \argst \sliste{ NP[\type{str}]$_j$ }\\[2pt]
b. & \emph{unterstützt} `supports': & \argst \sliste{ NP[\type{str}]$_j$, NP[\type{str}]$_k$ }\\[2pt]
c. & \emph{hilft} `helps':          & \argst \sliste{ NP[\type{str}]$_j$, NP[\type{ldat}]$_k$ }\\[2pt]
d. & \emph{schenkt} `gives':        & \argst \sliste{ NP[\type{str}]$_j$, NP[\type{ldat}]$_k$, NP[\type{str}]$_l$ }\\
\end{tabular}
\z

\pause
Passivization results in the following \argst lists:
\ea
\begin{tabular}[t]{@{}l@{~}l@{~}l}
a. & \emph{geschlafen}  `slept':     & \argst \sliste{ }\\[1mm]
b. & \emph{unterstützt} `supported': & \argst \sliste{ NP[\type{str}]$_k$ }\\[1mm]
c. & \emph{geholfen}    `helped':    & \argst \sliste{ NP[\type{ldat}]$_k$ }\\[1mm]
d. & \emph{geschenkt}   `given':     & \argst \sliste{ NP[\type{ldat}]$_k$, NP[\type{str}]$_l$ }\\
\end{tabular}
\z
Different NP in first position. If it has structural case, it gets nominative.\\
If the case is not structural it remains as is: lexically specified.
}

\subsection{Verb position}

\outline{
\begin{itemize}
\item General remarks on the representational format
\item Passive
\item \alert{Verb position}
\item Local reordering (aka scrambling)
\item Long distance dependencies
\item Summary and classification
\end{itemize}

}


\frame{
\frametitle{Verb position}

\begin{itemize}
\item \citet{Hoehle97a}: Finite verbs and complementizers form a natural class:

\eal
\ex 
\gll \gruen{dass} [jeder diesen Roman \rot<2->{kennt}]\\
     that \spacebr{}everybody this novel knows\\
\glt `that everybody knows this novel'
\ex 
\gll \gruen{Kennt} [jeder diesen Roman \rot<3->{\_} ]\\
	 knows \spacebr{}everybody this novel\\
\glt `Does everybody know this novel?'
\zl

\pause

\item The complementizer takes a clause with verb-final verb.
\pause
\item The initial finite verb takes a verb-final clause with the verb at the end missing.
\end{itemize}


}



\frame[shrink=5]{
\frametitle{\large Representations and lexical rules: Verb movement}

\vfill
\medskip
\hfill%
%\scalebox{0.85}{%
\begin{forest}
sm edges
[VP
	[\gruen<2-3>{V \sliste{ VP\only<4>{\gruen{/\!/V}} }}, name=vini
	   [\gruen<3>{V},name=vlast [kennt$_k$;knows]]]
	[\gruen<2>{VP}\only<4>{\gruen{/\!/V}}, name=vp
	   [NP [jeder;everyone]]
	   [V$'$\only<4>{\gruen{/\!/V}}, name=vbar
	     [NP [diesen Roman;this novel, roof]]
		[\gruen<1>{V\only<4>{\gruen{/\!/V}}},name=vtrace [ \gruen<1>{\_$_k$}]]]]]
%\draw[<->] (vone) to (vtwo);
%%\draw (-2,-5) to[grid with coordinates] (4,0.5);
%% \draw[<-] (3,-3.4) .. controls (3.2,-3.6) .. (3.5,-3.4)
%%                    .. controls ()         .. (;
\visible<4>{\gruen<4>{
\draw[<->] ($(vtrace.south)+(-.25,.1)$)    to [bend right=45]  ($(vtrace.south)+(.25,.1)$);
\draw[<->] (vtrace)                        to [out=45, in=0]  (vbar);
\draw[<->] ($(vbar.north east)+(-0.2,0)$)  to [out=80, in=0]  (vp);
\draw[<->] ($(vp.north east)+(-0.25,-.1)$)  to [out=145,in=35] ($(vini.north east)+(-.5,-.1)$);
\draw[<->] ($(vini.south east)+(-.45,.1)$) to [bend left=30] ($(vlast.north east)+(-.1,-.1)$);
}}
\end{forest}
%}
\hfill\hfill\mbox{}
\vfill

\begin{itemize}[<+->]
\item There is a trace in verb-final position.
\item The verb in initial position is a special form of the verb\\selecting a projection of the verb trace.
\item This special lexical item is licensed by a lexical rule.
\item Connection between verb and trace is done via percolation of information in the tree.
\end{itemize}

}


\subsection{Local reordering}

\outline{
\begin{itemize}
\item General remarks on the representational format
\item Passive
\item Verb position
\item \alert{Local reordering (aka scrambling)}
\item Long distance dependencies
\item Summary and classification
\end{itemize}

}


\frame{
\frametitle{Local reordering}


\begin{itemize}
\item Arguments can appear in almost any order in the German \mf.
\eal
\ex 
\gll {}[weil] \gruen{der} \gruen{Delphin} \rot{dem} \rot{Kind} \blau{den} \blau{Ball} gibt\\
     \spacebr{}because the.\NOM{} dolfin the.\DAT{} child the.\ACC{} ball gives\\
\glt `because the dolfin gives the ball to the child'
\ex 
\gll {}[weil] \gruen{der} \gruen{Delphin} \blau{den} \blau{Ball} \rot{dem} \rot{Kind} gibt\\
     \spacebr{}because the.\NOM{} dolfin the.\ACC{} ball the.\DAT{} child gives\\
\ex\label{ex-den-buch-der-dolfinn-der-frau-gibt} 
\gll {}[weil] \blau{den} \blau{Ball} \gruen{der} \gruen{Delphin} \rot{dem} \rot{Kind} gibt\\
     \spacebr{}because the.\ACC{} ball the.\NOM{} dolfin the.\DAT{} child gives\\
\ex 
\gll {}[weil] \blau{den} \blau{Ball} \rot{dem} \rot{Kind} \gruen{der} \gruen{Delphin} gibt\\
     \spacebr{}because the.\ACC{} ball the.\DAT{} child the.\NOM{} dolfin gives\\
\ex 
\gll {}[weil] \rot{dem} \rot{Kind} \gruen{der} \gruen{Delphin} \blau{den} \blau{Ball} gibt\\
     \spacebr{}because the.\DAT{} child the.\NOM{} dolfin the.\ACC{} ball gives\\
\ex 
\gll {}[weil] \rot{dem} \rot{Kind} \blau{den} \blau{Ball} \gruen{der} \gruen{Delphin} gibt\\
     \spacebr{}because the.\DAT{} child the.\ACC{} ball the.\NOM{} dolfin gives\\
\zl
\end{itemize}

}


\frame{
\frametitle{Local reordering: Three options}

Two approaches:
\begin{itemize}
\item flat structures like in GPSG
\pause
\item binary branching structures with arbitrary order of combination
\pause
\item lexical rules reordering the elements in the valence lists

\end{itemize}

}

\subsubsection{Binary branching structures}

\frame{
\frametitle{Example: Normal order (nom, acc)}

\eal
\ex 
\gll {}[weil] \gruen{jeder} \blau{diesen} \blau{Roman} kennt\\
	 {}\spacebr{}because everyone.\NOM{} this.\ACC{} novel knows\\
\ex 
\gll {}[weil] \blau{diesen} \blau{Roman} \gruen{jeder} kennt\\
	 {}\spacebr{}because this.\ACC{} novel everyone.\NOM{} knows\\
\glt `because everyone knows this novel'
\zl

\centerline{%
\scalebox{.9}{%
\begin{forest}
sm edges
[V{[\comps \sliste{}]}
	[\gruen<2>{\ibox{1} NP[\type{nom}]}
		[jeder;everybody]]
	[V{[\comps \sliste{ \gruen<2>{\ibox{1}} }]}
		[\blau{\ibox{2} NP[\type{acc}]}
			[diesen Roman;this novel, roof]]
		[V{[\comps \sliste{ \ibox{1}, \blau{\ibox{2}} }]}
			[kennt;knows]]]]
\end{forest}
}}%

}

\frame{
\frametitle{Example: Marked order (acc, nom)}


\centerline{%
\begin{forest}
sm edges
[V{[\comps \sliste{}]}
	[\blau<2>{\ibox{2} NP[\type{acc}]}
		[diesen Roman;this novel, roof]]
	[V{[\comps \sliste{ \blau<2>{\ibox{2}} }]}
        	[\gruen{\ibox{1} NP[\type{nom}]}
	        	[jeder;everybody]]
		[V{[\comps \sliste{ \gruen{\ibox{1}}, \ibox{2} }]}
			[kennt;knows]]]]
\end{forest}
}%

\medskip
Difference in order of saturation of elements in the \comps list.
}

\frame{
\frametitle{Generalized Head-Complement Schema}

\begin{itemize}
\item Earlier version: An element was taken off from the end of the \comps list.
\pause
\item We permit to take an element from any position of the \comps list.
\pause
\item We use append to split the list in three parts:\\
      a beginning, a one-element list, an end

\medskip
\type{head"=complement"=phrase} \impl\\*
\onems{
      synsem$|$loc$|$cat$|$comps \ibox{1} $\oplus$ \ibox{3}\\
      head-dtr$|$synsem$|$loc$|$cat$|$comps \ibox{1} $\oplus$ \sliste{ \ibox{2} } $\oplus$ \ibox{3}\\
      non-head-dtrs \sliste{ [ \textsc{synsem} \ibox{2} ] }\\
}
\medskip

\pause
\item strict VO: We take arguments from the beginning of the list (\ibox{1} = \eliste).
\pause
\item strict OV: We take arguments from the end of the list (\ibox{3} = \eliste).
\pause
\item VO/OV with scrambling: We take arguments from wherever.

\end{itemize}

}


\subsection{Long"=distance dependencies}

\outline{
\begin{itemize}
\item General remarks on the representational format
\item Passive
\item Verb position
\item Local reordering (aka scrambling)
\item \alert{Long distance dependencies}
\item Summary and classification
\end{itemize}

}


\frame{

%\frametitle{Repräsentationen und Lexikonregeln: 
\frametitle{Long"=distance dependencies}

\vfill
\medskip
\hfill%
\scalebox{0.68}{
\settowidth{\offset}{N}
\centering
\begin{forest}
sm edges
[VP
	[\gruen<3>{NP},name=np
		[diesen Roman$_i$;this novel,roof]]
	[VP\only<2->{\gruen<2>{/\gruen<3>{NP}}},name=vpnp2
		[V
			[V
				[kennt$_k$;knows]]]
		[VP\only<2->{\gruen<2>{/NP}},name=vpnp1
			[\gruen<1>{NP}\only<2->{\gruen<2>{/NP}}, name=npnp
				[\gruen<1>{\_$_i$}]]
			[\hspaceThis{$'$}V$'$
				[NP
					[jeder;everyone]]
				[V
				  [\trace$_k$]]]]]]
\visible<2->{\gruen<2>{%
\draw[<->] ($(npnp.east)$)  to [bend right=45] ($(vpnp1.south east)+(-.25,.1)$);
\draw[<->] ($(vpnp1.north east)+(-.26,-.1)$)  to [bend right=45] ($(vpnp2.east)+(-0,0)$);
}}
\visible<3->{
\draw[<->] ($(vpnp2.north)+(.26,-0)$) parabola[parabola height=5mm] ($(np.north)+(-.15,0)$);
}
\end{forest}
}
\hfill\hfill\mbox{}
\vfill
\begin{itemize}[<+->]
\item Like verb movement: Trace in ``normal'' position.
\item Percolation of information in the tree
\item Binding off nonlocal dependency
\item Constituent movement is not local, verb movement is.\\
      Hence, two different features are used ({\sc slash} vs.\ {\sc dsl}).
\end{itemize}

}



\subsection{Summary and classification}

% \outline{
% \begin{itemize}
% \item General remarks on the representational format
% \item Passive
% \item Verb position
% \item Local reordering (aka scrambling)
% \item Long distance dependencies
% \item Summary and classification
% \end{itemize}

% }


\frame{
\frametitle{Summary}

\begin{itemize}
\item Carpenter called HPSG a Frankenstein theory \citep{Mineur95a},\\
      since it was sewed together from so many other theories.
\pause
\item I would say it is a best-of:
\begin{itemize}
\pause
\item Linearization from GPSG, 
\pause
\item valence from CG, 
\pause
\item verb placement (in German) from GB, 
\pause
\item constructional patterns from CxG, \ldots
\end{itemize}

\end{itemize}

}




%      <!-- Local IspellDict: en_US-w_accents -->
