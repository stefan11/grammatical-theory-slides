\section{Organizational matters}

%% \frame{
%% \frametitle{Mailliste für alle mit Interesse an Linguistik}


%% Mailliste für Stellenangebote, Vortragsankündigungen, u.s.w.:

%% \url{https://lists.fu-berlin.de/listinfo/ling-stud}

%% Bitte mit FU-Addresse eintragen (andere Adressen ändern sich vielleicht)!

%% }

\frame{
\frametitle{Organizational matters}


\begin{itemize}
%% \item alle Teilnehmer bitte Mail in Maillisten eintragen\\
%%       zugänglich von \url{http://www.cl.uni-bremen.de/~stefan/Lehre/HPSG/}
%%   \begin{itemize}
%%   \item Mail-Liste für alle Linguistik-Studierenden
%%   \item Mail-Liste für alle Teilnehmer dieser Veranstaltung
%%   \end{itemize}
\item Please register via Moodle
\pause
\item Phone and office hours see: \url{https://hpsg.hu-berlin.de/~stefan/}
\pause
\item Complaints and suggestions:
      \begin{itemize}
      \item in person
      \item via mail 
      \item anonymously via the web:\\
            \url{https://hpsg.hu-berlin.de/~stefan/Lehre/}
      \end{itemize}
\item Please stick to the mail rules!\\
\url{https://hpsg.hu-berlin.de/~stefan/Lehre/mailregeln.html}
\end{itemize}
}



\frame{
\frametitle{Documents}


\begin{itemize}

\item Course information:\\
\usebox{\veranstaltungsurl}\\[2ex]

% Als leer definieren, wenn es keins gibt.
\usebox{\veranstaltungsbuch}

\end{itemize}
}

% \frame{
% \frametitle{General idea}


% \begin{itemize}
% \item Handouts ausdrucken, immer mitbringen und persönliche Anmerkungen einarbeiten
% \item Veranstaltungen vorbereiten
% \item Veranstaltungen unbedingt nacharbeiten!
% \item Fragen!
% \end{itemize}
% }


\frame{
\frametitle{General idea in Corona times}


\begin{enumerate}
\item Read the respective sections in the textbook.
\item Slides with spoken comments can be found in moodle.\\
      Please watch them before the lesson.
\item You can do 1 and 2 in your preferred order.
\item Use the online tasks to check whether you understand everything.
\item Use quick questions and exercises in the book.
\item Ask questions during the online sessions!
\end{enumerate}
}
