
\subtitle{Phrase structure grammars}

\huberlintitlepage[22pt]

\section{Phrase structure grammars}


\subsection{Phrase structure}

\frame[shrink=2]{
\frametitle{Phrase structure}

\smallframe

\hfill%
\begin{tabular}{@{}l@{\hspace{1cm}}l@{}}
\scalebox{.85}{%
\begin{forest}
sm edges
[S
  [NP [er;he] ]
  [NP
    [Det [das;the] ]
    [N [Buch;book] ] 
  ]
  [NP
    [Det [dem;the] ]
    [N [Mann;mann] ] 
  ]
  [V [gibt;gives] ]
]
\end{forest}} &
\scalebox{.85}{%
\begin{forest}
sm edges
[V
  [NP [er;he] ]
  [V
    [NP
      [Det [das;the] ]
      [N [Buch;book] ] ]
    [V
      [NP
        [Det [dem;the] ]
        [N [Mann;man] ] ]
      [V [gibt;gives] ] ] ] ]
\end{forest}}
\\
\\[-0.4ex]
\begin{tabular}{@{~}l@{ }l@{}}
NP & $\to$ Det, N            \\
S  & $\to$ NP, NP, NP, V  \\
\end{tabular} & \begin{tabular}{@{~}l@{ }l@{}}
NP & $\to$ Det, N  \\
V  & $\to$ NP, V\\
\end{tabular}\\
\end{tabular}
\hfill\mbox{}

\medskip
What we are after is phrase structure rules! Trees are just their visualization.\\
%
\pause%
%
Sometimes bracketed strings are used to safe space:\\
{}[\sub{S} [\sub{NP} er] [\sub{NP} [\sub{Det} das] [\sub{N} Buch]]  [\sub{NP} [\sub{Det} dem] [\sub{N} Mann]] [\sub{V} gibt]]

\handoutspace
}

\iftoggle{psgbegriffe}{
\subsection{Terminology}


\frame{
\frametitle{Node}

\vfill
\psset{xunit=5mm,yunit=5mm,nodesep=8pt}
\hfill
\begin{pspicture}(0,0)(14,7.4)
\rput(3,7){\rnode{xp}{A}}
\rput(1,4){\rnode{up}{B}}\rput(5,4){\rnode{xs1}{C}}
\rput(5,1){\rnode{vp}{D}}

\psset{angleA=-90,angleB=90,arm=0pt}
\ncdiag{xp}{up}\ncdiag{xp}{xs1}%
\ncdiag{xs1}{vp}\ncdiag{xs1}{xs2}%
\ncdiag{xs2}{wp}\ncdiag{xs2}{x}%
\ncdiag{wp}{yp}\ncdiag{wp}{ws}%

\pause

%\mode<beamer>{
\psset{linecolor=red}%radius=1em}
%}
%\pscircle(3,7){2ex}
\cnode[linewidth=1.5pt](3,7){1.7ex}{nodeA}
\pscircle[linewidth=1.5pt](1,4){1.7ex}\cnode[linewidth=1.5pt](5,4){1.7ex}{nodeC}
\cnode[linewidth=1.5pt](5,1){1.7ex}{nodeD}

\pause

\rput[l](8,7){\rnode{verz}{branching}}
\rput[l](8,6){\rnode{nverz}{n}on-branching}

%\psset{angleA=180,angleB=0,arm=0pt,arrows=->}
\only<3>{
\ncline{->}{verz}{nodeA}
}
\pause
\only<4>{
\ncline{->}{nverz}{nodeC}
}
%\psgrid
\end{pspicture}
\hfill\hfill\mbox{}
\vfill
}

\frame{

\frametitle{Mother, daughter and sister}

\vfill
\psset{xunit=5mm,yunit=5mm,nodesep=8pt}
\hspace{1cm}%
%\begin{tabular}{@{}l@{\hspace{1cm}}l@{}}
\begin{pspicture}(0,0)(7.4,7.4)
\rput(3,7){\rnode{xp}{A}}
\rput(1,4){\rnode{up}{B}}\rput(5,4){\rnode{xs1}{C}}
\rput(5,1){\rnode{vp}{D}}

\psset{angleA=-90,angleB=90,arm=0pt}
\ncdiag{xp}{up}\ncdiag{xp}{xs1}%
\ncdiag{xs1}{vp}\ncdiag{xs1}{xs2}%
\ncdiag{xs2}{wp}\ncdiag{xs2}{x}%
\ncdiag{wp}{yp}\ncdiag{wp}{ws}%

%\psgrid
\end{pspicture}
\hspace{1cm}\raisebox{3cm}{\begin{tabular}[t]{@{}l@{}}
A is mother of B and C\\
C is mother of D\\
B is sister of C\\
\end{tabular}}


Relationships like in family trees

\vfill

}

\iftoggle{einfsprachwiss-exclude}{
\frame{
\frametitle{Dominance}

\vfill
\psset{xunit=5mm,yunit=5mm,nodesep=8pt}
\hspace{1cm}
\begin{pspicture}(0,0)(7.4,7.4)
\rput(3,7){\rnode{xp}{A}}
\rput(1,4){\rnode{up}{B}}\rput(5,4){\rnode{xs1}{C}}
\rput(5,1){\rnode{vp}{D}}

\psset{angleA=-90,angleB=90,arm=0pt}
\ncdiag{xs1}{xs2}%
\ncdiag{xs2}{wp}\ncdiag{xs2}{x}%
\ncdiag{wp}{yp}\ncdiag{wp}{ws}%

\alt<2>{
\mode<beamer>{
\psset{linecolor=red}
}
\ncdiag{->}{xp}{up}\ncdiag{->}{xp}{xs1}
}{
\ncdiag{xp}{up}\ncdiag{xp}{xs1}%
}
\alt<2,4>{
\mode<beamer>{
\psset{linecolor=red}
}
\ncdiag{->}{xs1}{vp}
}{
\ncdiag{xs1}{vp}
}
%\psgrid
\end{pspicture}
\hspace{1cm}\raisebox{3cm}{\begin{tabular}[t]{@{}l@{}}
A dominates \only<2->{B, C and D}\\
\only<3->{C dominates} \only<4->{D} \\
\end{tabular}}

\bigskip

A \alert{dominates} B if and only if A is higher in the tree and \\
if there is a line from A to B that exclusively goes downwards.

\pause\pause\pause

\vfill

}

\frame{

\frametitle{Immediate dominance}

\psset{xunit=5mm,yunit=5mm,nodesep=8pt}
\hspace{1cm}
\begin{pspicture}(0,0)(7.4,7.4)
\rput(3,7){\rnode{xp}{A}}
\rput(1,4){\rnode{up}{B}}\rput(5,4){\rnode{xs1}{C}}
\rput(5,1){\rnode{vp}{D}}

\psset{angleA=-90,angleB=90,arm=0pt}
\ncdiag{xs1}{xs2}%
\ncdiag{xs2}{wp}\ncdiag{xs2}{x}%
\ncdiag{wp}{yp}\ncdiag{wp}{ws}%

\alt<2>{
\mode<beamer>{
\ncdiag[linecolor=red]{->}{xp}{up}\ncdiag[linecolor=red]{->}{xp}{xs1}
}
%\ncline{->}{xs1}{vp}
}{
\ncdiag{->}{xp}{up}\ncdiag{->}{xp}{xs1}%
}
\alt<4>{
\mode<beamer>{
\psset{linecolor=red}
}
\ncdiag{->}{xs1}{vp}
}{
\ncdiag{->}{xs1}{vp}
}
%\psgrid
\end{pspicture}
\hspace{1cm}\raisebox{3cm}{\begin{tabular}[t]{@{}l@{}}
A immedeately dominates \only<2->{B and C}\\
\only<3->{C immedeately domminates} \only<4->{D} \\
\end{tabular}}

\bigskip

A immedeately dominates B if and only if \\
A dominates B and there is no node C between A and B.

\pause\pause\pause


}


\frame{
\frametitle{Precedence}

\begin{description}[<+->]
\item[Precedence]~\\ A precedes B, if A is located to the left of B in a tree and\\
    none of these nodes dominates the other one.
\item[Immediate precedence]~\\ A precedes B and there is no element C between A and B.
\end{description}

}
}%\end{einfsprachwiss-exclude}
}%psgbegriffe


\subsection{A sample grammar}


\frame[shrink=8]{
\frametitle{Example derivation assuming flat structures}

\vfill

\bigskip
\parskip0pt
\begin{tabular}[t]{@{}l@{ }l}
\highlight{NP}<5,8> & \highlight{$\to$ Det N}<5,8>\\          
\highlight{S}<10>  & \highlight{$\to$ NP NP NP V}<10>
\end{tabular}\hspace{2cm}%
\begin{tabular}[t]{@{}l@{ }l}
\highlight{NP}<2> & \highlight{$\to$ er}<2>\\
\highlight{Det}<3>  & \highlight{$\to$ das}<3>\\
\highlight{Det}<6>  & \highlight{$\to$ dem}<6>\\
\end{tabular}\hspace{8mm}
\begin{tabular}[t]{@{}l@{ }l}
\highlight{N}<4> & \highlight{$\to$ Buch}<4>\\
\highlight{N}<7> & \highlight{$\to$ Kind}<7>\\
\highlight{V}<9> & \highlight{$\to$ gibt}<9>\\
\end{tabular}
\vfill

\begin{tabular}{@{}llllll@{\hspace{2.5cm}}l}
er            & das          & Buch          & dem          & Kind & gibt                \pause\\
\highlight{NP}<2> & das          & Buch          & dem          & Kind & gibt & \only<handout>{NP $\to$ er}  \pause\\
NP            & \highlight{Det}<3> & Buch          & dem          & Kind & gibt & \only<handout>{Det $\to$ das}  \pause\\
NP            & Det            & \highlight{N}<4>  & dem          & Kind & gibt & \only<handout>{N $\to$ Buch} \pause\\
NP            &              & \highlight{NP}<5> & dem          & Kind & gibt & \only<handout>{NP $\to$ Det N}\pause\\
NP            &              & NP            & \highlight{Det}<6> & Kind & gibt & \only<handout>{Det $\to$ dem}  \pause\\
NP            &              & NP            & Det            & \highlight{N}<7>    & gibt & \only<handout>{N $\to$ Kind} \pause\\
NP            &              & NP            &              & \highlight{NP}<8>       & gibt & \only<handout>{NP $\to$ Det N}\pause\\
NP            &              & NP            &              & NP       & \highlight{V}<9>   & \only<handout>{V $\to$ gibt}  \pause\\
              &              &               &              &      & \highlight{S}<10>   & \only<handout>{S $\to$ NP NP NP V}\\
\end{tabular}

\vfill
}

\frame{
\frametitle{Sentences described by the grammar}

\begin{itemize}
\item The grammar is not precise enough:\\
\begin{tabular}{@{}l@{ }l}
NP & $\to$ Det N\\
S  & $\to$ NP NP NP V\\
\end{tabular}
\eal
\ex[]{
\gll er das Buch dem Kind  gibt\\
     he the book the child gives\\
}
\ex[*]{
\gll ich das Buch dem Kind gibt\\
     I   the book the child give\\
\pause\\
(Subject verb agreement \emph{ich}, \emph{gibt})}
\pause
\ex[*]{
\gll er das Buch das Kind gibt\\
     he the book the child gives\\\pause\\
(case requirement of the verb, \emph{gibt} requires dative)
}
\pause
\ex[*]{
\gll er den Buch dem Kind gibt\\
     he the book the child gives\\
\pause\\
(determinator noun agreement \emph{den}, \emph{Buch})
}
\zl
\end{itemize}

}

\begin{frame}[fragile]
\frametitle{Do try this at home!}

You can actually play with such grammars.
\begin{itemize}
\item Go to \url{https://swish.swi-prolog.org/}.
\item Click ``Program''.
\item Enter:
\begin{verbatim}
s --> np, v, np, np.
np --> det, n.
np --> [er].
det --> [das].
det --> [dem].
n --> [buch].
n --> [kind].
v --> [gibt].
\end{verbatim}
\item Type in the following into the right lower box: \texttt{s([er,gibt,das,buch,dem,kind],[]).}
\item If there appears a ``true'' in the box above this box, celebrate.
\end{itemize}

\end{frame}

\frame{
\frametitle{A generative grammar}

\begin{itemize}
\item The grammar you just entered can generate sentences.
\pause
\item You may test which sentences it generates by typing in:
\texttt{s([X],[]),print(X),nl,fail.}

\pause
\item \texttt{s([X],[])} asks Prolog to come up with an X that is an ``s''.
\pause
\item \texttt{print(X),nl} prints the X and a newline and
\pause
\item \texttt{fail} tells Prolog that we are not happy and that it should try again.
\pause
\item It keeps trying till there are no further solutions and then fails.
\pause
\item Some grammars generate infinitely many Xes. So this process would never end (unless the
  computer runs out of memory \ldots).

\end{itemize}

}

% geht hier nicht, weil das von anderen eingebunden wird
%\exewidth{\exnrfont(12)}

\frame{
\frametitle{Subject verb agreement (I)}


\begin{itemize}
\item Agreement in person (1, 2, 3) and number (sg, pl)
\eal
\ex Ich schlafe. (1, sg)
\ex Du schläfst.  (2, sg)
\ex Er schläft.  (3, sg)
\ex Wir schlafen. (1, pl)
\ex Ihr schlaft.  (2, pl)
\ex Sie schlafen. (3,pl)
\zl
\item How can we express this in rules?
\end{itemize}

}

\frame{
\frametitle{Subject verb agreement (II)}

\begin{itemize}
\item We make the symbols more informative.\\
      Instead of S $\to$ NP NP NP V we use:\\[2ex]
\begin{tabular}{@{}l@{ }l}
S  & $\to$ NP\_1\_sg NP NP V\_1\_sg\\
S  & $\to$ NP\_2\_sg NP NP V\_2\_sg\\
S  & $\to$ NP\_3\_sg NP NP V\_3\_sg\\
S  & $\to$ NP\_1\_pl NP NP V\_1\_pl\\
S  & $\to$ NP\_2\_pl NP NP V\_2\_pl\\
S  & $\to$ NP\_3\_pl NP NP V\_3\_pl\\
\end{tabular}

\item six symbols for nominal phrases, six for verbs
\item six rules instead of one
\end{itemize}

}

\frame{
\frametitle{Case assignment by the verb}

\begin{itemize}
\item Case must be part of the symbols used in the rules:
\begin{tabular}{@{}l@{ }l}
S  & $\to$ NP\_1\_sg\_nom NP\_dat NP\_acc V\_1\_sg\_ditransitiv\\
S  & $\to$ NP\_2\_sg\_nom NP\_dat NP\_acc V\_2\_sg\_ditransitiv\\
S  & $\to$ NP\_3\_sg\_nom NP\_dat NP\_acc V\_3\_sg\_ditransitiv\\
S  & $\to$ NP\_1\_pl\_nom NP\_dat NP\_acc V\_1\_pl\_ditransitiv\\
S  & $\to$ NP\_2\_pl\_nom NP\_dat NP\_acc V\_2\_pl\_ditransitiv\\
S  & $\to$ NP\_3\_pl\_nom NP\_dat NP\_acc V\_3\_pl\_ditransitiv\\
\end{tabular}
\item 3 * 2 * 4 = 24 new categories for NPs in total
\item 3 * 2 * x  categories for V (x = number of attested valence patterns)
\end{itemize}

}


\frame[shrink=15]{
\frametitle{Determinator noun agreement}

\begin{itemize}
\item There is agreement in gender (fem, mas, neu), number (sg, pl) and\\
      case (nom, gen, dat, acc)
\eal
\ex der Mann `the man', die Frau `the woman', das Kind `the child'  (gender)
\ex das Buch `the book', die Bücher `the books' (number)
\ex des Buches `the.\GEN book.\GEN', dem Buch `the.\DAT book' (case)
\zl
\pause
\item instead of NP $\to$ Det N we have\\[2ex]
\resizebox{\linewidth}{!}{
\begin{tabular}{@{}l@{ }l@{\hspace{4mm}}l@{ }l}
NP\_3\_sg\_nom  & $\to$ Det\_fem\_sg\_nom N\_fem\_sg\_nom & NP\_gen  & $\to$ Det\_fem\_sg\_gen N\_fem\_sg\_gen\\
NP\_3\_sg\_nom  & $\to$ Det\_mas\_sg\_nom N\_mas\_sg\_nom & NP\_gen  & $\to$ Det\_mas\_sg\_gen N\_mas\_sg\_gen\\
NP\_3\_sg\_nom  & $\to$ Det\_neu\_sg\_nom N\_neu\_sg\_nom & NP\_gen  & $\to$ Det\_neu\_sg\_gen N\_neu\_sg\_gen\\
NP\_3\_pl\_nom  & $\to$ Det\_fem\_pl\_nom N\_fem\_pl\_nom & NP\_gen  & $\to$ Det\_fem\_pl\_gen N\_fem\_pl\_gen\\
NP\_3\_pl\_nom  & $\to$ Det\_mas\_pl\_nom N\_mas\_pl\_nom & NP\_gen  & $\to$ Det\_mas\_pl\_gen N\_mas\_pl\_gen\\
NP\_3\_pl\_nom  & $\to$ Det\_neu\_pl\_nom N\_neu\_pl\_nom & NP\_gen  & $\to$ Det\_neu\_pl\_gen N\_neu\_pl\_gen\\[2mm]


\ldots & \hspaceThis{$\to$} dative                                                            & \ldots & \hspaceThis{$\to$} accusative\\[2mm]
\end{tabular}
}
\item 24 symbols for determiners, 24 symbols for nouns
\item 24 rules instead of one
\end{itemize}
}

\subsection{Using features to augment phrase structure grammars}


\frame{

\frametitle{Problems of simple phrase structure grammars}

\begin{itemize}
\item Gernalisations are not captured.
\item neither in rules nor in category symbols
      \begin{itemize}
      \item Where can an NP or an NP\_nom be placed?\\
            The only question we can ask is: Where can I put an NP\_3\_sg\_nom?
      \item Commonalities between rules are not obvous.
      \end{itemize}
\pause
\item Solution: features with values and identity of values\\
      Category symbol: NP feature: Per, Num, Cas, \ldots\\

We get rules like the following:\\

\begin{tabular}{@{}l@{ }l}
NP(3,sg,nom)  & $\to$ Det(fem,sg,nom) N(fem,sg,nom)\\
NP(3,sg,nom)  & $\to$ Det(mas,sg,nom) N(mas,sg,nom)\\
\end{tabular}
\end{itemize}
}


\frame{
\frametitle{Features and rule schemata (I)}

\begin{itemize}
\item Rules with specific values can be generalized to rule schemata:

\medskip

\begin{tabular}{@{}l@{ }l@{ }l}
NP(\blau<3>{3},\blau<2>{Num},\blau<2>{Cas}) & $\to$ & Det(\gruen<2>{Gen},\blau<2>{Num},\blau<2>{Cas}) N(\gruen<2>{Gen},\blau<2>{Num},\blau<2>{Cas})\\
\end{tabular}
\pause
\item Actual Gen, Num and Cas values do not matter\\
      as long as they are identical.
\pause
\item The value of the person feature (first slot in NP(3,Num,Cas))\\
      is fixed by the rule: 3.
\end{itemize}
}


\frame{
\frametitle{Features and rule schemata (II)}

\begin{itemize}
\item Rules with specific values can be generalized into rule schemata:

\medskip
\begin{tabular}{@{}l@{ }l@{ }l}
NP({3},{Num},{Cas}) & $\to$ & Det(Gen,{Num},{Cas}) N(Gen,{Num},{Cas})\\
S  & $\to$ & NP(\blau<1>{Per1},\blau<1>{Num1},\blau<3>{nom})\\
   &       & NP(Per2,Num2,\blau<3>{dat})\\
   &       & NP(Per3,Num3,\blau<3>{acc})\\
   &       & V(\blau<1>{Per1},\blau<1>{Num1})\\\\
\end{tabular}
\item Per1 and Num1 value of verb and subject are identical.
\pause
\item The values of other NPs do not matter.\\
      (Notation for irrelevant values: `\_')
\pause
\item Case values of the NPs are fixed in the second rule.
\end{itemize}

}

%% Kommt dann in Theorie anders, deshalb hier raus
%% \frame{

%% \small
%% \frametitle{Bündelung von Merkmalen}

%% \begin{itemize}
%% \item Kann es Regeln geben, in denen nur der Per-Wert oder nur der Num-Wert identisch sein muß?\\[2ex]

%% \begin{tabular}{@{}l@{ }l@{ }l}
%% S  & $\to$ & NP(Per1,Num1,nom)\\
%%    &       & NP(Per2,Num2,dat)\\
%%    &       & NP(Per3,Num3,akk)\\
%%    &       & V(Per1,Num1)\\\\
%% \end{tabular}
%% \pause
%% \item Gruppierung von Information $\to$ stärkere Generalisierung, stärkere Aussage\\[2ex]

%% \begin{tabular}{@{}l@{ }l@{ }l}
%% S  & $\to$ & NP(Agr1,nom)\\
%%    &       & NP(Agr2,dat)\\
%%    &       & NP(Agr3,akk)\\
%%    &       & V(Agr1)\\\\
%% \end{tabular}

%% wobei Agr ein Merkmal mit komplexen Wert ist: \zb agr(1,sg)
%% \end{itemize}


%% }


\section{Home work}

\frame[shrink=5]{
\frametitle{Home work}

\begin{enumerate}
\item Write a phrase structure grammar that can analyze at least the sentences in (\mex{1})
      but excludes the sequences in (\mex{2}).
      \eal
      \ex[]{
      \gll Der Mann hilft dem Kind.\\
           the man  helps the child\\
      }
      \ex[]{
      \gll Er gibt ihr das Buch.\\
           he gives her the book\\
      }
      \ex[]{
      \gll Er wartet auf ein Wunder.\\
           he waits  for a   miracle\\
      }
%       \ex[]{
%       Er wartet neben dem Bushäuschen auf ein Wunder.
%       }
      \zl
      \eal
      \ex[*]{
      \gll Der Mann hilft er.\\
           the man  helps he\\
      }
      \ex[*]{
      \gll Er gibt ihr den Buch.\\
           he gives her the book\\
      }
      \zl
      The result should be one grammar for all grammatical sentences, not one for each sentence.

      You may use Prolog to make sure your grammar actually works:
      \url{https://swish.swi-prolog.org} 
      See \url{https://en.wikipedia.org/wiki/Definite_clause_grammar} for the syntax of Definite
      Clause Grammars.
\end{enumerate}

}



%      <!-- Local IspellDict: en_US-w_accents -->
