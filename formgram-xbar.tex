%% -*- coding:utf-8 -*-

\subsection{\xbar Theory}

\subtitle{\xbar Theory}

\huberlintitlepage[22pt]

\frame{
\frametitle{Reading material}

Please read \citew[Section~2.5]{MuellerGT-Eng}.


}


\subsubsection{Nominal phrases}
\label{sec-psg-np}

\frame[shrink]{
\frametitle{Nominal phrases}

\begin{itemize}
\item Until now NP $\to$ Det N, but noun phrases can be much more complex:
\eal
\ex 
\gll ein Buch\\
     a   book\\
\ex 
\gll ein Buch, das  wir kennen\\
     a   book  that we  know\\
\ex 
\gll ein Buch aus  Japan\\
     a   book from Japan\\
\ex 
\gll ein interessantes Buch\\
     an   interesting   book\\
\ex 
\gll ein Buch aus  Japan, das  wir kennen\\
     a   book from Japan  that we  know\\
\ex 
\gll ein interessantes Buch aus  Japan\\
     an  interesting   book from Japan\\
\ex 
\gll ein interessantes Buch, das  wir kennen\\
     an  interesting   book  that we  know\\
\ex 
\gll ein interessantes Buch aus  Japan, das  wir kennen\\
     an  interesting   book from Japan  that we  know\\
\zl

The additional constituents in (\mex{0}) are adjuncts.

\end{itemize}

}

\frame{
\frametitle{Adjectives in NPs}

\begin{itemize}
\item Suggestion:
\eal
\ex NP $\to$ Det N
\ex NP $\to$ Det A N
\zl
\pause
\item What about (\mex{1})?
\ea
\gll alle weiteren schlagkräftigen Argumente\\
     all further strong arguments\\
\glt `all other strong arguments'
\z
\pause
\item We need a rule like (\mex{1}) for (\mex{0}):
\ea 
NP $\to$ Det A A N
\z
\pause
\item But we do not want to state a limit on how many adjectives there may be:
\ea 
NP $\to$ Det A* N
\z
\end{itemize}

}

\frame{
\frametitle{Adjectives in NPs}

\begin{itemize}

\item Problem: adj \& noun do not form constituent in structures licensed by (\mex{1}).
\ea 
NP $\to$ Det A* N
\z
But constituency tests suggest that A + N is a constituent:
\ea
\gll alle [[großen Seeelefanten] und [grauen Eichhörnchen]]\\
     all  \spacebr{}\spacebr{}big elephant.seals and  \spacebr{}grey squirrels\\
\glt `all the big elephant seals and grey squirrels'	 
\z

\end{itemize}

}

\frame{
\frametitle{Adjective + noun as constituent}

\begin{itemize}
\item The following rule is better suited:
\eal
\ex NP $\to$ Det \nbar
\ex \nbar $\to$ A \nbar
\ex \nbar $\to$ N
\zl


\hfill%
\scalebox{.65}{%
\begin{forest}
sm edges
[NP
   [Det [ein;a] ]
   [\nbar
      [N [Eichhörnchen;squirrel] ] ] ]
\end{forest}}
%
\hfill
\scalebox{.65}{%
\begin{forest}
sm edges
[NP
   [Det [ein;a] ]
   [\nbar
      [A [graues;grey] ]
      [\nbar
        [N [Eichhörnchen;squirrel] ] ] ] ]
\end{forest}
}
%
\hfill
\scalebox{.65}{%
\begin{forest}
sm edges
[NP
  [Det [ein;a] ]
    [\nbar
    [A [großes;big] ]
       [\nbar
       [A [graues;grey] ]
         [\nbar
         [N [Eichhörnchen;squirrel] ] ] ] ] ]
\end{forest}
}
\hfill\mbox{}
%
\end{itemize}

}





\frame{
\frametitle{Other adjuncts}


\begin{itemize}
\item Other adjuncts work analogously:
\eal
\ex \nbar $\to$ \nbar PP
\ex \nbar $\to$ \nbar relative\_clause
\zl
\pause
\ex All given determiner-adjective-noun combinations given so far can be analyzed with these few rules.

\end{itemize}

}

\frame{
\frametitle{Complements}


\begin{itemize}
\item Until now, \nbar consists of a single noun only,\\
      but some nouns allow arguments in addition to adjuncts.
\eal
\ex 
\gll der Vater von Peter\\
	 the father of Peter\\
\glt `Peter's father'
\ex 
\gll das Bild vom Gleimtunnel\\
	 the picture of.the Gleimtunnel\\
\glt `the picture of the Gleimtunnel'
\ex 
\gll das Kommen der Installateurin\\
	 the coming of.the plumber\\
\glt `the plumber's visit'
\zl

\pause
\item Therefore:
\ea
\nbar $\to$ N PP
\z

\end{itemize}

}

\frame{
\frametitle{Complements (and adjuncts)}

\hfill

\centerfit{%
\begin{forest}
sm edges
[NP
 [Det [das;the] ]
 [\nbar
   [N [Bild;picture] ]
   [PP [vom Gleimtunnel;of.the Gleimtunnel,roof ] ] ] ]
\end{forest}%
\hspace{2em}%
\begin{forest}
sm edges
[NP
  [Det [das;the] ]
  [\nbar
    [\nbar
      [N [Bild;picture] ]
      [PP [vom Gleimtunnel;of.the Gleimtunnel,roof ] ] ] 
    [PP [im Gropiusbau;in.the Gropiusbau,roof ] ] ] ]
\end{forest}}


}

\frame{
\frametitle{Missing noun (adjuncts present)}

\savespace
\begin{itemize}
\item Noun is missing but adjuncts are present:
\eal
\ex 
\gll ein interessantes \_\\
     an  interesting\\
\glt `an interesting one'
\ex 
\gll ein neues interessantes \_\\
     a   new   interesting\\
\glt `a new interesting one'

\ex 
\gll ein interessantes \_ aus  Japan\\
     an  interesting   {} from Japan\\
\glt `an interesting one from Japan'
\ex 
\gll ein interessantes \_, das  wir kennen\\
     an  interesting   {}  that we  know\\
\glt `an interesting one that we know'
\zl
\end{itemize}
}

\frame[shrink=20]{
\frametitle{Missing noun (complement present)}

\savespace
\begin{itemize}

\item noun missing, but a complement of the noun is present:
\eal
\ex 
\gll (Nein, nicht der Vater von Klaus), der \_ von Peter war gemeint.\\
	\spacebr{}no not the father of Klaus the {} of Peter was meant\\
\glt `No, it wasn't the father of Klaus, but rather the one of Peter that was meant.'
\ex 
\gll (Nein, nicht das Bild von der Stadtautobahn), das \_ vom Gleimtunnel war beeindruckend.\\
	 \spacebr{}no not the picture of the motorway the {} of.the Gleimtunnel was impressive\\
\glt `No, it wasn't the picture of the motorway, but rather the one of the Gleimtunnel that was impressive.'
\ex 
\gll (Nein, nicht das Kommen des Tischlers), das \_ der Installateurin ist wichtig.\\
	 \spacebr{}no not the coming of.the carpenter the {} of.the plumber is important\\
\glt `No, it isn't the visit of the carpenter, but rather the visit of the plumber that is important.'
\zl
\pause
\item PSG: \alert{Epsilon production}
\pause
\item Notation:
\eal
\ex N $\to$
\ex N $\to$ $\epsilon$
\zl 

\pause
\item Rules in (\mex{0}) = empty boxes with the same label as boxes containing normal nouns.

\end{itemize}
}

\frame{
\frametitle{Analysis with empty noun}

\hfill
\begin{forest}
sm edges
[NP
  [Det [ein;an] ]
  [\nbar
    [A [interessantes;interesting] ]
    [\nbar
      [N [\trace ] ] ] ] ]
\end{forest}
\hfill
\begin{forest}
sm edges
[NP
  [Det [das;the] ]
  [\nbar
    [N [\trace] ]
    [PP [vom Gleimtunnel;of.the Gleimtunnel, roof] ] ] ]
\end{forest}
\hfill%
\mbox{}

}


\frame{
\frametitle{Missing determiners: Plural}

\begin{itemize}
\item Determiners can be dropped as well.

Plural:
\eal
\ex 
\gll Bücher\\
     books\\
\ex 
\gll Bücher, die  wir kennen\\
     books   that we  know\\
\ex 
\gll interessante Bücher\\
     interesting  books\\
\ex 
\gll interessante Bücher, die  wir kennen\\
     interesting  books   that we know\\
\zl

\end{itemize}

}


\frame{
\frametitle{Missing determiners: Mass nouns}

\begin{itemize}
\item For mass nouns dropping is possible in the singular as well:
\eal
\ex 
\gll Getreide\\
     grain\\
\ex 
\gll Getreide, das gerade gemahlen wurde\\
	 grain that just ground was\\
\glt `grain that has just been ground'
\ex 
\gll frisches Getreide\\
	 fresh grain\\
\ex 
\gll frisches Getreide, das gerade gemahlen wurde\\
	 fresh grain that just ground was\\
\glt `fresh grain that has just been ground'
\zl

\end{itemize}

}


\frame{
\frametitle{Missing determiners: The Structure}

\vfill

\centering
\begin{forest}
sm edges
[NP
  [Det [\trace] ]
  [\nbar
    [N [Bücher;books] ] ] ]
\end{forest}
\vfill

}

\frame{
\frametitle{Missing determiners and missing nouns}

Determiners and nouns can even be omitted simultaneously:
\eal
\ex 
\gll Ich lese interessante.\\
     I   read interesting\\
\glt `I read interesting ones.'
\ex 
\gll Dort drüben steht frisches, das gerade gemahlen wurde.\\
	 there over stands fresh that just ground was\\
\glt `Over there is some fresh (grain) that has just been ground.'
\zl
}


\frame{
\frametitle{Missing determiners and missing nouns: The structure}

\vfill
\centerline{%
\begin{forest}
sm edges
[NP
  [Det [\trace] ]
  [\nbar
    [A [interessante;interesting] ]
    [\nbar
      [N [\trace] ] ] ] ]
\end{forest}}

\vfill

}




% %%%%%%%%%%%%%%%%%%%%%%%%%%%%%%%%%%%%%%%%%%%%%%%%%%%%%%%%%%%%%%%%%%%%%%%%%%%%%%%%%%%%%%%%%%%%%%%%%%

\subsubsection{Adjective phrases}

\frame{
\frametitle{Adjective phrases}

\begin{itemize}
\item Until now simple adjectives like \emph{klug} `smart' only.
\pause
\item But adjective phrases can be very complex:
\eal
\ex
\gll der seiner Frau treue Mann\\
	 the his.\DAT{} wife faithful man\\
\glt `the man faithful to his wife'
\ex 
\gll der auf seine Tochter stolze Mann\\
	 the on his.\ACC{} daughter proud man\\
\glt `the man proud of his daughter'
\ex 
\gll der seine Frau liebende Mann\\
	 the his.\ACC{} woman loving man\\
\glt `the man who loves his wife'
\ex 
\gll der von seiner Frau geliebte Mann\\
     the by his.\DAT{} wife loved man\\
\glt `the man loved by his wife'	 
\zl

\end{itemize}

}

\frame{
\frametitle{Adjective phrases}

\begin{itemize}
\item 
\ea
\gll der auf seine Tochter stolze Mann\\
	 the on his.\ACC{} daughter proud man\\
\glt `the man proud of his daughter'
\z
\item We have to adapt the rule for attributive adjectival modifiers:
\ea
\nbar $\to$ AP \nbar
\z
\pause
\item
Rules for AP:
\eal
\ex AP $\to$ NP A
\ex AP $\to$ PP A
\ex AP $\to$ A
\zl

\end{itemize}

}



\subsubsection{Prepositional phrases}

\frame[shrink=10]{
\frametitle{Prepositional phrases}

\savespace
\begin{itemize}
\item The syntax of PPs is relatively straight-forward. First attempt:
\ea
PP $\to$ P NP
\z
\pause
\item But PPs can be augmented by measurement phrases \citep[\S 1300]{Duden2005-Authors}:
\eal
\ex
\gll {}[[Einen Schritt] vor dem Abgrund] blieb er stehen.\\
	 {}\spacebr{}\spacebr{}one step before the abyss remained he stand\\
\glt `He stopped one step in front of the abyss.'
\ex 
\gll {}[[Kurz] nach dem Start] fiel die Klimaanlage aus.\\
	 {}\spacebr{}\spacebr{}shortly after the take.off fell the air.conditioning out\\
\glt `Shortly after take off, the air conditioning stopped working.'
\ex 
\gll {}[[Schräg] hinter der Scheune] ist ein Weiher.\\
	 {}\spacebr{}\spacebr{}diagonally behind the barn is a pond\\
\glt `There is a pond diagonally across from the barn.'
\ex 
\gll {}[[Mitten] im Urwald] stießen die Forscher auf einen alten Tempel.\\
	 {}\spacebr{}\spacebr{}middle in.the jungle stumbled the researchers on an old temple\\
\glt `In the middle of the jungle, the researches came across an old temple.'
\zl

\end{itemize}

}

\frame{
\frametitle{Prepositional phrases: The rules}

\savespace
\begin{itemize}
\item 
\ea
\gll {}[[Einen Schritt] vor dem Abgrund]\\
	 {}\spacebr{}\spacebr{}one step before the abyss \\
\glt `one step in front of the abyss'
\z

\eal
\ex PP $\to$ NP \pbar
\ex PP $\to$ AP \pbar
\ex PP $\to$ \pbar\label{Regel-PP-P}
\ex \pbar $\to$ P NP
\zl
\end{itemize}

}



\frame{
\frametitle{Prepositional phrases: The structure}

\vfill
\hfill
\begin{forest}
sm edges
[PP
  [\pbar
    [P [vor;before] ]
    [NP [dem Abgrund;the abyss, roof] ] ] ]
\end{forest}
\hfill
\begin{forest}
sm edges
[PP
  [AP [kurz;shortly,roof] ]
  [\pbar
    [P [vor;before] ]
    [NP [dem Abgrund;the abyss,roof] ] ] ]
\end{forest}
\hfill
\mbox{}
\vfill

}



\subsubsection{\xbar rules}
\label{sec-xbar}

\frame{
\frametitle{Generalization over rules}

\begin{itemize}
\item head + complement = intermediate level:
\eal
\ex \nbar $\to$ N PP
\ex \pbar $\to$ P NP
\zl
\pause
\item intermediate level + further constituent = maximal projection
\eal
\ex NP $\to$ Det \nbar
\ex PP $\to$ NP \pbar
\zl
\pause
\item parallel structures for English AP and VP as well
\end{itemize}

}


\frame{
\frametitle{English adjective phrases}

\begin{exe}
\ex Kim and Sandy are
\begin{xlist}
\ex proud.
\ex very proud.
\ex proud of their child.
\ex very proud of their child.
\end{xlist}
\end{exe}

\pause

\eal
\ex AP $\to$ \abar
\ex AP $\to$ Adv \abar
\ex \abar $\to$ A PP
\ex \abar $\to$ A
\zl

\pause
(\citet[Section~13.1.2]{MuellerGT-Eng}: Does not work for German.)

}


\frame{
\frametitle{English adjective phrases: The structure}


\eal
\ex AP $\to$ \abar
\ex AP $\to$ AdvP \abar
\ex \abar $\to$ A PP
\ex \abar $\to$ A
\zl


\hfill
\begin{forest}
sm edges
[AP
  [\abar
    [A [proud] ] ] ]
\end{forest}
\hfill
\begin{forest}
sm edges
[AP
  [AdvP [very] ]
  [\abar
    [A [proud] ] ] ]
\end{forest}
\hfill
\begin{forest}
sm edges
[AP
  [\abar
    [A [proud] ]
    [PP [of their child,roof] ] ] ]
\end{forest}
\hfill
\begin{forest}
sm edges
[AP
  [AdvP [very] ]
  [\abar
    [A [proud] ]
    [PP [of their child,roof] ] ] ]
\end{forest}
\hfill
\mbox{}
\hfill
\mbox{}

}

\frame{
\frametitle{Further abstraction}

\begin{itemize}
\item We saw that abstraction over case and gender values is possible (variables in rule schemata).

\ea
NP({3},{Num},{Cas}) $\to$ D(Gen,{Num},{Cas}), N(Gen,{Num},{Cas})
\z

\pause
\item Similarly we can abstract over the part of speech.\\
      Instead of AP, NP, PP, VP, we write XP.
\pause
\item Instead of (\mex{1}), we write (\mex{2}):
\eal
\ex PP $\to$ \pbar
\ex AP $\to$ \abar
\zl
\ea
XP $\to$ \xbar
\z
\end{itemize}


}


\frame{
\frametitle{\xbar Theory: Assumptions}

Phrases have at least three levels:
\begin{itemize}
\item X$^0$ = head
\item X$'$ = intermediate level (= \xbar, pronounced X bar; $\to$ name of the scehma) 
\item XP = highest node (=~X$''$ = $\overline{\overline{\mbox{X}}}$), also called \emph{maximal projection}
\end{itemize}
%Neuere Analysen $\to$ teilweise Verzicht auf nichtverzweigende X$'$-Knoten
\nocite{Muysken82a}


}

\frame[shrink]{
\frametitle{Minimal and maximal expansion of phrases}

\bigskip

\small\hfill
\begin{forest}
sm edges
[XP
  [\xbar [X] ] ]
\end{forest}
\hfill
\begin{forest}
%where n children=0{}{},
%sm edges
%for tree={parent anchor=south, child anchor=north,align=center,base=bottom}
[XP
  [specifier]
  [\xbar
    [adjunct]
    [\xbar
      [complement] [X] ] ] ]
\end{forest}
\hfill\mbox{}


\begin{itemize}
\item Adjuncts are optional\\
$\to$ X$'$ with adjunct daughter may be missing.
\pause
\item Some categories do not have a specifier or it is optional (\eg A).\\
%(Zusätzliche Regel nötig $\overline{\overline{\mbox{X}}} \rightarrow \xbar$)
\pause
\item Sometimes in addition adjunction to XP and head adjunction to X.
\end{itemize}

}

\frame{
\frametitle{\xbar Theory: Rules following \citew{Jackendoff77a}}\nocite{KP90a}\nocite{Pullum85a}



\oneline{%
\begin{tabular}[t]{@{}l@{\hspace{5mm}}l@{\hspace{5mm}}l@{}}
\xbar\mbox{ rule} & \mbox{with specific categories} & \mbox{example strings}\\[2mm]
$\overline{\overline{\mbox{X}}} \rightarrow \overline{\overline{\mbox{specifier}}}$~~\xbar &
$\overline{\overline{\mbox{N}}} \rightarrow \overline{\overline{\mbox{DET}}}$~~\nbar & \mbox{the [picture of Paris]} \\
$\xbar \rightarrow$ \xbar~~$\overline{\overline{\mbox{adjunct}}}$            & \nbar $\rightarrow$ \nbar~~$\overline{\overline{\mbox{REL\_CLAUSE}}}$ & \mbox{[picture of Paris]}\\
                            &                                              & \mbox{[that everybody knows]}\\
\xbar $\rightarrow \overline{\overline{\mbox{adjunct}}}$~~\xbar            & \nbar $\rightarrow \overline{\overline{\mbox{A}}}$~~\nbar & \mbox{beautiful [picture of Paris]}\\
\xbar $\rightarrow$ \mbox{X}~~$\overline{\overline{\mbox{complement}}}*$   & \nbar $\rightarrow$ \mbox{N}~~$\overline{\overline{\mbox{P}}}$ & \mbox{picture [of Paris]}\\
\end{tabular}}

X stands for some arbitrary category, X is the head,\\
`*' stands for arbitrarily many repretitions

\medskip
X may appear in any position in the right-hand side of the rule.

}



%      <!-- Local IspellDict: en_US-w_accents -->
